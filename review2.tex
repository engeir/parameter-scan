\documentclass{reviewresponse}

\usepackage{caption,enumitem,hyperref}
\usepackage{apacite}
\let\cite\shortcite %xx So get et al. with three authors the first time.
\let\citeA\shortciteA %xx Ditto.
\bibliographystyle{apacite}%
\DeclareCaptionFont{gray}{\color{gray}}%
\captionsetup{ labelfont={gray}, textfont={gray} }
\usepackage{siunitx}

\usepackage[version=4]{mhchem}

\definecolor{colorevalfg}{RGB}{70 0 87}% rgb(70 0 87)
\definecolor{colorevalbg}{RGB}{243 221 249}% rgb(243 221 249)
\definecolor{colorevalframe}{RGB}{124 0 155}% rgb(124 0 155)
\definecolor{colorcommentbg}{HTML}{e0f0f6}% #e0f0f6
\definecolor{colorcommentresolved}{RGB}{155 0 0}

\newenvironment{evaluation}{%
  \begin{tcolorbox}[attach title to upper, title={Evaluations}, after title={.\enskip},
    fonttitle={\bfseries}, coltitle={colorevalfg}, colback={colorevalbg},
    colframe={colorevalframe},]
  }{
  \end{tcolorbox}
}

% Check mark list
\usepackage{amssymb}
\newlist{todolist}{itemize}{2}
\setlist[todolist]{label=$\square$}
\usepackage{pifont}
\newcommand{\cmark}{\ding{51}}
%
\newcommand{\xmark}{\ding{55}}
%
\newcommand{\done}{\rlap{\raisebox{0.3ex}{\hspace{0.4ex}\small \cmark}}$\square$}
\newcommand{\wontfix}{\rlap{\raisebox{0.3ex}{\hspace{0.4ex}\small \xmark}}$\square$}

\title{Radiative forcing by supereruptions}
\author{Eirik Rolland Enger, Rune Graversen, Audun Theodorsen}
\journal{Journal of Geophysical Research --- Atmospheres} \manuscript{\#2024JD041098R}
\editorname{Yafang Cheng}

% \affiliation{1}{UiT The Arctic University of Norway, Tromsø, Norway}
% \affiliation{2}{Norwegian Meteorological Institute, Troms\o, Norway}

\begin{document}

  \maketitle

  \begin{itemize}
    \item
    \emph{Stuff to-do}
  \end{itemize}

  \editor % Response to editor
  \subsection*{Associate Editor Evaluations}
  \begin{evaluation}
    \begin{itemize}[leftmargin=4.5cm,noitemsep]
      \item
      [\textbf{Accurate Key Points:}] No
    \end{itemize}
  \end{evaluation}
  \begin{generalcomment}
    Thank you for submitting ``Radiative forcing by supereruptions'' [Paper
      \#2024JD041098R] to Journal of Geophysical Research - Atmospheres. I have received
    3 reviews of your manuscript, which are included below and/or attached. As you can
    see, the reviews indicate that major revisions are needed before we can consider
    proceeding with your paper. I am therefore returning the paper to you so that you
    can make the necessary changes.
  \end{generalcomment}
  \begin{revresponse}[We appreciate your handling of the review process.]
    According to the reviewers' comments, we have checked our manuscript and addressed
    them in the following way:
    \begin{enumerate}
      \item
      We added content.
      \item
      We removed our wrong statements in Section~I.
    \end{enumerate}
  \end{revresponse}
  \begin{concludingresponse}[to the Editor]
    Thank you for your valuable comments on our manuscript. We have done our best to
    incorporate changes to reflect the suggestions, which allowed us to improve the
    quality of our work.
  \end{concludingresponse}

  % Reviewer 3
  \reviewer[3]

  They want:
  \begin{itemize}
    \item
    Overall goal is still unclear
    \item
    Make simulations more clear (how many in ensembles for example)
    \item
    Why this (heavy) model specifically?
    \item
    Should ERF and RF be mixed?
    \item
    Why are some papers are used in some figs, but others not?
    \item
    Remove geoengineering results from the figure and only keep it in the discussion
    \item
    Why are certain behaviours modelled? (timing of AOD and ERF)
    \item
    Discuss the caveats of the model study
    \item
    Why include the outdated Jones et al. 2005?
    \item
    Adjust the title
  \end{itemize}

  \subsection*{Reviewer \#3 Evaluations}

  \begin{evaluation}
    \begin{itemize}[leftmargin=4.5cm,noitemsep]
      \item
      [\textbf{Recommendation}] Return to author for major revisions
      \item
      [\textbf{Significant}] The paper has some unclear or incomplete reasoning but will
      likely be a significant contribution with revision and clarification.
      \item
      [\textbf{Supported}] Mostly yes, but some further information and/or data are
      needed.
      \item
      [\textbf{Referencing}] Mostly yes, but some additions are necessary.
      \item
      [\textbf{Quality}] The organization of the manuscript and presentation of the data
      and results need some improvement.
      \item
      [\textbf{Data}] Please Select
      \item
      [\textbf{Accurate Key Points}] Please Select
    \end{itemize}
  \end{evaluation}
  \subsection*{Reviewer \#3 Formal Review for Author (shown to authors)}
  \begin{generalcomment}

    I would like to thank the authors to have addressed many of my comments. Here is now
    my second review on the revised manuscript which still needs substantial revisions.
    Pls take all my comments into account.
  \end{generalcomment}

  \subsection*{Reviewer \#3 Specific Comments}
  \begin{revresponse}[Thank you for your feedback.]
    We have carefully addressed all the issues item by item as follows.
  \end{revresponse}

  \begin{revcomment}[colframe={colorcommentresolved}]
    There are still unclear parts in what the overall goal of your study is. F.e. the
    climate sensitivity in the introduction and 3.4 is not taken up in the
    Conclusion-Summary and Abstract. So, what is the take home message of this part also
    in relation to the rest of the results?
  \end{revcomment}
  \begin{revcomment}[colframe={colorcommentresolved}]
    The model runs details are now becoming clearer, thanks for explaining these, but
    some others still remain unresolved such as:

    \begin{itemize}[noitemsep]
      \item
      Why not adding the simulations for the ERF calculations into Table 1 as well to
      make your two-step model approach clearer?
      \item
      How many ensembles have you run per SO2 injection experiments?
      \item
      How many years do you run? Which initial state (ENSO, QBO, AMOC, NAO, polar
      vortex) have you chosen and why?
      \item
      Were all started from the same initial state? Is the QBO included in the set-up?
    \end{itemize}
  \end{revcomment}
  \begin{revresponse}
    \emph{Check \citeA{toohey2014}.}
  \end{revresponse}
  \begin{revcomment}[colframe={colorcommentresolved}]
    Why have you run and setup this computational heavy CESM2WACCM6 model? Why do you
    need exactly this model version? The motivation for this is still not becoming clear
    as you are also comparing with model studies with prescribed aerosols which are
    linear scaled to Pinatubo. I feel explaining this will help to define the aim of
    your study better.
  \end{revcomment}
  \begin{revcomment}[colframe={colorcommentresolved}]
    It would be interesting to see what the difference in ERF-AOD and RF-AOD is between
    your two different model set-ups. Following this line of thought in Fig. 4 you have
    added RF values as ERF from other studies (Os20, B20, T10...). Is this appropriate
    to mix? I guess not as you explain now in your introduction. So I suggest to clean
    up and only add studies with an ERF model set-up and calculations which may explain
    some of your limitations to use certain studies and others not.
  \end{revcomment}
  \begin{revresponse}
    \emph{Can cite probably also \citeA{oconnor2021}.}
  \end{revresponse}
  \begin{revcomment}[colframe={colorcommentresolved}]
    Related to the previous bullet point. It is not becoming clear to me why certain
    studies are included in some of your figures and argumentation lines and some others
    are neglected. It would rather help to understand your argumentation flow better why
    you have done so. F.e. why is \citeA{jones2005} in Figs.\ 2 \& 4 but not
    \citeA{timmreck2010} (T10)? Why is \citeA{robock2009} overall neglected? Be
    consistent throughout the ms and in all figures or explain why this is impossible.
  \end{revcomment}
  \begin{revcomment}[colframe={colorcommentresolved}]
    The geoengineering study is mixed within the context of the volcanic modelling
    studies, which I think is not appropriate as continuously SO2 injections give
    different results. Thus, I suggest to take \citeA{niemeier2015} out of the main
    context and all figures and include it instead solely at the end of the discussion.
  \end{revcomment}
  \begin{revcomment}[colframe={colorcommentresolved}]
    Why are you modelling certain behaviors (Figs. 1-3) such as the different timing of
    AOD and ERF? Details are not shown and explained with your own results but instead
    discussed based on other models and CESM model studies.
  \end{revcomment}
  \begin{revcomment}[colframe={colorcommentresolved}]
    Discuss the caveats of your own model study: model set up, model comparison, \ldots
  \end{revcomment}
  \begin{revcomment}[colframe={colorcommentresolved}]
    Why including the outdated study by \citeA{jones2005}? I think there is enough more
    recent papers with prognostic aerosol schemes which are more suitable here. Or
    motivate why you need to focus exactly on \citeA{jones2005}.
  \end{revcomment}
  \begin{revcomment}[colframe={colorcommentresolved}]
    Reference style needs to be corrected to JGR style.
  \end{revcomment}
  \begin{revcomment}[colframe={colorcommentresolved}]
    Wording: ``observe'' for model studies shall be rather model/simulate/project
  \end{revcomment}
  \begin{revcomment}[colframe={colorcommentresolved}]
    Include \citeA{english2013,robock2009,metzner2014}
  \end{revcomment}

  \subsubsection*{Minor comments}

  \begin{revcomment}[colframe={colorcommentresolved}]
    Title to be adjusted to address the new and main aim of the paper along the lines of
    f.e. ``the focus regarding the non-linearity relationship between AOD and ERF is on
    the development over time in the post eruption period'' (from your Reply to Reviewer
    3).
  \end{revcomment}

  \subsubsection*{Abstract}

  \begin{revcomment}[colframe={colorcommentresolved}]
    ``Volcanic activity'' should to be ``Volcanic eruptions''
  \end{revcomment}
  \begin{revcomment}[colframe={colorcommentresolved}]
    ``Where the climate effect is only loosely tied to the magnitude of the eruption.''
    \citeA{metzner2014} has covered this in detail before; see also my below comments.
    So, this sentence needs to be rewritten. Be specific what you mean with the
    magnitude. The magnitude in volcanology is based on the mass of the erupted magma
    but I guess you mean the release of SO2 here?
  \end{revcomment}
  \begin{revcomment}[colframe={colorcommentresolved}]
    L34: peek {\textgreater} peak
  \end{revcomment}
  \begin{revcomment}[colframe={colorcommentresolved}]
    L36-37: Change ``While the largest uncertainty in the models is found to relate to
    the chemistry and physics of aerosol evolution''. This is not a new finding, what is
    new and different in your study?
  \end{revcomment}
  \begin{revcomment}[colframe={colorcommentresolved}]
    L43: ``where they cause a {\textgreater}surface{\textless} cooling''
  \end{revcomment}
  \begin{revcomment}[colframe={colorcommentresolved}]
    L46: The two measures {\textgreater}which?
  \end{revcomment}
  \begin{revcomment}[colframe={colorcommentresolved}]
    L48: seen {\textgreater} occurring
  \end{revcomment}
  \begin{revcomment}[colframe={colorcommentresolved}]
    L49: Pls add to be correct: up to the largest know eruptions . F.e. the Siberian
    Trap volcanism was larger.
  \end{revcomment}
  \begin{revcomment}[colframe={colorcommentresolved}]
    L55: ``for larger eruptions the ratio is also found to also change over time''. Be
    more specific what you mean here. Why does this matter?
  \end{revcomment}
  \begin{revcomment}[colframe={colorcommentresolved}]
    L56-57: Do you mean ``reaches limit of ...''? The linear aspect is well known from
    before. Also reaching a limiting value {\textless}of ...{\textgreater}?
  \end{revcomment}

  \subsubsection*{Introduction}

  \begin{revcomment}[colframe={colorcommentresolved}]
    L61: AOD is defined for the whole atmosphere unless otherwise specified, Pls clarify
    what you mean here.
  \end{revcomment}
  \begin{revcomment}[colframe={colorcommentresolved}]
    L64-71: Thanks for including the ERF background here, which I think could be a bit
    shortened.
  \end{revcomment}
  \begin{revcomment}[colframe={colorcommentresolved}]
    L71-72: ``and a general assumption of linearly dependence is commonly adopted.'' I
    strongly disagree here. To which kind of studies are you referring here?
  \end{revcomment}
  \begin{revcomment}[colframe={colorcommentresolved}]
    L73-76: ``of volcanic eruptions'' For which sizes and periods? Be specific.
  \end{revcomment}
  \begin{revcomment}[colframe={colorcommentresolved}]
    L80-81: small eruption with AOD ... at most 0.7: These are not small eruptions.
  \end{revcomment}
  \begin{revcomment}[colframe={colorcommentresolved}]
    L98: ``are readily removed by wet deposition''. Really only wet?
  \end{revcomment}
  \begin{revcomment}[colframe={colorcommentresolved}]
    The original reference that climate variability is forced by volcanic eruptions
    during the past millennium is \citeA{hegerl2006} next to \citeA{schurer2013}. For
    the Holocene period you need to add \citeA{vandijk2024}.
  \end{revcomment}
  \begin{revcomment}[colframe={colorcommentresolved}]
    L111: observe {\textgreater} model/ simulates
  \end{revcomment}
  \begin{revcomment}[colframe={colorcommentresolved}]
    L126-142: Structure and content to be reworked. Separate geoengineering study
    \citeA{niemeier2015} from volcanic studies. Next, I suggest structuring the
    paragraph along earlier studies with prescribed aerosols which are linear scaled AOD
    from the more sophisticated prognostic aerosol climate (chemistry) models studies.
    This will allow a better flow in case you really need to add both.
  \end{revcomment}
  \begin{revcomment}[colframe={colorcommentresolved}]
    L143-144: \citeA{metzner2014} has done this before with a two-step model approach
    using MAECHAM5HAM with varying SO2 injections from 0.4 to 687 Tg SO2 to derive
    aerosols, AOD, RF and then fed this into the CLIMBER model to calculate the climate
    response. The model results were then compared to the traditional linear approach
    (see Table 3, Figs. 6-8 of \citeA{metzner2014}). So, you need to include this study
    here as well to introduce what has been done before.
  \end{revcomment}
  \begin{revcomment}[colframe={colorcommentresolved}]
    L145-163: Do you need this ECS paragraph for your study? Consider shortening and
    clearly relating it to the results of your study here. There is Section 3.4 but not
    much in the Discussion, Conclusion-Summary nor Abstract.
  \end{revcomment}
  \begin{revcomment}[colframe={colorcommentresolved}]
    L164: ``Magnitude or more greater'': See my above comment. Magnitude has a specific
    meaning in volcanology based on the erupted magma mass which I assume you do not
    refer to here or?
  \end{revcomment}
  \begin{revcomment}[colframe={colorcommentresolved}]
    L176: Needs to add \citeA{toohey2019} as this is the key paper in this context here.
  \end{revcomment}
  \begin{revcomment}[colframe={colorcommentresolved}]
    L184: Why have you chosen these erupted SO2 masses, any motivation?
  \end{revcomment}
  \begin{revcomment}[colframe={colorcommentresolved}]
    L189-190: Cut here. Separate geoengineering \citeA{niemeier2015} from the volcanic
    studies and put it into a separate discussion at the end.
  \end{revcomment}
  \begin{revcomment}[colframe={colorcommentresolved}]
    Introduction: Introduce different SO2 strength of eruptions and their climate
    impacts \cite{miles2004,metzner2014,schmidt2022}. To just cite SB22 is not enough.
  \end{revcomment}

  \subsubsection*{Method}

  \begin{revcomment}[colframe={colorcommentresolved}]
    L201: MAM3: It is MAM4 in WACCM6 \cite{liu2016,gettleman2019}.
  \end{revcomment}
  \begin{revcomment}[colframe={colorcommentresolved}]
    L210-211: So for what do these long component names stand, their context? Be
    specific and explain details for the non CESM model experts.
  \end{revcomment}
  \begin{revcomment}[colframe={colorcommentresolved}]
    L215: How did you inject the SO2 mass; into one grid point at one time step? Please
    be specific and explain your model experimental set-up in more details as this
    matters.
  \end{revcomment}
  \begin{revcomment}[colframe={colorcommentresolved}]
    L229: Why have you chosen these dates starting on the 15 Feb...? Pinatubo erupted on
    June 12th, unknown eruptions are put on Jan eruption month in PIMP4. So why those
    dates?
  \end{revcomment}
  \begin{revcomment}[colframe={colorcommentresolved}]
    L235: A 400 Tg SO2 eruptions does not represent a 144-170 Tg SO2 eruption. Pls
    reword.
  \end{revcomment}
  \begin{revcomment}[colframe={colorcommentresolved}]
    Table 1 is not complete: Pls also add the simulations with prescribed ocean sea ice.
    Don't you have a control run? How many ensembles? Initial states?...
  \end{revcomment}
  \begin{revcomment}[colframe={colorcommentresolved}]
    L239: Why citing \citeA{jones2005}? You should add \citeA{robock2009} as well.
  \end{revcomment}
  \begin{revcomment}[colframe={colorcommentresolved}]
    L239-242: Your injection height has likely a significant impact on your results,
    which needs to be motivated and discussed. Can you show your vertical SO2 injection
    distribution in the supplement? Why have you chosen it that way? The motivation for
    this is missing. Why not at 24 km altitude as was observed for Pinatubo?
  \end{revcomment}
  \begin{revresponse}
    The default CESM2 historic (link to file) uses altitudes 18-20 for Pinatubo.
  \end{revresponse}
  \begin{revcomment}[colframe={colorcommentresolved}]
    L267: show{\textless}s{\textgreater}
  \end{revcomment}
  \begin{revcomment}[colframe={colorcommentresolved}]
    Fig. 1: Why the different temporal behavior for S26 compared to the others?
  \end{revcomment}
  \begin{revcomment}[colframe={colorcommentresolved}]
    Fig. 2: grey lines are showing what? STrop? Make the link to the Gregory study/ECS
    more clear; how does this relate to the rest of your study? What is the take home
    message of it?
  \end{revcomment}
  \begin{revcomment}[colframe={colorcommentresolved}]
    Fig. 3 Why adding linear fits?
  \end{revcomment}
  \begin{revcomment}[colframe={colorcommentresolved}]
    Fig. 4 Exclude N15 line and markers. Include in the discussion.
  \end{revcomment}
  \begin{revcomment}[colframe={colorcommentresolved}]
    L514-519: How realistic is a Reff {\textgreater}2.5 um \citeA{mcgraw2024}? ``good
    model agreement'': Well which model did McGraw use...?
  \end{revcomment}
  \begin{revcomment}[colframe={colorcommentresolved}]
    L525-526: Include \citeA{metzner2014} here.
  \end{revcomment}
  \begin{revcomment}[colframe={colorcommentresolved}]
    L534: Include \citeA{marshall2018} here as well.
  \end{revcomment}
  \begin{revcomment}[colframe={colorcommentresolved}]
    L548ff: Discuss considering the \citeA{toohey2019} and \citeA{zhuo2024} results.
  \end{revcomment}
  \begin{revcomment}[colframe={colorcommentresolved}]
    L556-559: Cite original studies such as \citeA{pinto1989,bekki1995} here as well.
    (commented before)
  \end{revcomment}
  \begin{revcomment}[colframe={colorcommentresolved}]
    L574-576: Cut. I suggest to not relate and cite \citeA{jones2005} here anymore.
  \end{revcomment}
  \begin{revcomment}[colframe={colorcommentresolved}]
    L594: ``is in contrast ... MAM 3'' Isn't it MAM4 \citeA{liu2016,gettleman2019}?
  \end{revcomment}
  \begin{revcomment}[colframe={colorcommentresolved}]
    L595-596: Distinguish between modal schemes and line by line schemes. Take out N15.
  \end{revcomment}
  \begin{revcomment}[colframe={colorcommentresolved}]
    L614: ensembles? How many ensembles did you run per SO2 injection experiment?
  \end{revcomment}
  \begin{revcomment}[colframe={colorcommentresolved}]
    L626-627: Not a new finding see \citeA{zhuo2024}.
  \end{revcomment}
  \begin{revcomment}[colframe={colorcommentresolved}]
    \citeA{brenna2020} does not use the same model configuration f.e. they use 1 deg
    resolution, which may matter for the aerosol evolution.
  \end{revcomment}

  \begin{concludingresponse}[]
    Thank you for your valuable comments on our manuscript.
  \end{concludingresponse}

  % Reviewer 4
  \reviewer[4] \label{rev:4}

  They want:
  \begin{itemize}
    \item
    Specify model/method better
    \item
    Better describe ERF, IRF, SARF
    \item
    Use correct longitude notation
    \item
    Clarify figure
  \end{itemize}

  \subsection*{Reviewer \#4 Evaluations}

  \begin{evaluation}
    \begin{itemize}[leftmargin=4.5cm,noitemsep]
      \item
      [\textbf{Recommendation}] Return to author for minor revisions
      \item
      [\textbf{Significant}] Yes, the paper is a significant contribution and worthy of
      prompt publication.
      \item
      [\textbf{Supported}] Yes
      \item
      [\textbf{Referencing}] Yes
      \item
      [\textbf{Quality}] Yes, it is well-written, logically organized, and the figures
      and tables are appropriate.
      \item
      [\textbf{Data}] Yes
      \item
      [\textbf{Accurate Key Points}] Yes
    \end{itemize}
  \end{evaluation}

  \subsection*{Reviewer \#4 Formal Review for Author (shown to authors)}
  \begin{generalcomment}
    This paper is a modeling study investigating the non-linear response of effective
    radiative forcing (ERF) to stratospheric aerosol injection due to supereruptions.
    Supereruptions in this paper refer to eruptions with a Volcano-Climate Index (VCI)
    ranging from 3 to 6. The authors perform simulations with 4 levels of injected
    sulfur encompassing the range of supereruptions inclusive of and larger than Mt.
    Pinatubo, thereby extending the understanding of the climate effects of volcanic
    eruptions. The temporal variation and sensitivity to SO2 injection magnitude of ERF
    and GMST were determined, and comparisons with previous studies' and model findings
    were included for corroboration.

    This work is relevant to the journal and may also be of interest to the broader
    scientific community, as geoengineering applications of stratospheric aerosols
    continue to be explored. I recommend this work for publication with a few minor
    revisions.
  \end{generalcomment}

  \subsection*{Reviewer \#4 Specific Comments}
  \begin{revcomment}[colframe={colorcommentresolved}]
    Line 47, ``ocean and sea-ice is held fixed.'': Please clarify what aspect is being
    held fixed.
  \end{revcomment}
  \begin{revcomment}[colframe={colorcommentresolved}]
    Line 49, ``While ERF takes into account rapid adjustments'': It may be helpful to
    briefly describe the differences between how ERF, IRF and SARF are determined.
  \end{revcomment}
  \begin{revresponse}
    \emph{Use \citeA{oconnor2021}.}
  \end{revresponse}
  \begin{revcomment}
    Line 135, ``cilmate'': Correct to ``climate''
  \end{revcomment}
  \begin{revresponse}[Fixed.]
  \end{revresponse}
  \begin{revcomment}[colframe={colorcommentresolved}]
    Line 225, ``287.7{degree sign}E'': The longitude should lie between 0 and 180
    degrees East. Please clarify what this coordinate means and why 56{degree sign}N was
    chosen for the high-latitude simulation.
  \end{revcomment}
  \begin{revcomment}[colframe={colorcommentresolved}]
    Figure 2: Please clarify how many yearly means are included in the plot for each of
    the STrop simulations. Is it every year from eruption to year 20? What about the
    season of eruption?
  \end{revcomment}
  \begin{revcomment}[colframe={colorcommentresolved}]
    Figure 4: It may be worth mentioning that the quantities described in the plots
    (ERF, GMST) are absolute values, since most of them are actually negative.
  \end{revcomment}
  \begin{revcomment}
    Line 554, ``peek'': Correct to ``peak''
  \end{revcomment}
  \begin{revresponse}[Fixed.]
  \end{revresponse}

  % Reviewer 5
  \reviewer[5] \label{rev:5}

  They want:
  \begin{itemize}
    \item
    Better model/method description
    \item
    Improve language
  \end{itemize}

  \subsection*{Reviewer \#5 Evaluations}

  \begin{evaluation}
    \begin{itemize}[leftmargin=4.5cm,noitemsep]
      \item
      [\textbf{Recommendation}] Return to author for minor revisions
      \item
      [\textbf{Significant}] The paper has some unclear or incomplete reasoning but will
      likely be a significant contribution with revision and clarification.
      \item
      [\textbf{Supported}] Mostly yes, but some further information and/or data are
      needed.
      \item
      [\textbf{Referencing}] Yes
      \item
      [\textbf{Quality}] Yes, it is well-written, logically organized, and the figures
      and tables are appropriate.
      \item
      [\textbf{Data}] Yes
      \item
      [\textbf{Accurate Key Points}] Yes
    \end{itemize}
  \end{evaluation}

  \subsection*{Reviewer \#5 Formal Review for Author (shown to authors)}
  \begin{generalcomment}
    Review of ``Radiative forcing by supereruptions'' by Eirik R. Enger et al. (revised
    MS for JGR, following earlier peer-review of initial-MS submission to the same
    journal.)

    This manuscript presents an analysis of interactive stratospheric aerosol model
    experiments to predict the radiative forcing that would result from so-called
    ``super-volcanic eruptions'', i.e. from emission of sulfur more than 10 times larger
    than Pinatubo.

    The model simulations apply the WACCM6-MAM3 model, within the CESM model framework,
    and explore ensembles of simulations at 400, 1600 and 3000Tg of SO2, i.e. around 25,
    100 and 200 times the sulfur emission from Pinatubo. The MAM3 model is an aerosol
    microphysics scheme, and for such very large eruptions, this model capability is
    particuarly important, to capture a key ``self-limiting effect'' within
    supervolcano-climate impacts. Such microphysics schemes are able to represent how
    the high sulfur generates much larger sulphate particles, with then a reduced
    lifetime in the stratosphere, for a short-lived, and also less severe surface
    cooling effect.

    Although there have been other model studies published quite recently on this topic
    (e.g. \citeA{mcgraw2024}), there can be quite large differences between predictions
    with different interactive stratospheric aerosol models, and this study is very
    valuable, for providing further experiments to quantify volcano-climate impacts in
    this super-eruption regime.

    The \citeA{mcgraw2024} study applied aerosol microphysics scheme also within a
    similar (but different) GCM, the NASA-GISS ModelE, with a different modal aerosol
    microphysics scheme ``MATRIX''. With there being relatively few studies to have
    explored the self-limiting effect, and the curve for how the effective radiative
    forcing and aerosol optical depth vary for SO2 injection larger than Pinatubom it is
    important these results are published.

    This manuscript is a revised version, from an initial version already reviewed by 3
    reviewers. I was not one of the initial reviewers, but have read through the
    reviewer comments, and the replies from the author team, and have referred also to
    the track changes manuscript (which indicates the changes from the original
    version).

    Whilst my review focuses on this latest manuscript, without prejudice from referring
    to the reviews, I can see that the authors have made substantial changes to the
    Abstract, Plain Language Summary, and the Introduction, to address the comments from
    the 3 reviewers, and I can see then that the manuscript is much improved from the
    original submission.

    I was initially quite surprised that there are only 4 Figures within this submission
    to JGR, but the authors have included, within the key synthesis Figure 4, results
    from a remarkable number of other studies, further putting the results from this
    study into then a useful broader context, including the recent similar curves for
    ModelE-MATRIX from \citeA{mcgraw2024}.

    There is an encouraging consistency in the results from the two models, for the
    curves of ERF vs emitted SO2 (Figure 4b) and global cooling (delta-GMST) vs emitted
    SO2 (Figure 4c). There is good alignment also with the \citeA{osipov2020} study, but
    that applied the same ModelE-MATRIX model from the \citeA{mcgraw2024} study, so the
    alternate model predictions presented here will be of substantial interest to the
    community.

    The authors improved the labelling of the model experiments to better communicate
    the relative magnitude of the SO2 emission and hemisphere-focus, and whilst my
    review identifies still a substantial number of minor revisions are required, these
    are straight-forward to make, and remain at a minor level overall, the
    recommendation then to publish after minor revisions.

    The main area of the manuscript that still requires substantial revision, is the
    description of the model (section 2.1) and the model experiments (section 2.2), and
    these are the first 2 of the 5 ``main minor comments'' I raise in this review.

    Within those initial 2 ``main minor comments'', I list there specific sets of
    revisions required, in each case, to enact the substantial improvement required to
    sections 2.1 and 2.2 respectively.

    I can understand the focus has been to address the main comments from the prior
    review, but these earlier sections, to describe the model and its experiments are
    important, to ensure reproducibility and traceability for the model experiments
    forming the analysis.

    Overall, the manuscript is much improved, and my judgement is that the manuscript is
    now within ``minor revisions'', to then be able to be published once this specific
    set of revisions set out below are addressed.
  \end{generalcomment}

  \subsection*{Reviewer \#5 Specific Comments}

  \subsubsection*{Main minor comments}

  \begin{revcomment}[after title={: M1},colframe={colorcommentresolved}]
    Section 2.1, lines 174-185 -- set of minor revisions to improve the model
    description.

    \begin{enumerate}
      \item
      [M1.1] Lines 174-178 refer to the fully coupled atmosphere-ocean configuration of
      CESM2, and lines 187 refers to applying ``the coupled model version BWma1850
      component setup'', but then line 189 explains also the fixed SST version is also
      used
      \item
      [M1.2] Line 178 add a sentence to this first para to give the chemistry scheme
      used. The aerosol scheme is explained in lines 179-185 but readers need to be
      aware also of how the SO2 oxidation is calculated in the model. Are the
      simulations applying interactive stratospheric chemistry, as well as interactive
      aerosol? Or are these ``specified chemistry'' simulations, based on climatological
      or ``prescribed'' oxidant fields?
      \item
      [M1.3] Line 181 Further to comment M1.2, the text here says ``default setting'',
      but this needs to be clarified which settings are ``default'' here. I am assuming
      this refers to the aerosol setting, and am guessing maybe this refers to the
      specified geomtric standard deviation (sometimes referred to as ``mode width''),
      which are specified then within the MAM3 scheme. I see the text here refers to Liu
      et al. (2016), but that is a tropospheric aerosol paper, and I'm aware some models
      (e.g. MA-ECHAM-HAM) use different mode widths for when simulating volcanic aerosol
      clouds, to those used when the model predicts tropospheric sulphate aerosol. Note
      that the \citeA{liu2016} paper is re: MAM4, the 4-mode version, not the 3-mode
      version. Please check this, and re: the \citeA{mills2017} paper, applying MAM3.

      Please clarify what is meant here by ``default setting'' and suggest to change to
      ``modal settings'' if that is the case, adding ``(for the MAM3 aerosol
      microphysics scheme)''. Rather than the \citeA{liu2016}, better to cite a paper
      applying MAM3 for stratospheric aerosol, such as \citeA{mills2017}. Further to
      comment above, is the approach here equivalent to one of the ``specified
      chemistry'' runs within that paper? Or is the SO2 oxidation calculated from
      interactive sulfur chemistry?
      \item
      [M1.4] Line 181 Further to comment M3 below, add a sentence to the end of this
      paragraph, to state explicitly that the model predictions here include, from the
      MAM3 aerosol microphysics, the key self-limiting effect of reduced stratospheric
      residence time from larger sulphate aerosol particles
      \cite<e.g.>{pinto1989,rampino1982,turco1979}.
    \end{enumerate}
  \end{revcomment}
  \begin{revcomment}[after title={: M2},colframe={colorcommentresolved}]
    Section 2.2, lines 187-195

    \begin{enumerate}
      \item
      [M2.1] Line 187 -- ``We are using'' is too colloqial for JGR article, revise to
      ``For predicting the super-eruption impacts, we use the ..'' and suggest to add
      ``pre-industrial setting'' before ``coupled model version'' (assuming that is the
      case, re: the 1850 there).
      \item
      [M2.2] Line 187 -- ``coupled model version BWma1850 component setup'' -- whilst
      this labelling here will mean something to the group of modellers running CESM,
      this needs to be re-worded. Is this ``BWma1850'' referring to the year 1850, and
      then that greenhouse gases and ozone depleting substances etc. are set at
      pre-industrial levels within the GCM? Please just state what the corresponding
      ``setup'' corresponds to, referring to the basic forcings within the climate
      model. I guess the 2 main aspects of the ``setup'' will be the greenhouse gases
      and the anthropogenic precursor emissions (for tropospheric ozone and aerosol).
      Please provide the specifics of this 1850 setup here.
      \item
      [M2.3] Lines 188-189 -- Similarly, ``an accompanying'' is too colloquial, change
      ``and an accompanying fixed sea-surface...'' to ``with a 2nd ensemble applying
      fixed sea-surface temperatures''. Delete the word ``version'' there, as this could
      be confusing, it's the same version of the atmosphere model, just the boundary
      condition is different. So avoid ``version'' (unless it's actually a different
      version of one of the model components).
      \item
      [M2.4] Lines 190-192 -- For JGR article it is not appropriate to specify variable
      names from a code or script, even within the model description text. This sentence
      here seems to be explaining what is different about ``fSST1850'', but my
      understanding is that is just a label for the fact that the model is using
      climatological sea-surface temperatures from a previous pre-industrial control run
      of the fully coupled atmosphere-ocean model. If that's the case, replace the
      current sentence with a sentence that states that.
    \end{enumerate}
  \end{revcomment}
  \begin{revcomment}[after title={: M3},colframe={colorcommentresolved}]
    Introduction, lines 76-79 (and suggest to join-up with 81-82).

    Whilst the authors have replied to the reviewer comments acknowledging the role of
    aerosol microphysics, and effective radius, the Introduction in the revised
    manuscript still does not include any sentences explaining about this, only the 3
    words ``including aerosol size'' on line 73.

    The simulations here apply the MAM3 aerosol microphysics scheme, and then the
    predicted AOD and ERF include the key self-limiting effect of a shortened residence
    time of volcanic aerosol from these very large SO2 emission cases.

    My suggestion here is to add the following 2 sentences about this

    ``Within very high SO2 eruptions, a key effect identified since the late 1970s
    \cite{rossow1977,rossow1978,turco1979} is a self-limiting particle-size effect
    implicit within the microphysics of the volcanic aerosol. In summary, for larger
    amounts of emitted SO2, particles in the aerosol cloud are larger, to then have
    greater fall speed, and reduced residence time in the stratosphere.''

    These sentences (or similar 2 sentences about this) could be incorporated into the
    text via adding a paragraph break before ``In the case of tropical eruptions''

    The 2 new sentences would follow-on well after the current 2 sentences there (the
    first being ``In the case of \ldots'' and the second being ``Upon descending below
    the \ldots''). The 4 sentences would then be a paragraph referring to the processes
    and transport within the stratospheric circulation.

    See also main minor revision M1.4 above, to also add a sentence to the model
    description to be clear this effect is resolved in the model predictions here.
  \end{revcomment}
  \begin{revcomment}[after title={: M4},colframe={colorcommentresolved}]
    Introduction, lines 108-113

    Whilst volcanic forcing is often given as a natural analogue for geoengineering, and
    certainly shares common processes with impacts from a one-off volcanic SO2 emission,
    the continued emission from SAI may result in a quite different resultant size
    increase.

    The studies on lines 108-113 are from continuous SO2 emission, and although that's
    interesting to consider, then with the inverse exponential relationship also
    indicating a self-limiting effect, the effect may not be comparable to one-off
    super-eruption case.

    This paragraph from lines 108 to 125 is a long paragraph, and suggest to have the
    text ``Even the 100 x Pinatubo \ldots'' be the start of a new paragraph (but
    re-worded to begin simply ``The 100 x Pinatubo \ldots''

    That then gives space to add a sentence after line 113, to note this, something
    similar to the 1st sentence above, re-worded slightly (delete ``certainly'' for
    example).
  \end{revcomment}
  \begin{revcomment}[after title={: M5},colframe={colorcommentresolved}]
    Results line 230-231

    The sentence here refers to ``means across the ensembles'', but the section 2 hasn't
    given any information about the ensembles

    See ``other minor revision'' O15) (comment \ref{revcom:5-o15}) below, where I've
    suggested to delete the sentence beginning ``The AOD \ldots'' on lines 204-207, this
    being superfluous. Doing so would then give space within section 2.2 to give
    information about the model ensembles.

    I am assuming that there are 2 types of ensemble, one within the alternate fixed-SST
    configuration, and other other in coupled atmos-ocean model configurations (from the
    explanation in section 2.1).

    With the additional variability in the ocean, I am assuming the ensemble woudl need
    to be larger for the coupled atmos-ocean ensemble, unless the sampling protocol has
    prior state, to then be enacting only a limited ocean variation within the coupled
    ensemble? Please clarify how the ensembles are set-up from the prior coupled
    atmos-ocean 1850 control.
  \end{revcomment}

  \subsubsection*{List of other minor comments}

  \begin{revcomment}[after title={: O1}]
    Manuscript title, line 1

    The authors have revised the title, replacing ``super-volcano eruptions'' with
    ``supereruptions'', which is OK, except that the word supereruptions would be much
    easier to read if hyphenated as ``super-eruptions''. I notice hyphenated spelling
    ``super-eruption'' is used quite extensively, although certainly not exclusively,
    and for example see this BBC article
    \url{https://www.bbc.co.uk/sn/tvradio/programmes/supervolcano/article.shtml}
    consistently has the hyphenated spelling. In my opinion it is much easier to read
    that way.

    Please change all instances of ``supereruption'' to ``super-eruption''. Thanks.
  \end{revcomment}
  \begin{revresponse}[Fixed.]
  \end{revresponse}
  \begin{revcomment}[after title={: O2},colframe={colorcommentresolved}]
    Introduction, line 45 -- ``are crucial metrics used to \ldots'' the word ``crucial''
    is hyperbole, and not really appropriate for a JGR article. Perhaps ``the main
    climate-relevant metrics''?
  \end{revcomment}
  \begin{revcomment}[after title={: O3}]
    Introduction, line 46 -- change ``represent the opacity'' to ``represents the
    opacity''
  \end{revcomment}
  \begin{revresponse}[Fixed.]
  \end{revresponse}
  \begin{revcomment}[after title={: O4}]
    Introduction, line 46 -- change ``while ERF specifically is the energy imbalance''.
    I get what you mean by ``imbalance'', in that the forcing agent causes a difference.
    But it's not necessarily moving the system towards an imbalanced state. Better to
    refer to the ERF simply as a difference in radiative flux or ``flux difference''.
    Suggest then to re-word to ``while the ERF is the energy flux-difference at the
    top-of-the-atmosphere'' or similar.
  \end{revcomment}
  \begin{revresponse}[Fixed.]
  \end{revresponse}
  \begin{revcomment}[after title={: O5},colframe={colorcommentresolved}]
    Introduction, lines 47-49 -- change ``Radiative forcing can however be calculated
    differently''. That wording there is not specific enough, and the follow-on ``an
    agreed-upon methodology has thus not always existed'' is a strange thing to say. One
    could say ``has not always existed'' about any scientific quantity. Suggest to
    re-word, to explain instead the difference between instantaneous radiative forcing
    and effective radiative forcing (i.e. re: rapid adjustments within the atmosphere
    system).
  \end{revcomment}
  \begin{revcomment}[after title={: O6},colframe={colorcommentresolved}]
    Introduction, line 53 -- ``the most precise indicator'' -- it's not really correct
    to refer to an issue of ``precision'' here. Any calculation can be precise or
    imprecise. Maybe you meant ``a reliable indicator''? But even then, I'd say that's
    not really the issue, it's more accounting for the energy changes at the top of the
    atmosphere, rather than any predictabillity of surface temperature change.
  \end{revcomment}
  \begin{revcomment}[after title={: O7},colframe={colorcommentresolved}]
    Introduction, line 59 -- Re-word ``Yet, a wide spread in the estimated aerosol
    forcing efficiencies''. The word ``Yet'' there doesn't really make sense. The models
    are making different predictions, but they're all predicting that based on different
    simulated clouds etc., so whether there's a linear relationship or not isn't really
    relevant to the spread in the model predicted forcings. It's certainly relevant to
    the magnitude of any 1 model's prediction, but it's not really relevant to the
    spread between different model predicted forcings. Basically the spread among
    different models isn't really relevant to this discussion about the relationship
    between ERF and AOD. One can consider any 1 model has some error bar associated with
    its predictions, and then that is relevant, but multi-model spread is a different
    issue.
  \end{revcomment}
  \begin{revcomment}[after title={: O8}]
    Introduction, line 67 -- delete ``molecules'' there, and suggest to change
    ``transformation'' also, instead to ``reaction''
  \end{revcomment}
  \begin{revresponse}[Fixed.]
  \end{revresponse}
  \begin{revcomment}[after title={: O9},colframe={colorcommentresolved}]
    Introduction, line 68 -- suggest here to simply refer to sulphate aerosol (rather
    than sulphuric acid).
  \end{revcomment}
  \begin{revcomment}[after title={: O10}]
    Introduction, line 123 -- change ``revealed'' to ``found''.
  \end{revcomment}
  \begin{revresponse}[Fixed.]
  \end{revresponse}
  \begin{revcomment}[after title={: O11},colframe={colorcommentresolved}]
    Section 2.2, lines 197-207

    These sentences beginning ``ERF is calculated as the \ldots'' need to be re-worded
    to avoid referring to variable names in the code. The explanation is OK, but for JGR
    article, needs to refer to the actual quantities, rather than variable names in the
    code. It's also not quite correct to refer to an ``energy imbalance'', it's better
    just to explain the forcing as the difference in the radiative flux at the top of
    the atmosphere caused by the forcing agent. Please discuss with the co-author team,
    and replace with a revised wording here.

    The text beginning ``The AOD'' on lines 204-207 is superfluous, and can be deleted.
    See main minor revision 5, I've suggested to delete this text, and give information
    here instead about how the ensembles are set-up, within the alternate fixed-SST and
    coupled atmos-ocean model configurations explained in section 2.1
  \end{revcomment}
  \begin{revcomment}[after title={: O12},colframe={colorcommentresolved}]
    Line 208 -- Further to the comment earlier about ensemble -- suggest to include in
    Appendix A the detailed description of how the ensembles are set up, and then in the
    text (within lines 230-231) can simply refer to the Appendix A for details, perhaps
    with a label for each ensemble such as ``FixedSST\_ENS10'' and ``CoupledAO\_ENS30''
    or similar (the 10 or 30 indicating the ensemble size).
  \end{revcomment}
  \begin{revcomment}[after title={: O13},colframe={colorcommentresolved}]
    Table 1 -- Change the column ``Eruption months'' instead to ``Eruption timing''. (To
    be clear it's not simply a sampling at different months in the year, it's when the
    eruption is enacted in the model).
  \end{revcomment}
  \begin{revcomment}[after title={: O14},colframe={colorcommentresolved}]
    \label{revcom:5-o14} Table 1 caption -- the text here states ``The three smallest in
    eruption magnitude tropical ensembles have four members, indicated by the number of
    eruption months''. But presumably each eruption month has more than 1 ensemble
    member, right? These are composition-climate model predictions so the ensembles are
    presumably of a reasonable number of members, not just 1 per eruption month? Please
    clarify, I'm assuming the ensembles are comparable in magnitude to the ensembles
    within \citeA{mcgraw2024}, right?
  \end{revcomment}
  \begin{revcomment}[after title={: O15},colframe={colorcommentresolved}]
    \label{revcom:5-o15} Figure 1 legend -- add ``Wm-2'' and delete the 3 characters ``C
    = ''. (It's important to communicate the units of the normalisation constant here.)
    Changing the ``18.27'' and ``12.43'' to ``18.3'' and ``12.4'' here will give an
    extra 4th character, then sufficient space to add `` Wm-2'' where the ``-2'' is
    super-script. and do the same in panel b), for ``-71.42'' and ``-63.71'' to
    ``-71.4'' and ``-63.7''. This also ensures numbers are all reported to 3 significant
    figures).
  \end{revcomment}
  \begin{revcomment}[after title={: O16},colframe={colorcommentresolved}]
    Figure 1c -- it's important to be able to see the tick-marks on the right-hand inset
    Figure. You could give the values to 1 decimal place for the S3000 and S1629 to give
    then room for the Wm-2, after also deleting the ``C = '' (only 2 significant figures
    is fine here).
  \end{revcomment}
  \begin{revcomment}[after title={: O17},colframe={colorcommentresolved}]
    Figure 1 caption -- Change ``The time series have been normalised to have peak
    values at unity'' instead to ``Each time series has been normalised to the peak
    value''. You don't need to say ``at unity'', people will understand that's what you
    mean by ``normalised''. Also change ``, where C is the normalisation constant''
    instead to ``, values of C given in the legend.''. The value of C is more
    significant than simply a normalisation constant, and better just to give this
    without that labelling.
  \end{revcomment}
  \begin{revcomment}[after title={: O18},colframe={colorcommentresolved}]
    Figure 1 caption -- The ``across the ensembles'' is not clear. See
    comment~\ref{revcom:5-o14}: O14, the text currently does not communicate
    sufficiently what each ensemble is representing, and once this is clarified the
    Figure needs to be amended to refer to the label for each ensemble. Please also add
    the number of ensemble members within the fixed SST and coupled-ocean ensembles (or
    clarify which are shown in this Figure).
  \end{revcomment}
  \bibliography{/home/een023/science/ref/ref} % don't specify bibliographystyle
\end{document}
