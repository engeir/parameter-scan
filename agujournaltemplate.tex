%%%%%%%%%%%%%%%%%%%%%%%%%%%%%%%%%%%%%%%%%%%%%%%%%%%%%%%%%%%%%%%%%%%%%%%%%%%%
% AGUJournalTemplate.tex: this template file is for articles formatted with LaTeX
%
% This file includes commands and instructions
% given in the order necessary to produce a final output that will
% satisfy AGU requirements, including customized APA reference formatting.
%
% You may copy this file and give it your
% article name, and enter your text.
%
% guidelines and troubleshooting are here: 

%% To submit your paper:
\documentclass[draft]{agujournal2019}
\usepackage{url} %this package should fix any errors with URLs in refs.
\usepackage{lineno}
\usepackage[inline]{trackchanges} %for better track changes. finalnew option will compile document with changes incorporated.
\usepackage{soul}
\linenumbers
% Adding packages
\usepackage{multirow}
\usepackage{colortbl}
\usepackage{amsmath,siunitx}
\usepackage[version=4]{mhchem}
\usepackage{booktabs} % To thicken table lines
\definecolor{LightGray}{gray}{0.9}
\urldef\fssturl\url{
1850_CAM60%WCCM_CLM50%BGC-CROP_CICE%PRES_DOCN%DOM_MOSART_CISM2%NOEVOLVE_SWAV_TEST
}
\newcommand{\iso}[1][i]{{#1}njected \ce{SO2}}
\newcommand{\cwmp}{C2W\(-\)}
\newcommand{\cwm}{C2W\(\downarrow\)}
\newcommand{\cws}{C2WN\(\uparrow\)}
\newcommand{\cwsn}{C2W\(\uparrow\)}
%%%%%%%
% As of 2018 we recommend use of the TrackChanges package to mark revisions.
% The trackchanges package adds five new LaTeX commands:
%
%  \note[editor]{The note}
%  \annote[editor]{Text to annotate}{The note}
%  \add[editor]{Text to add}
%  \remove[editor]{Text to remove}
%  \change[editor]{Text to remove}{Text to add}
%
% complete documentation is here: http://trackchanges.sourceforge.net/
%%%%%%%

\draftfalse

%% Enter journal name below.
%% Choose from this list of Journals:
%
% JGR: Atmospheres
% JGR: Biogeosciences
% JGR: Earth Surface
% JGR: Oceans
% JGR: Planets
% JGR: Solid Earth
% JGR: Space Physics
% Global Biogeochemical Cycles
% Geophysical Research Letters
% Paleoceanography and Paleoclimatology
% Radio Science
% Reviews of Geophysics
% Tectonics
% Space Weather
% Water Resources Research
% Geochemistry, Geophysics, Geosystems
% Journal of Advances in Modeling Earth Systems (JAMES)
% Earth's Future
% Earth and Space Science
% Geohealth
%
% ie, \journalname{Water Resources Research}

\journalname{JGR: Atmospheres}


\begin{document}

%%%%%%%%%%%%%%%%%%%%%%%%%%%%%%%%%%%%%%%%%%%%%%%
%  TITLE
%
% (A title should be specific, informative, and brief. Use
% abbreviations only if they are defined in the abstract. Titles that
% start with general keywords then specific terms are optimized in
% searches)
%
%%%%%%%%%%%%%%%%%%%%%%%%%%%%%%%%%%%%%%%%%%%%%%%

% Example: \title{This is a test title}

\title{Radiative forcing by super-volcano eruptions}

%%%%%%%%%%%%%%%%%%%%%%%%%%%%%%%%%%%%%%%%%%%%%%%
%
%  AUTHORS AND AFFILIATIONS
%
%%%%%%%%%%%%%%%%%%%%%%%%%%%%%%%%%%%%%%%%%%%%%%%

% Authors are individuals who have significantly contributed to the
% research and preparation of the article. Group authors are allowed, if
% each author in the group is separately identified in an appendix.)

% List authors by first name or initial followed by last name and
% separated by commas. Use \affil{} to number affiliations, and
% \thanks{} for author notes.
% Additional author notes should be indicated with \thanks{} (for
% example, for current addresses).

% Example: \authors{A. B. Author\affil{1}\thanks{Current address, Antartica}, B. C. Author\affil{2,3}, and D. E.
% Author\affil{3,4}\thanks{Also funded by Monsanto.}}

\authors{Eirik R. Enger\affil{1}, Rune Graversen\affil{1}, Audun Theodorsen\affil{1}}

% \affiliation{1}{First Affiliation}
% \affiliation{2}{Second Affiliation}
% \affiliation{3}{Third Affiliation}
% \affiliation{4}{Fourth Affiliation}

\affiliation{1}{UiT The Arctic University of Norway, Tromsø, Norway}
%(repeat as many times as is necessary)

% Corresponding author mailing address and e-mail address:

% (include name and email addresses of the corresponding author.  More
% than one corresponding author is allowed in this LaTeX file and for
% publication; but only one corresponding author is allowed in our
% editorial system.)

% Example: \correspondingauthor{First and Last Name}{email@address.edu}

\correspondingauthor{Eirik R. Enger}{eirik.r.enger@uit.no}

%%%%%%%%%%%%%%%%%%%%%%%%%%%%%%%%%%%%%%%%%%%%%%%
% KEY POINTS
%%%%%%%%%%%%%%%%%%%%%%%%%%%%%%%%%%%%%%%%%%%%%%%
%  List up to three key points (at least one is required)
%  Key Points summarize the main points and conclusions of the article
%  Each must be 140 characters or fewer with no special characters or punctuation and must be complete sentences

% Example:
% \begin{keypoints}
% \item	List up to three key points (at least one is required)
% \item	Key Points summarize the main points and conclusions of the article
% \item	Each must be 140 characters or fewer with no special characters or punctuation and must be complete sentences
% \end{keypoints}

\begin{keypoints}
  \item RF to AOD ratio has a time-after-eruption dependence
  \item AOD peak occur at later post-eruption for larger eruptions
  \item The linear RF dependence on AOD break down for eruptions larger than Pinatubo
\end{keypoints}

%%%%%%%%%%%%%%%%%%%%%%%%%%%%%%%%%%%%%%%%%%%%%%%
%
%  ABSTRACT and PLAIN LANGUAGE SUMMARY
%
% A good Abstract will begin with a short description of the problem
% being addressed, briefly describe the new data or analyses, then
% briefly states the main conclusion(s) and how they are supported and
% uncertainties.

% The Plain Language Summary should be written for a broad audience,
% including journalists and the science-interested public, that will not have 
% a background in your field.
%
% A Plain Language Summary is required in GRL, JGR: Planets, JGR: Biogeosciences,
% JGR: Oceans, G-Cubed, Reviews of Geophysics, and JAMES.
% see http://sharingscience.agu.org/creating-plain-language-summary/)
%
%%%%%%%%%%%%%%%%%%%%%%%%%%%%%%%%%%%%%%%%%%%%%%%

%% \begin{abstract} starts the second page

\begin{abstract}
We investigate the climatic effects of volcanic eruptions spanning from medium-size
events, such as Mt.\ Pinatubo still having a global impact, to super-volcanoes. The
study is based on ensemble simulations in the CESM2 climate model using the
WACCM atmosphere model. Here we focus on the dependence of the climate response to
the magnitude of the volcanic eruption. Our analysis centres on the impact of injections
of different magnitudes of \ce{SO2} on AOD, RF, and global temperature
anomalies. Unlike the traditional linear models used for smaller eruptions, our
results reveal a non-linear relationship between RF and AOD for larger
eruptions. We also uncover a notable time-dependent decrease in post-eruption aerosol
forcing efficiency across all eruption magnitudes. In addition, the study reveals that
larger volcanic events produce a delayed and sharper peak in AOD, alongside a
similar development of the RF and temperature time series. These findings
emphasise the complexity of volcanic impacts on climate, demonstrating significant
differences in climatic response depending on eruption magnitude.
\end{abstract}

% \section*{Plain Language Summary}
% Enter your Plain Language Summary here or delete this section.
% Here are instructions on writing a Plain Language Summary:
% https://www.agu.org/Share-and-Advocate/Share/Community/Plain-language-summary

%%%%%%%%%%%%%%%%%%%%%%%%%%%%%%%%%%%%%%%%%%%%%%%
%
%  BODY TEXT
%
%%%%%%%%%%%%%%%%%%%%%%%%%%%%%%%%%%%%%%%%%%%%%%%

%%% Suggested section heads:
% \section{Introduction}
%
% The main text should start with an introduction. Except for short
% manuscripts (such as comments and replies), the text should be divided
% into sections, each with its own heading.

% Headings should be sentence fragments and do not begin with a
% lowercase letter or number. Examples of good headings are:

% \section{Materials and Methods}
% Here is text on Materials and Methods.
%
% \subsection{A descriptive heading about methods}
% More about Methods.
%
% \section{Data} (Or section title might be a descriptive heading about data)
%
% \section{Results} (Or section title might be a descriptive heading about the
% results)
%
% \section{Conclusions}

%%

%  Numbered lines in equations:
%  To add line numbers to lines in equations,
%  \begin{linenomath*}
%  \begin{equation}
%  \end{equation}
%  \end{linenomath*}

%% Enter Figures and Tables near as possible to where they are first mentioned:
%
% DO NOT USE \psfrag or \subfigure commands.
%
% Figure captions go below the figure.
% Acronyms used in figure captions will be spelled out in the final, published version.

% Table titles go above tables;  other caption information
%  should be placed in last line of the table, using
% \multicolumn2l{$^a$ This is a table note.}
% NOTE that there is no difference between table caption and table heading in the final, published version
%
%----------------
% EXAMPLE FIGURES
%
% \begin{figure}
% \includegraphics{example.png}
% \caption{caption}
% \end{figure}
%
% Giving latex a width will help it to scale the figure properly. A simple trick is to use \textwidth. Try this if large figures run off the side of the page.
% \begin{figure}
% \noindent\includegraphics[width=\textwidth]{anothersample.png}
%\caption{caption}
%\label{pngfiguresample}
%\end{figure}
%
%
% If you get an error about an unknown bounding box, try specifying the width and height of the figure with the natwidth and natheight options. This is common when trying to add a PDF figure without pdflatex.
% \begin{figure}
% \noindent\includegraphics[natwidth=800px,natheight=600px]{samplefigure.pdf}
%\caption{caption}
%\label{pdffiguresample}
%\end{figure}
%
%
% PDFLatex does not seem to be able to process EPS figures. You may want to try the epstopdf package.
%

%
% ---------------
% EXAMPLE TABLE
%
% \begin{table}
% \caption{Time of the Transition Between Phase 1 and Phase 2$^{a}$}
% \centering
% \begin{tabular}{l c}
% \hline
%  Run  & Time (min)  \\
% \hline
%   $l1$  & 260   \\
%   $l2$  & 300   \\
%   $l3$  & 340   \\
%   $h1$  & 270   \\
%   $h2$  & 250   \\
%   $h3$  & 380   \\
%   $r1$  & 370   \\
%   $r2$  & 390   \\
% \hline
% \multicolumn{2}{l}{$^{a}$Footnote text here.}
% \end{tabular}
% \end{table}

%%%%%%%%%%%%%%%%%%%%%%%%%%%%%%%%%%%%%%%%%%%%%%%
% SIDEWAYS FIGURES and TABLES
% AGU prefers the use of {sidewaystable} over {landscapetable} as it causes fewer problems.
%
% \begin{sidewaysfigure}
% \includegraphics[width=20pc]{figsamp}
% \caption{caption here}
% \label{newfig}
% \end{sidewaysfigure}
%
%  \begin{sidewaystable}
%  \caption{Caption here}
% \label{tab:signif_gap_clos}
%  \begin{tabular}{ccc}
% one&two&three\\
% four&five&six
%  \end{tabular}
%  \end{sidewaystable}

%% If using numbered lines, please surround equations with \begin{linenomath*}...\end{linenomath*}
%\begin{linenomath*}
%\begin{equation}
%y|{f} \sim g(m, \sigma),
%\end{equation}
%\end{linenomath*}

\section{Introduction}

% NOTE: Suggested layout for the introduction
% - The objectives of the work.
% - The justification for these objectives: Why is the work important?
% - Background: Who else has done what? How? What have we done previously?
% - Guidance to the reader: What should the reader watch for in the paper? What are the
%   interesting high points? What strategy did we use?
% - Summary/conclusion: What should the reader expect as conclusion? In advanced
%   versions of the outline, you should also include all the sections that will go in
%   the Experimental section (at the level of paragraph subheadings) and indicate what
%   information will go.

% Toohey et al 2011 have a nice end of introduction.

RF and AOD are crucial metrics representing the energy imbalance at
TOA and the stratospheric opacity due to aerosol scattering, respectively. They
are extensively used to quantify the impact of major volcanic eruptions. The assumption
of a linear dependency of RF on AOD is commonly adopted
\cite{myhre2013,andersson2015}, and applying such a linear relationship have yielded
reasonably accurate estimates in climate model simulations of volcanic eruptions
\cite{mills2017,hansen2005,gregory2016,marshall2020,pitari2016b}. Yet, there is a wide
spread in the estimated aerosol forcing efficiencies (RF normalised by AOD)
among studies, spanning approximately from \(\sim
\SI{15}{\watt\metre^{-2}\ce{AOD}^{-1}}\) \cite{pitari2016b} to \(\sim
\SI{25}{\watt\metre^{-2}\ce{AOD}^{-1}}\) \cite{myhre2013}. Additionally, these
estimates are predominantly based on small volcanoes with AOD values up to at most
\(\sim 0.7\).

Although \ce{H2O}, \ce{N2} and \ce{CO2} are the most abundant gases emitted by volcanoes
\cite{robock2000}, sulphur species such as \ce{SO2} provide greater influence due to
the comparatively high background concentrations of the former gases in the atmosphere.
The transformation of \ce{SO2} molecules through reactions with \ce{OH} and \ce{H2O}
leads to the formation of sulphate acid (\ce{H2SO4}) \cite{robock2000}, which scatter
sunlight hereby elevating planetary albedo and reducing the RF. As the conversion
from \ce{SO2} to \ce{H2SO4} occurs over weeks \cite{robock2000}, the peak RF
experiences a slight delay from the eruption's peak \ce{SO2} injection. The lifetime of
the \ce{H2SO4} aerosols in the stratosphere depends on various factors, including
latitude \cite{marshall2019, toohey2019}, volcanic plume height \cite{marshall2019},
aerosol size \cite{marshall2019}, the quasi-biennial oscillation phase
\cite{pitari2016b} and the season of the year (determining to which hemisphere aerosols
are transported) \cite{toohey2011,toohey2019}. In the case of tropical eruptions,
aerosols are typically transported poleward in the stratosphere and descend back to
mid-latitude troposphere within one to two years \cite{robock2000}. Upon descending
below the tropopause, these aerosols are readily removed by wet deposition
\cite{liu2012}.

Before the current era of significant anthropogenic climate forcing, volcanic eruptions
were the primary forcing mechanism dictating Earth's climate variability during the
Holocene period \cite{sigl2022}. Despite this substantial impact, few climate-model
experiments have included volcanic forcing when simulating climate evolution during the
Holocene \cite{sigl2022}, likely implying an exaggerated positive forcing
\cite{gregory2016,solomon2011}. This absence of persistent cooling is one of several
factors that have been suggested to contribute to the common disparity between simulated
and observed global warming \cite{andersson2015}. Despite extensive attention on
understanding the way volcanic eruptions influence climate, questions regarding aerosol
particle processes --- such as growth and creation rates when \ce{OH} is scarce ---
remain unanswered \cite <e.g.,>[]{robock2000,zanchettin2019,marshall2020,marshall2022}.
These processes impact aerosol scattering efficiency and potentially the RF to
AOD relationship. \citeA{marshall2020} observed higher aerosol forcing efficiency
in post-eruption years \(2\) and \(3\) compared to year 1, and attributing this
post-eruption increase in aerosol forcing efficiency to strong spatial concentration in
the initial year and subsequent distribution of aerosols over a larger area. This
spatial redistribution increases the albedo per global mean AOD hereby causing a
stronger RF to AOD ratio \cite{marshall2020}.

Previous studies of both Mt.\ Pinatubo \cite{mills2017,hansen2005} and volcanoes within
the instrumental era \cite{gregory2016} have been used to estimate the relationship
between the RF energy imbalance and change in AOD caused by volcanic
eruptions. While \citeA{myhre2013} employ a formula scaling RF by AOD to
obtain \(\SI{-25}{\watt\metre^{-2}\mathrm{AOD}^{-1}}\), recent literature reports
estimates down to \(\SI{-19.0(5)}{\watt\metre^{-2}\mathrm{AOD}^{-1}}\)
\cite{gregory2016} and \(\SI{-18.3(10)}{\watt\metre^{-2}\mathrm{AOD}^{-1}}\)
\cite{mills2017}. Synthetic volcano simulations in \citeA{marshall2020} yield a scaling
factor of \(\SI{-20.5(2)}{\watt\metre^{-2}\mathrm{AOD}^{-1}}\) across an ensemble of
\(82\) simulations featuring varying injection heights and latitudes of volcanic
emissions, with \iso{} ranging from \(10\) to \(\SI{100}{\tera\gram(\ce{SO2})}\).

An in-some-way similar simulation setup, albeit with notable differences, was conducted
by \citeA{niemeier2015}, involving an ensemble of \(14\) levels of injected sulphur
spanning between \(\SI{1}{\tera\gram(\ce{S})\mathrm{yr}^{-1}}\)
(\(\SI{2}{\tera\gram(\ce{SO2})\mathrm{yr}^{-1}}\)) and
\(\SI{100}{\tera\gram(\ce{S})\mathrm{yr}^{-1}}\)
(\(\SI{200}{\tera\gram(\ce{SO2})\mathrm{yr}^{-1}}\)). These geoengineering simulations
maintained continuous sulphur injections, running until a steady sulphur level was
achieved. Results indicated an inverse exponential relationship between RF and
\iso{} rate, converging to \(\SI{-65}{\watt\metre^{-2}}\)
(Eq.~\ref{eq:niemeier_exponential}). However, even the \(100\times\) Mt.\ Pinatubo
super-volcano simulation by \citeA{jones2005}, who obtained a peak RF of
\(\SI{-60}{\watt\metre^{-2}}\), is below the suggested limit of
\(\SI{-65}{\watt\metre^{-2}}\). Moreover, \citeA{timmreck2010} find a peak RF
anomaly of \(\SI{-18}{\watt\metre^{-2}}\) from a \(\SI{1700}{\tera\gram(\ce{SO2})}\)
eruption simulation, which corresponds well with the function estimated by
\citeA{niemeier2015} at the given \ce{SO2} level.

One avenue that has garnered considerable attention is comparing the magnitude of
volcanic or volcano-like forcings to increased \ce{CO2} levels. Several studies explore
the connection between volcanic forcing and the climate sensitivity to a doubling of
\ce{CO2}
\cite{boer2007,marvel2016,merlis2014,ollila2016,richardson2019,salvi2022,wigley2005}.
This comparison aims to mitigate the large uncertainty in estimates of the sensitivity
of the real climate system. Inferring climate sensitivity from volcanic events has been
attempted as a way to constrain the sensitivity \cite{boer2007}, assuming that volcanic
and \ce{CO2} forcings produce similar feedbacks \cite{pauling2023}. Earlier studies
suggest the potential for constraining ECS using volcanoes \cite{bender2010},
provided that ECS is constrained by ERF rather than IRF, as ERF
accounts for rapid atmospheric adjustments in contrast to IRF
\cite{richardson2019}. However, other studies refute this approach, pointing out that
different sensitivities of volcanic forcing and \ce{CO2} doubling seem to exist
\cite{douglass2006}, or that constraining the ECS by ERF lacks accuracy due
to the precision of climate simulations \cite{boer2007,salvi2022}. Although ERF
offers a more suitable indicator of forcing than IRF
\cite{marvel2016,richardson2019}, more recent studies conclude that ECS cannot be
constrained from volcanic events \cite{pauling2023}.

Several studies have demonstrated a linear relationship of approximately
\(-\SI{20}{\watt\metre^{-2}\mathrm{AOD}^{-1}}\) between RF and AOD, although
substantial variability exists in the slope among studies
\cite{mills2017,hansen2005,gregory2016,marshall2020,pitari2016b}. Moreover, a
time-after-eruption dependence on the RF to AOD ratio is found in
\citeA{marshall2020}, whereas \citeA{niemeier2015} revealed a non-linear relationship
between RF and \iso{}. Thus, a consensus on the relationship between \iso{},
AOD and RF has yet to be established.

To address these issues, we conducted ensemble simulations of volcanic eruptions in the
CESM2 coupled with the WACCM. The ensembles span \iso{} levels of three
orders of magnitude with four members for each: \(\SI{26}{\tera\gram(\ce{SO2})}\),
\(\SI{400}{\tera\gram(\ce{SO2})}\) and \(\SI{1629}{\tera\gram(\ce{SO2})}\). Details
regarding the experimental setup are provided in section~\ref{sec:method}. Our findings
reveal clear non-linear RF to AOD dependencies for medium to super-volcano
size eruptions. Additionally, we observe a time-dependent variation in the RF to
AOD ratio, detailed in section~\ref{sec:results} and discussed in
section~\ref{sec:discussion}. Furthermore, our data, along with insights from previous
studies, suggest that the RF dependency on \iso{} identified by
\citeA{niemeier2015} acts as a lower boundary. Our conclusions are presented in
section~\ref{sec:conclusions}.

\section{Method}\label{sec:method}

\subsection{Model}

We utilised the CESM2 \cite{danabasoglu2020} in conjunction with the WACCM
\cite{gettleman2019} and the fully dynamical ocean component POP
\cite{smith2010, danabasoglu2020}. The atmosphere model was run at nominal
\(\SI{2}{\degree}\) resolution with \(70\) vertical levels in the MA
configuration.

The WACCM version employed in the MA configuration uses the MAM3
\cite{gettleman2019}, a simplified and computationally efficient default setting within
the CAM5 \cite{liu2016}, as described in \citeA{liu2012}. The MAM3 was
developed from MAM7, consisting of the seven modes Aitken, accumulation, primary carbon,
fine dust, fine sea salt, coarse dust and coarse sea salt. Instantaneous internal mixing
of primary carbonaceous aerosols with secondary aerosols and instantaneous ageing of
primary carbonaceous particles is assumed by emitting primary carbon in the accumulation
mode \cite{liu2016}. Fine sea salt is assimilated into the accumulation mode, and as
dust absorbs water efficiently, aerosols are expected to be removed by wet deposition
similarly to sea salt, allowing fine dust to be merged into the accumulation mode as
well. Likewise, coarse dust is merged with coarse sea salt into a coarse mode, and as
both absorb water efficiently, the coarse mode will quickly retain its background state
below the tropopause \cite{liu2012}. Consequently, MAM3 features the three modes
Aitken, accumulation and coarse \cite{liu2016}.

\subsection{Simulations}

Appendix A provides a detailed description of the simulation setup and utilised output
variables. Table~\ref{tab:simulation-overview} summarizes the simulations, encompassing
three \ce{SO2} injection magnitudes across four seasons: February 15th, May 15th, August
15th, and November 15th. The magnitudes vary over three orders of magnitude:
\(\SI{26}{\tera\gram(\ce{SO2})}\), \(\SI{400}{\tera\gram(\ce{SO2})}\), and
\(\SI{1629}{\tera\gram(\ce{SO2})}\).

The smallest eruption case, \cwm{}, is similar in magnitude as compared to events
like Mt.\ Pinatubo
\cite<\(\sim10\)--\(\SI{20}{\tera\gram(\ce{SO2})}\);>[]{timmreck2018} and Mt.\ Tambora
\cite<\(\sim\SI{56.2}{\tera\gram(\ce{SO2})}\);>[]{zanchettin2016}. The intermediate
case, \cwmp{}, resembles the Samalas eruption in 1257
\cite<\(\sim{118.8}\)--\(\SI{173.1}{\tera\gram(\ce{SO2})}\);>[]{toohey2017,ottobliesner2016},
while the largest eruption case, \cws{}, is similar to the YTT eruption
occurring about \(\SI{75000}{\mathrm{yr}}\) ago
\cite<\(100\)--\(\SI{10000}{\tera\gram(\ce{SO2})}\);>[]{jones2005}. All eruptions were
situated at the equator (\(\SI{0}{\degree N}\), \(\SI{1}{\degree E}\)) with \ce{SO2}
injected between \(\SI{18}{\kilo\meter}\) and \(\SI{20}{\kilo\meter}\) altitude.
Collectively, the three eruption cases \cwm{}, \cwmp{} and \cws{} are
referred to as C2W. Two additional high-latitude eruptions, labelled \cwsn{},
of the same \iso{} magnitude as \cws{} were simulated at \(\SI{56}{\degree N}\),
\(\SI{287.7}{\degree E}\) with a six-month separation (February 15th and August 15th).

Employing eruptions in the medium to super-volcano size enhances the signal-to-noise
ratio without necessitating an extensive and computationally expensive ensemble, and as
such, is a tempting way to mimic a large ensemble of smaller volcanic eruptions.
However, there is no guarantee that the AOD, RF and temperature signatures
will be a simple scaling of that of smaller volcanic eruptions. Previous studies have
simulated super-volcanoes using AOD as the input forcing, where the AOD was
that of Mt.\ Pinatubo scaled by one hundred \cite{jones2005}. This approach may yield
incorrect results, both since the peak of the AOD may be too small or big, but
also since the evolution of the AOD could be wrong. Similarly, a different
AOD evolution may be found from similar size volcanoes, but at different
latitudes.

\begin{table*}
  \centering

  \caption{Simulations done with the CESM2\(^{a}\)}\label{tab:simulation-overview}%
  \begin{center}
    \begin{tabular}[c]{cccc}
      \toprule
      Ensemble name                                                                    & \(\si{\tera\gram(\ce{SO2})}\)         &
      Lat, lon, alt [\si{\degree\mathrm{N}}, \si{\degree\mathrm{E}}, \si{\kilo\metre}] & Eruption months                         \\
      \midrule
      \cwsn{}                                                                      & \(1629\)                              &
      \(56\), \(287.7\),
      \(18\)--\(20\)                                                                   & Feb,\hphantom{May,}Aug\hphantom{,Nov}   \\
      \cws{}                                                                       & \(1629\)                              &
      \(\hphantom{1}0\), \(\hphantom{28}1\hphantom{.7}\), \(18\)--\(20\)
                                                                                       & Feb,May,Aug,Nov                         \\
      \cwmp{}                                                                      & \(\hphantom{1}400\)                   &
      \(\hphantom{1}0\),
      \(\hphantom{28}1\hphantom{.7}\),
      \(18\)--\(20\)                                                                   & Feb,May,Aug,Nov                         \\
      \cwm{}                                                                       & \(\hphantom{14}26\)                   &
      \(\hphantom{1}0\),
      \(\hphantom{28}1\hphantom{.7}\), \(18\)--\(20\)
                                                                                       &
                                                                                       Feb,May,Aug,Nov \\
      \toprule
      \multicolumn{4}{l}{\parbox{\linewidth}{\(^{a}\)The ensembles \cwsn{} and \cws{}
    have the same eruption magnitude, but while \cws{} is located at the equator,
    \cwsn{} is located at high northern latitude. \cwmp{} and \cwm{} are located
    at the equator, but with different magnitudes as compared to \cws{}. All tropical
    ensembles have four members, indicated by the amount of eruption months, while the
northern latitude ensemble consist of two members}}
    \end{tabular}
  \end{center}
\end{table*}

\section{Results}\label{sec:results}

% NOTE: the results should be laid out in a logical way, with the most
% interesting/important stuff first, then tangents that dig deeper at specific things
% later.
% 1. RF to AOD time-after-eruption dependence should be top priority (8 figs atm.)
% 2. Then probably temperature scaling since we discuss the shape of both AOD and RF
%    time series before that (MOTIVATION: can we expect a specific temperature time
%    series shape based on the shape of either of or both of the RF and AOD time
%    series?)
% 3. If there is something interesting to say about the rest of the figures (all the
%    comparing of parameters), then this should come here.

\subsection{Analysis of the time series}

Figure~\ref{fig:compare-waveform-temp} illustrates time series of global mean AOD,
RF, and surface air temperature. The black lines represent the medians across the
four-member ensembles, while shading indicates the 5th to 95th percentiles. Three
distinct forcing magnitudes (\cwm{}, \cwmp{}, and \cws{}), outlined in
table~\ref{tab:simulation-overview}, have been used. The time series in
Fig.~\ref{fig:compare-waveform-temp} are normalised by setting the peak value to unity,
defined based on the peak of a fit from a Savitzky-Golay filter of 3rd order and a
one-year window length \cite{savitzky1964}.

A notable feature across all three subfigures of Fig.~\ref{fig:compare-waveform-temp} is
the earlier peak occurrence of the \cwm{} case compared to the larger eruption
cases. Cases \cwmp{} and \cws{} peak at similar times, but \cwmp{} exhibits
a faster rise and slower decay in the AOD time series
(Fig.~\ref{fig:compare-waveform-temp}a). Generally, the AOD time series from
stronger eruption cases seem to display a sharper peak, or a slower rise and faster
decay. The rise across the three eruption cases in Fig.~\ref{fig:compare-waveform-temp}b
is similar, although \cwm{} reaches the peak just prior to the other two cases.
During the decay phase, all cases appear to decay at a similar rate, however maintaining
the offset caused by the earlier peak arrival of \cwm{}. Cases \cwmp{} and
\cws{} exhibit indistinguishable RF time series. Similarly, when observing the
temperature evolution in Fig.~\ref{fig:compare-waveform-temp}c, \cwmp{} and
\cws{} are indistinguishable, while \cwm{} has a sharper peak relative to the
two other cases. Nevertheless, across all cases for the parameters shown in
Fig.~\ref{fig:compare-waveform-temp}, a similar temporal development is found.
Therefore, similar dynamics are expected to be at play in all cases.

\begin{figure}
  \centering
  \includegraphics{figures/figure1.png}

  \caption{AOD (a), RF (b) and temperature response (c) time series to the
    three tropical volcanic eruption cases, \cwm{}, \cwmp{} and \cws{}. The time
    series have been normalised to have peak values at unity, where \(C\) is the
    normalisation constant. Black lines indicate the median across the four-member
    ensembles, while shading marks the 5th and 95th
    percentiles.}\label{fig:compare-waveform-temp}%
\end{figure}

Upon asking whether the shape of the temperature time series can be inferred from the
shape of either of the forcing time series (AOD or RF), we find that the
shapes of the RF time series are consistent over the different eruption strengths
(Fig.~\ref{fig:compare-waveform-temp}b), suggesting a strong dependence of temperature
on RF. About the same can largely be said about the AOD time series, though
they show a slight change in shape from smaller to larger eruptions
(Fig.~\ref{fig:compare-waveform-temp}a). Specifically, the AOD time series from
smaller eruptions displays a fast rise and a flat peak before decaying back to its
equilibrium state. From the larger eruptions, we find a slower rise but a sharper peak,
resulting in a decay to equilibrium happening at a similar time after the eruption and
at a similar rate.

\subsection{RF dependency on AOD}

We next focus on the development of the AOD and RF time series relative to
each other. Similar comparisons were conducted in \citeA[their Fig.\ 4]{gregory2016}
and \citeA[their Fig.\ 1]{marshall2020}, with RF plotted against AOD.
Figure~\ref{fig:aod_vs_toa_ses_avg} displays annual mean values from the four simulation
cases in table~\ref{tab:simulation-overview}; the small eruption case (\cwm{}) as
blue downward pointing triangles, the intermediate eruption case (\cwmp{}) as orange
thick diamonds, the large tropical eruption case (\cws{}) as green upward pointing
triangles, and the large northern hemisphere eruption case (\cwsn{}) as brown upward
pointing three-branched twigs. Also shown are the data from \citeA[Fig.\ 4, black
  crosses from HadCM3 sstPiHistVol]{gregory2016} as grey crosses labelled G16
(described in Appendix B, section~\ref{ap:g16}). Additionally, the estimated peak values
from the Mt.\ Pinatubo and Mt.\ Tambora eruptions are plotted as a black star and plus,
while the peak from the \citeA{jones2005} simulation is shown as a pink square labelled
J05. Finally, red circles represent the peak values obtained from the C2W
tropical eruption cases. The straight lines are the same as shown by
\citeA{gregory2016}. The full data range is shown in Fig.~\ref{fig:aod_vs_toa_ses_avg}a
while Fig.~\ref{fig:aod_vs_toa_ses_avg}b highlights a narrow range, focusing on the
\cwm{} case.

The annual mean data from the Pinatubo-like \cwm{} case in
Fig.~\ref{fig:aod_vs_toa_ses_avg}b have RF values as a function of AOD that
follow almost the same constant slope as the G16 data. However, in
Fig.~\ref{fig:aod_vs_toa_ses_avg}a we observe that the stronger eruptions lead to
dissimilar responses in AOD and RF, where \cwmp{}
seems to follow close to a \(-10\) slope and \cws{} is closer to a
\(-5\) slope. The peak values (red circles) suggest a non-linear dependence, while
within each eruption strength (same colour) the annual mean values fall relatively close
to a straight line.

\begin{figure}
  \centering
  \includegraphics{figures/figure2.png}

  \caption{RF as a function of AOD, yearly means. Data from the four
    simulations listed in table~\ref{tab:simulation-overview} (\cwm{}, \cwmp{},
    \cws{} and \cwsn{}) are shown along with the data from the HadCM3 sstPiHistVol
    simulation by \citeA{gregory2016} (grey crosses, G16). Also shown are the estimated peak
    values of the Mt.\ Pinatubo (black star) and Mt.\ Tambora (black plus) eruptions. In (a)
    the simulated super-volcano of \citeA{jones2005} (pink square) is shown, as well as the
    peak values from the simulations \cwm{}, \cwmp{} and \cws{} (red circles).
    All peak values (as opposed to annual means) have an asterisk (\(\ast{}\)) in their
    label. The grey lines are the same regression fits as in \citeA[Fig.\ 4]{gregory2016},
    where the solid line is the fit to G16. (b): Zooming
    in on the smallest AOD values.}\label{fig:aod_vs_toa_ses_avg}%
\end{figure}

To find the development of AOD and RF relative to each other over time, we
plot in Fig.~\ref{fig:aod_vs_toa_avg_loop_ratios} seasonal means of the RF to
AOD ratio, where the start of the time series is taken as the eruption day. The
plot shows all the eruption cases given in table~\ref{tab:simulation-overview}, as well
as the tropical eruptions from the \citeA{marshall2020dataset} dataset (\(6\) of \(82\)
eruptions), labelled M20 and described in Appendix B, section~\ref{ap:m20}. In
Fig.~\ref{fig:aod_vs_toa_avg_loop_ratios}a, slopes are linear regression fits to the
seasonal means across all ensemble members, summarised in
table~\ref{tab:slope-gradients}. Shaded regions are the standard deviation around the
seasonal means. A similar shading is plotted in
Fig.~\ref{fig:aod_vs_toa_avg_loop_ratios}b, but where the regression fits have been
omitted for clarity. Years \(1\) and \(2\) have the lowest signal-to-noise ratio, as
well as year \(0\) (the noise is mostly due to the RF time series, shown in
Fig.~\ref{fig:compare-waveform-temp}b). For this reason, the ratio of RF to
AOD is calculated for the second season of the first year until the end of the
third year.

Although the ratio changes across the eruption magnitudes, we find that all the tropical
cases (C2W) follow a positive slope during the first period, as seen in
Fig.~\ref{fig:aod_vs_toa_avg_loop_ratios}a and described in
table~\ref{tab:slope-gradients}. A positive slope is also found from the tropical
eruptions in the M20 dataset. The northern latitude case in \cwsn{} show a
much flatter slope compared to C2W and M20. The distinction between the
slopes from the tropical and non-tropical cases is perhaps more clear in
Fig.~\ref{fig:aod_vs_toa_avg_loop_ratios}b and corresponding rows in
table~\ref{tab:slope-gradients}. Again, \cwsn{} show an almost flat slope compared
to the tropical cases. During the second period more noise is introduced, but a weak
tendency of negative slopes is found among the tropical cases.

\begin{table}
  \centering

  \caption{Slope and standard deviation for the data in
    Fig.~\ref{fig:aod_vs_toa_avg_loop_ratios}\(^{a}\)}\label{tab:slope-gradients}%
  \begin{tabular}{cccc}
    \toprule
    Figure                                                  & Ensemble name & 1st period      & 2nd period       \\
    \midrule
                                                            & \cwsn{}   & \(0.45\pm1.15\) & \(1.51\pm1.45\)  \\
                                                            & \cws{}    & \(3.85\pm0.52\) & \(-3.29\pm0.60\) \\
    \ref{fig:aod_vs_toa_avg_loop_ratios}a                   & \cwmp{}   & \(4.36\pm0.82\) & \(-3.37\pm0.59\) \\
                                                            & \cwm{}    & \(3.64\pm2.41\) & \(-1.41\pm3.25\) \\
                                                            & M20     & \(6.34\pm1.77\) & \(-0.36\pm1.33\) \\
    \midrule
                                                            & \cwsn{}   & \(0.08\pm0.20\) & \(0.27\pm0.26\)  \\
                                                            & \cws{}    & \(0.75\pm0.10\) & \(-0.64\pm0.12\) \\
    \ref{fig:aod_vs_toa_avg_loop_ratios}b                   & \cwmp{}   & \(0.43\pm0.08\) & \(-0.34\pm0.06\) \\
                                                            & \cwm{}    & \(0.18\pm0.12\) & \(-0.07\pm0.16\) \\
                                                            & M20     & \(0.33\pm0.07\) & \(-0.02\pm0.08\) \\
    \toprule
    \multicolumn{4}{l}{\parbox{\linewidth}{\(^{a}\)The regression fits in the top half of the
    table are for Fig.~\ref{fig:aod_vs_toa_avg_loop_ratios}a, while the bottom half is for
    Fig.~\ref{fig:aod_vs_toa_avg_loop_ratios}b. The columns ``1st period'' and ``2nd
    period'' refer to the period year \(0\)--\(1\) and \(1\)--\(3\), respectively. The
    ensembles are the same as those given in table~\ref{tab:simulation-overview}, in
    addition to the \(6\) tropical eruptions from the \(82\) member ensemble in
    \citeA{marshall2020}.}} \\
  \end{tabular}
\end{table}

\citeA[their Fig.\ 1c,d]{marshall2020} present results that demonstrate a
time-dependent relationship in the conversion between AOD and RF. Contrary
to the results from the C2W cases, they obtain an RF to AOD ratio with
a negative slope over time. As such, \citeA{marshall2020} find that the aerosol forcing
efficiency increases with time rather than decrease. This phenomenon is explained by
\citeA{marshall2020} as the aerosols initially are spatially confined to the hemisphere
where the eruption occurred. Subsequently, during the second and third years, they
spread globally, resulting in a higher global-mean albedo per AOD and consequently
stronger RF per AOD ratio with time. However, this included the full
\(82\)-member ensemble. When constraining the ensemble to only include eruptions within
\(-10\) to \(\SI{10}{\degree\mathrm{N}}\), we obtain the positive slope as stated above
and shown in Fig.~\ref{fig:aod_vs_toa_avg_loop_ratios} and
table~\ref{tab:slope-gradients}. Therefore, the much flatter slope of \cwsn{} should
be expected based on the results from \citeA{marshall2020}, which indicate that tropical
eruptions contribute to a positive slope while high-latitude eruptions contribute to a
negative slope. As a result, the aerosol forcing efficiency seems strongly dependent on
eruption latitude.

\begin{figure}
  \centering
  \includegraphics{figures/figure3.png}

  \caption{(a): The ratio of RF to AOD, with time-after-eruption on the
    horizontal axis. Straight lines indicate linear regression fits and are described in
    table~\ref{tab:slope-gradients}, while shaded regions are the standard deviation across
    the ensembles for each season. (b): Same as in (a), but where the underlying AOD
    and RF time series have been scaled to have peak values at unity. Shown are data
    from table~\ref{tab:simulation-overview} along with tropical eruptions from
    M20.}\label{fig:aod_vs_toa_avg_loop_ratios}%
\end{figure}

\subsection{Parameter scan}

In Fig.~\ref{fig:parameter_scan}, we compare all relevant parameters against each other.
The primary input parameter in the CESM2 is \iso{}. For our tropical cases
(C2W), we observe an almost linear relationship between AOD peak values
against \iso{}. The latitude also plays a role for the magnitude of the AOD
perturbation, evident from \cwsn{}. This weak yet significant latitude dependence
aligns with findings by \citeA{marshall2019}, indicating that \(\SI{72}{\percent}\) of
the AOD variance can be attributed to \iso{}, while latitude accounts for only
\(\SI{16}{\percent}\) of the variance. Peak values from their data (82 simulations)
plotted as red thin diamonds displays a similar pattern, with AOD exhibiting close
to linear dependence on \iso{}, but with latitude introducing a spread in AOD.
Peak values from Mt.\ Pinatubo (P) and Mt.\ Tambora (T) are shown for reference, along
with peak values from \citeA{jones2005} labelled J05 and \citeA{timmreck2010}
labelled T10.

The almost linear relationship between AOD and \iso{} for the C2W data
suggest a comparable trend for RF versus \iso{} as seen in RF versus
AOD. In Fig.~\ref{fig:parameter_scan}b, RF plotted against \iso{} (with the
absolute value of RF on the \(y\)-axis) indicates a substantial damping effect on
RF as \iso{} increases for the C2W data, in agreement with results from
\citeA{ottobliesner2016}, labelled OB16. The analysis details of OB16 can be
found in Appendix B, section~\ref{ap:ob16}. Despite the model complexity difference,
\citeA{ottobliesner2016}'s simulations using CESM1 with a low-top atmosphere
(CAM5) produce RFs comparable to our findings.

\begin{figure*}
  \centering
  \includegraphics{figures/figure4.png}

  \caption{(a) AOD (b) RF and (c) temperature anomaly as a function of
    \iso{}\@. (d) RF and (e) temperature anomaly as a function of AOD. (f)
    Temperature anomaly as a function of RF. Blue diamonds labelled C2W
    represent tropical cases (\cwm{}, \cwmp{}, \cws{}), the brown three-branched
    twig signifies the \cwsn{} case, and green downward triangles denote OB16 data
    from \citeA{ottobliesner2016}. The red thin diamonds labelled M20 display the
    \citeA{marshall2020dataset} data. Black star and plus indicate Mt.\ Pinatubo and Mt.\
    Tambora estimates based on observations. The pink square labelled J05 refers to
    the one-hundred times Mt.\ Pinatubo super-volcano from \citeA{jones2005}, and the pink
    disk labelled T10 represents the YTT super-volcano from
    \citeA{timmreck2010}. The pink dashed line labelled N15 is from
    \citeA{niemeier2015}, indicating the function in
    Eq.~\ref{eq:niemeier_exponential}.}\label{fig:parameter_scan}%
\end{figure*}

% INFO: the conversion between S and SO2 is confirmed by Niemeier and Timmreck (2015)'s
% reference to the Bekki et al. (1996) paper. Bekki uses 6000 Mt SO2, Niemeier uses 3000
% Tg(S).
\citeA{niemeier2015} conducted simulations of continuous sulphur injections up to
\(\SI{200}{\tera\gram(\ce{SO2})\mathrm{yr}^{-1}}\) in the ECHAM5's middle atmosphere
version \cite{giorgetta2006} with aerosol microphysics from HAM \cite{stier2005}. They
observed an RF dependence on \ce{SO2} injection rate following an inverse
exponential, which converges to \(\SI{-65}{\watt\meter^{-2}}\), depicted in
Fig.~\ref{fig:parameter_scan}b as the stippled pink line labelled N15 and given
as;

\begin{equation}
  \Delta
  R_{\mathrm{TOA}} =
  -\SI{65}{\watt\metre^{-2}}
  \mathrm{e}^{-{\left(\frac{\SI{2246}{\tera\gram(S)yr^{-1}}}{x}\right)}^{0.23}}.
  \label{eq:niemeier_exponential}
\end{equation}
%
Both our simulations and OB16 exhibit a notably faster increase than this
exponential relationship. However, T10 closely correspond to the function in
Eq.~\ref{eq:niemeier_exponential}. Starting from an initial input of
\(\SI{850}{\tera\gram(\ce{S})}\) (equivalent to \(\SI{1700}{\tera\gram(\ce{SO2})}\),
representing the YTT eruption), their estimated AOD led to a peak RF
of \(\SI{-18}{\watt\metre^{-2}}\) (pink filled circle in
Fig.~\ref{fig:parameter_scan}b). These results were from a simulation utilising the
MPI-ESM climate model, driven by AOD data from the HAM aerosol model. Thus, the
alignment likely stem from using the same aerosol microphysical model in
\citeA{timmreck2010} and \citeA{niemeier2015}, alongside applying highly similar climate
models, MPI-ESM and ECHAM5, respectively \cite{kuma2023}. The climate model family
relations are further examined in Appendix C. Notably, the peak values from M20
align well within an upper boundary defined by C2W and OB16, and a lower
boundary defined by Eq.~\ref{eq:niemeier_exponential}. Eruptions closer to the equator
within M20 correspond to data points near the upper boundary, while eruptions at
more extreme latitudes yield weaker peak RF values being closer to the lower
boundary. Crucially, none of the eruption simulations violated the suggested upper
threshold of \(\SI{-65}{\watt\metre^{-2}}\) as defined in
Eq.~\ref{eq:niemeier_exponential}.

Figure~\ref{fig:parameter_scan}c illustrates the response of temperature against \iso{}.
Similar to Fig.~\ref{fig:parameter_scan}b, the increase of temperature response with
\iso{} decreases for higher \iso{}. Notably, OB16 takes a different trajectory
compared to C2W, showing smaller temperature dependence on \iso{}. In contrast to
our findings, T10 finds a considerably weaker temperature perturbation, noting a
maximum temperature anomaly of only \(\SI{-3.5}{\kelvin}\) for their
\(\SI{1700}{\tera\gram(\ce{SO2})}\) eruption, while J05 records a substantially
larger maximum temperature anomaly of \(\SI{-10.7}{\kelvin}\) compared to our C2W
simulations.

Moving to Fig.~\ref{fig:parameter_scan}d, we revisit the relationship between RF
and AOD, focusing on peak values rather than annual and seasonal averages. As
previously discussed, the RF to AOD ratio displays weaker slopes than
previous studies \cite{jones2005, marshall2020, timmreck2010}, with the C2W peak
values not conforming to a linear trend. This comparison suggests potential significant
dependencies on the model and its input parameters, such as latitude, but most notably
to an inherent non-linear RF dependence on AOD. Both the G16 and
J05 data originate from the same climate model, and similarly to what we find from
the C2W data, the ratio is much stronger for small eruptions in the industrial era
(G16) compared to the super-volcano eruption (J05).

For Fig.~\ref{fig:parameter_scan}e, C2W should resemble the patterns observed in
Fig.~\ref{fig:parameter_scan}c due to the nearly linear association identified between
AOD and \iso{} in Fig.~\ref{fig:parameter_scan}a. This is indeed the case, and in
addition both the \cwsn{} and the J05 cases align well, with the T10
case following a similar dependence. M20 shows temperature anomalies of smaller
extent, similar to what was found in Fig.~\ref{fig:parameter_scan}c. However, the
M20 experiment was conducted with prescribed sea-surface temperatures
\cite{marshall2020}, preventing the temperature from being fully perturbed.

Finally, in Fig.~\ref{fig:parameter_scan}f, we compare the temperature and RF
responses. Both C2W and OB16 show a near-linear relationship between
temperature and RF. The C2W data indicate a steeper slope, potentially
implying stronger temperature perturbations compared to OB16. However, there are
potential biases in the values from the analysis of the OB16, as outlined in
Appendix B, section~\ref{ap:ob16}. This, along with considerable noise, result in the
analysis of OB16 being less reliable. As in Fig.~\ref{fig:parameter_scan}e, the
\cwsn{} case along with the J05 and T10 cases closely follow the
temperature to RF dependence of C2W.

\section{Discussion}\label{sec:discussion}

% NOTE: Suggested layout for the
% Discussion:
% - Explain the results and emphasize significant findings clearly
% - Discuss the impact and importance of results compared with recent relevant research
% Conclusion
% - The justification for these objectives: Why is the work important?
% - Summarize the key points made in the other sections
% - Conclude overall discussion of article
% - Link this section to the introduction

\subsection{Linearity between AOD and RF}

Figures~\ref{fig:aod_vs_toa_ses_avg},~\ref{fig:aod_vs_toa_avg_loop_ratios} and
\ref{fig:parameter_scan}d demonstrate that as the AOD exceeds approximately
\(1.0\), the linear RF dependence of approximately
\(\SI{-20}{\watt\metre^{-2}\mathrm{AOD}^{-1}}\) no longer hold. The almost linear
relationship between AOD and \iso{} in Fig.~\ref{fig:parameter_scan}a indicates
that larger eruptions, injecting more \ce{SO2}, lead to larger aerosols, and hence less
effective radiation scattering, thereby reducing the RF for the same AOD
\cite{english2013, timmreck2010, timmreck2018}.

\citeA{timmreck2010} highlights that for sufficiently large eruptions such as Mt.\
Pinatubo and YTT, \ce{OH} radicals are too scarce, which limit \ce{SO2} oxidation.
The AOD peak in the YTT simulation of \citeA{timmreck2010} occur six months
after Mt.\ Pinatubo's peak. This result aligns with our findings, as illustrated in
Fig.~\ref{fig:compare-waveform-temp}a, where smaller eruptions show an earlier AOD
peak. While \citeA{timmreck2010} reports a peak RF anomaly occurring \(7\)--\(8\)
months post-eruption, \citeA{jones2005} suggests a peak anomaly one year post-eruption.
The RF peak preceding the AOD peak, approximately \(6\)--\(8\) months
post-eruption in CESM2 (see Fig.~\ref{fig:compare-waveform-temp}b), aligns well
with what was found by \citeA{timmreck2010} for the YTT. Hence, both the AOD
and RF time series appear influenced by the \iso{} magnitude and the \ce{OH}
abundance, affecting the peak timing as well as the magnitude.

Although J05 is comparable to \cws{} concerning AOD and RF peak
values, the temperature response reported by \citeA{jones2005} appears much stronger
than what our strongest eruption leads too. Since \citeA{jones2005} multiplies the
AOD time series from Mt.\ Pinatubo by one hundred to represent the AOD time
series of a super-volcano, this simple approach could potentially deviate significantly
from the real AOD time series of the super-volcano, both in shape and magnitude.
In addition, it may cause a substantially different temperature perturbation.
\citeA{timmreck2010} obtained their AOD estimate from an initial injection of
\ce{SO2}, which resulted in a delayed peak, but also much smaller peak compared to that
of J05. Also, the maximum temperature perturbation of T10 is much smaller
than that of J05, largely due to the large difference in AOD magnitude.
Since in addition the J05 temperature is greater than that from \cws{}, it is
expected that the shape of the AOD time series is also important in determining
the strength of the aerosol forcing and corresponding temperature perturbation.

The biggest spread in the data is found when converting from \iso{} to any of the three
output parameters when comparing across models. Conversion from \iso{} to AOD is
consistent within similar models, even when comparing simulations of volcanic eruptions
\cite{timmreck2010} and continuous injection of \ce{SO2} \cite{niemeier2015}, but has
a wide spread at large values of \iso{} across model families
(Figs.~\ref{fig:parameter_scan}a,b,c). Comparatively, the RF
(Fig.~\ref{fig:parameter_scan}d) and temperature (Fig.~\ref{fig:parameter_scan}e) as a
function of AOD demonstrate a smaller spread across models, and consequently, the
spread for temperature as a function of RF (Fig.~\ref{fig:parameter_scan}f) is
also small. Previous studies assumed a roughly linear relationship between RF and
AOD, particularly for lower values of AOD and RF, where the estimated
slope was notably steeper at around \(\SI{-20}{\watt\metre^{-2}\mathrm{AOD}^{-1}}\) for
\(\mathrm{AOD}<1\) compared to the approximately
\(\SI{-5}{\watt\metre^{-2}\mathrm{AOD}^{-1}}\) observed here at \(\mathrm{AOD}\gg1\).
Hence, a linear relationship appears to be an accurate estimate of RF dependence
on AOD for eruptions similar to or smaller than Mt.\ Pinatubo. However, for larger
eruptions, factors like \ce{OH} scarcity and aerosol growth, influencing reflectance,
and their gravitational pull substantially impact both AOD and RF evolution.

From the C2W cases, a post-eruption time dependence on the RF to AOD
ratio emerges. \citeA{marshall2020} discusses a similar aspect, finding that aerosol
forcing efficiency strengths from year 1 to year 2. This is by \citeA{marshall2020}
attributed to the time taken for aerosol dispersion, affecting global albedo and
consequently RF, whereas AOD is less affected by aerosol dispersion. Here,
the aerosol forcing efficiency become weaker during the first period, as depicted in
Fig.~\ref{fig:aod_vs_toa_avg_loop_ratios}. Focusing solely on tropical eruptions in
M20 (between \(-10\) and \(\SI{10}{\degree\mathrm{N}}\)), the RF to
AOD ratio closely resembles the findings from the \cwm{} case. Thus, while
\iso{} is crucial for estimating the time-average of the RF to AOD ratio,
latitude and in particular aerosol dispersion seem more influential in determining the
post-eruption evolution of the ratio. Given that the \cwsn{} case lack a significant
increase in ratio as compared to C2W, the substantial difference in eruption
latitude appears to be a likely cause.

\citeA{marshall2019}, \citeA{marshall2020} and \citeA{marshall2021} utilise a code with seven log-normal
modes to simulate aerosol mass and number concentrations, along with an atmosphere-only
configuration of the UM-UKCA with prescribed sea-surface temperatures and sea-ice extent
\cite{marshall2019}. This approach is in contrast with CESM2, operating as an
ESM, but with a simpler aerosol chemistry model in the MAM3. The family of
models to which M20 is based is different from that of C2W, and also
different from the T10 and N15, as described in Appendix C. Based on
Fig.~\ref{fig:parameter_scan}, the model family seems pivotal in determining the
estimated AOD and RF magnitudes from \iso{}, whereas the various models
generally demonstrate more consistency in representing RF from AOD. Given
that M20 employs a model from a distinct family compared to both OB16 and
C2W, and T10 and N15, while covering the parameter space between the
two families, it would be intriguing to include higher \iso{} values in the M20
experiments to explore whether RF against \iso{} remains bounded below (by
T10 and N15) and above (by OB16 and C2W), as
Fig.~\ref{fig:parameter_scan}b indicate. This also prompts questions about whether
\ce{SO2} saturation at a specific level yields a lower bound on the corresponding peak
RF response, and if this peak RF response is similar to what a high-latitude
eruption would produce. Alternatively, differences in model aerosol chemistry may be
what produces the wide range in RF as a function of \iso{}.

In summary, smaller eruptions and their impact produce a relatively well defined
RF to AOD ratio (\(\sim \SI{-20}{\watt\metre^{-2}\mathrm{AOD}^{-1}}\)),
whereas larger eruptions result in estimates with smaller magnitudes (\(\sim
\SI{-10}{\watt\metre^{-2}\mathrm{AOD}^{-1}}\) to \(\sim
\SI{-5}{\watt\metre^{-2}\mathrm{AOD}^{-1}}\), as depicted in
Fig.~\ref{fig:aod_vs_toa_avg_loop_ratios}). \citeA{niemeier2017} indicate a decrease in
aerosol forcing efficiency as the injection rate increases, due to larger volcanic
eruptions leading to larger aerosol particles that scatter sunlight less efficiently,
thereby decreasing the forcing efficiency per \iso{} \cite{english2013, timmreck2018}.

\subsection{Climate sensitivity estimate}

As previously mentioned, J05 agree well with \cws{} concerning both AOD
and RF values, yet differ in temperature. To investigate this discrepancy, we here
conduct a comparison between their climate feedback parameter \(\alpha \) (where
\(s=1/\alpha \) is the climate sensitivity parameter) with our climate resistance,
denoted as \(\rho \), and the TCRP \(1/\rho\) (where
\(\mathrm{TCS}=F_{2\times}\times \mathrm{TCRP}\) is the transient climate sensitivity).
As forcing of volcanic eruptions typically last for about a year, a duration too brief
for the timescales at which \(F=\rho T\) remains valid \cite{gregory2016}, an
alternative approach involves using a time-integral form introduced by
\citeA{merlis2014}:

\begin{equation}
  \int_0^{\tau}F \mathrm{d}t=\rho\int_{0}^{\tau}T \mathrm{d}t
\end{equation}
\begin{equation}
  \rho=\frac{\int_0^{\tau}F \mathrm{d}t}{\int_{0}^{\tau}T \mathrm{d}t}.
  \label{eq:climate-resistance}
\end{equation}

If the upper bound of the integral, \(\tau \), is sufficiently large, so that the upper
ocean heat content is the same at \(t=0\) and \(t=\tau \), this approach agrees with
\(F=\rho T\) for long-term forcing \cite{gregory2016} (\citeA{merlis2014} utilised
\(\tau =\SI{15}{\mathrm{yr}}\)). Additionally, it is worth noting that the climate
resistance and the climate feedback parameter are associated with the ocean heat uptake
efficiency (\(\kappa \)) through \(\rho =\alpha +\kappa \) \cite{gregory2016}.

The climate feedback parameter estimated by \citeA{jones2005} is \(\alpha \simeq
\SI{4}{\watt\metre^{-2}\kelvin^{-1}}\), exceeding twice the value obtained by
\citeA{gregory2016} in their simulations of Mt.\ Pinatubo using the same HadCM3 climate
model. We determine the climate resistance using the integral-form computation outlined
in Eq.~\ref{eq:climate-resistance} and adopting \(\tau =\SI{20}{\mathrm{yr}}\). The
estimated climate resistance from the three tropical simulation cases (with four in each
ensemble) yields \(\rho =\SI{3(2)}{\watt\metre^{-2}\kelvin^{-1}}\), and TCRP
values of \(1/\rho=\SI{0.39(9)}{\kelvin\watt^{-1}\metre^{2}}\), as reported in
table~\ref{tab:trcp}. One outlier was found in the dataset from the \cwm{} case, and
omitting this ensemble member results in \(\rho
=\SI{2.4(2)}{\watt\metre^{-2}\kelvin^{-1}}\) and \(1/\rho
=\SI{0.41(4)}{\kelvin\watt^{-1}\metre^{2}}\).

Even after \(20\) years, the temperature has not fully recovered, as seen in
Fig.~\ref{fig:compare-waveform-temp}. Yet, we assume the estimate of \(\rho
=\SI{2.4(2)}{\watt\metre^{-2}\kelvin^{-1}}\) to be a good estimate of \(\alpha \).
Importantly, the estimate of \(\alpha \simeq \SI{4}{\watt\metre^{-2}\kelvin^{-1}}\) by
\citeA{jones2005} is still significantly larger than our estimate, similar to what
\citeA{gregory2016} found.

Since the temperature perturbation obtained by J05 was higher than achieved here,
it indicates that the forcing used by J05 must be stronger. Even though the
AOD peak value used by J05 was \(100\) times that of Mt.\ Pinatubo, the
RF peak value was only about \(20\) times that of Mt.\ Pinatubo
\cite{gregory2016}. From the results shown here, this reduced aerosol forcing
efficiency for large volcanic eruptions is expected and akin to the RF dependency
on AOD found here. By inspection of Fig.~\ref{fig:parameter_scan} we find that the
aerosol forcing efficiency is somewhat smaller in J05. We therefore expect the
primary contributor to the overall increased forcing strength to originate from the
development of the forcing time series, not the magnitude.

\begin{table}
  \centering

  \caption{Estimated climate resistance and TCRP by use of the method outlined by
  \citeA{merlis2014}\(^{a}\)}\label{tab:trcp}%
  \begin{tabular}{ccc}
    \toprule
    Simulation type          & \(\rho [\si{\watt\metre^{-2}\kelvin^{-1}}]\) & \(1/\rho\)        \\
    \midrule
    \cws{}               & \(\num{2.2(1)}\)                             & \(\num{0.45(2)}\) \\
    \cwmp{}              & \(\num{2.5(1)}\)                             & \(\num{0.39(2)}\) \\
    \cwm{}               & \(\num{4(3)}\)                               & \(\num{0.3(1)}\)  \\
    \cwm{} (w/o outlier) & \(\num{2.5(3)}\)                             & \(\num{0.40(4)}\) \\
    Total                    & \(\num{3(2)}\)                               & \(\num{0.39(9)}\) \\
    Total (w/o outlier)      & \(\num{2.4(2)}\)                             & \(\num{0.41(4)}\) \\
    \toprule
    \multicolumn{3}{l}{\parbox{\linewidth}{\(^{a}\)Estimates are based on ensembles with four members and \(\tau
    =\SI{20}{\mathrm{yr}}\) using Eq.~\ref{eq:climate-resistance}. One ensemble member
    within \cwm{} had an outlier, and the same estimate but with this outlier removed is
indicated as ``w/o outlier'' (without outlier).}} \\
  \end{tabular}
\end{table}

% C2W^:         2.2+-0.1        0.45+-0.02
% C2W-:         2.5+-0.1        0.39+-0.02
% C2W_:         4+-3            0.3+-0.1
% C2W_ (1:):    2.5+-0.3        0.40+-0.04
% Total:        3+-2            0.39+-0.09
% Total (1:):   2.4+-0.2        0.41+-0.04

\section{Summary and conclusions}\label{sec:conclusions}

We consider three medium to super-volcano sized eruptions and compared them to
previously reported results. We find that the peak arrival in the AOD time series
is later post-eruption for larger volcanoes than smaller, and also that larger volcanoes
produce a sharper peak in the AOD time series. The RF time series are
similar across all volcano sizes, and while the smallest volcano experience a faster
temperature decay, the two larger volcanoes produce time series indistinguishable in
development for both RF and temperature. Thus, a simple scaling of the AOD
time series from a smaller volcano is insufficient in representing that of a larger
volcanic eruption.

We investigate the RF as a function of AOD, and find that an RF
dependence of \(\sim\SI{-20}{\watt\metre^{-2}\mathrm{AOD}^{-1}}\) is representative for
Mt.\ Pinatubo size volcanoes. Larger volcanoes with one to two orders of magnitude
larger injections of \ce{SO2} are found to have an RF dependence on AOD
closer to \(\sim \SI{-5}{\watt\metre^{-2}\mathrm{AOD}^{-1}}\). A more shallow slope for
larger volcanoes is also consistent with data from previous studies of super-volcanoes.

The time-after-eruption dependence of the ratio between RF and AOD is found
to weaken with time resulting in a reduced aerosol forcing efficiency. The effect is
found across all volcano sizes, but only the tropical cases show a clear trend. The
high-latitude case experience an almost constant efficiency with time. A similar
analysis has been carried out before by \citeA{marshall2020}, who found that the
efficiency increase with time when all eruptions were considered. However, we find that
when only their tropical eruptions are considered, a reduced efficiency is found, as
well as a similar ratio compared to volcanoes of similar size in our experiments. Thus,
it is evident that latitude generally is significant in determining the aerosol forcing
efficiency, and in particular as a function of time-after-eruption.

There is a large spread in the conversion between \iso{} and AOD and RF
across models, in particular among model families. Improving the consistency of the
chemistry and physics of \ce{SO2} and \ce{H2SO4} in models would be an important step in
improving the accuracy of volcanic eruptions influence on climate simulated by models.
Simulations of larger volcanic eruptions with \iso{} of at least
\(200\)--\(\SI{400}{\tera\gram(\mathrm{SO2})}\) would provide useful information for a
more precise determination of the evolution of the RF to AOD ratio. Allowing
for different latitudes, similar to the M20 dataset, would also be useful to study
if the function in Eq.~\ref{eq:niemeier_exponential} works as a lower bound on the
RF dependence on \iso{}, as indicated by Fig.~\ref{fig:parameter_scan}b.

%%% End of body of article

%%%%%%%%%%%%%%%%%%%%%%%%%%%%%%%%%%%%%%%%%%%%%%%
%% Optional Appendices go here
%
% The \appendix command resets counters and redefines section heads
%
% After typing \appendix
%
%\section{Here Is Appendix Title}
% will show
% A: Here Is Appendix Title
%
\appendix
\section{Simulation set up and output}

Input files used in the simulations were created by modifying the file
\url{http://svn.code.sf.net/p/codescripts/code/trunk/ncl/emission/createVolcEruptV3.ncl},
using a Python package available on GitHub at
\url{https://github.com/engeir/volcano-cooking} or the Python package manager PyPI\@.
The package is available both as a library and a CLI, and is used to create volcanoes
with a given \ce{SO2} amount that is injected during six
hours\footnote{\url{http://svn.code.sf.net/p/codescripts/code/trunk/ncl/emission/createVolcEruptV3.ncl}}
at a given latitude, longitude and altitude. All volcanic \ce{SO2} files are created
from a shell script by setting the eruption details in a ``.json'' file that is read to
the \texttt{volcano-cooking} CLI at a fixed version, ensuring a reproducible experiment
setup.

We are using the coupled model version \texttt{BWma1850} component
setup\footnote{\url{https://docs.cesm.ucar.edu/models/cesm2/config/2.1.0/compsets.html}}
to run the CESM2, and an accompanying fixed sea-surface temperature version,
\texttt{fSST1850}, to obtain estimates of the RF. The applied \texttt{fSST1850} is
not from a standardised component setup, but is instead explicitly
specified.\footnote{\fssturl} The component setup \texttt{BWma1850} and
\texttt{fSST1850} differ in that the latter uses a prescribed sea-ice (\texttt{CICE ->
CICE\%PRES}), a prescribed data ocean (\texttt{POP2\%ECO\%DEP -> DOCN\%DOM}) and a stub
wave component instead of the full WW3 (\texttt{WW3 -> SWAV}).

The important input data used in the model simulations are \iso{} in units of teragrams
(\(\si{\tera\gram(\ce{SO2})}\)), used to simulate volcanic eruptions. RF is
calculated as the combined (SW and LW) all-sky TOA energy imbalance,
where the CESM2 provide the output variables FSNT and FLNT. Thus,
\(\mathrm{RF_*}= \mathrm{FSNT} - \mathrm{FLNT}\), and taking the difference between
volcanic simulations and a control simulation gives the final estimate of RF
(\(\mathrm{RF}=\mathrm{RF_{VOLC}}-\mathrm{RF_{CONTROL}}\)) \cite{marshall2020}. The
RF calculation is based on \texttt{fSST1850}, hence this outline specifically
describe how to calculate ERF as opposed to IRF, which instead is the
difference between the ERF and the sum of all rapid atmospheric adjustments
\cite{marshall2020,smith2018}. The AOD is obtained from the output variable
AODM, while global temperature is saved by CESM2 to the variable
TREFHT. The analysis of this work is performed using these four variables. 

\section{External data}

\subsection{Otto-Bliesner data analysis}\label{ap:ob16}

Data from \citeA{ottobliesner2016} are the original input data of \iso{} as used in
their model simulations, along with RF and temperature output data. The \iso{} can
be downloaded with direct link
\url{https://svn-ccsm-inputdata.cgd.ucar.edu/trunk/inputdata/atm/cam/volc/IVI2LoadingLatHeight501-2000_L18_c20100518.nc},
or found at \url{https://www.cesm.ucar.edu/working-groups/paleo/simulations/ccsm4-lm}
and \url{https://svn-ccsm-inputdata.cgd.ucar.edu/trunk/inputdata/atm/cam/volc/}. Output
variables are available at
\url{https://www.earthsystemgrid.org/dataset/ucar.cgd.cesm2le.atm.proc.daily_ave.html}.

Since the OB16 dataset contain a five-member ensemble, the final RF and
temperature time series used were ensemble means. A single control simulation time
series is used to remove seasonal dependence from the temperature, where the control
simulation is averaged into a climatology mean. Further, a drift in the temperature is
removed by subtracting a linear regression fit. RF has seasonality removed in the
Fourier domain.

The time of an eruption is found based on a best attempt at aligning the \ce{SO2} time
series with both the RF time series and the temperature time series. The RF
and temperature peak values are taken as the value of the time series at the time of an
eruption according to the \ce{SO2} time series. Missing the true peak means the found
peaks will be biased towards lower values. However, instances where eruptions occur
close in time will contribute a bias to higher values. These biases contribute to a
greater uncertainty related to OB16 in Figs.~\ref{fig:parameter_scan}b,c,f.

\subsection{Marshall data analysis}\label{ap:m20}

Data used to generate the M20 values were from \citeA{marshall2020dataset},
available at \url{https://doi.org/10.5285/232164e8b1444978a41f2acf8bbbfe91}. As each
file includes a single eruption, peak values of AOD, RF and temperature were
found by applying a Savitzky-Golay filter of third order and one year window length, and
choosing the maximum value \cite{savitzky1964}.

\subsection{Gregory data analysis}\label{ap:g16}

Data used to generate G16 were kindly provided by Jonathan Gregory (personal
communication). The full 160-year long time series were further analysed by computing
annual means.

\section{Model families}

The model utilised here was the CESM2 which is an ancestor of CESM1 utilised
by OB16. They belong to a different model family than both the HadCM3 (J05
and G16) and the UM-UKCA (M20), which is an extended version of HadGEM3
\cite{dhomse2014}, and an ancestor of HadCM3. A third model family is represented
through ECHAM5 (N15) and MPI-ESM (T10), where the latter is related to the
former via the ECHAM6. A summary of the model code genealogy is detailed in
table~\ref{tab:model-family}, based on the model code genealogy map created by
\citeA{kuma2023}.

\begin{table*}
  \centering
  \caption{Overview of various model codes grouped into families according to the model
    code genealogy map by \citeA{kuma2023}, with each table entry also indicating the
    specific model code used in the referenced papers of this
    study.}\label{tab:model-family}

  \begin{tabular}{ccc}
    Family relation                                                         & Model name           & Data name  \\
    \multirow{2}{*}{CESM1 \(\rightarrow\) CESM1-CAM5 \(\rightarrow\) CESM2} & CESM1                & OB16 \\
                                                                            & CESM2
                                                                            & \emph{This
    contribution}                                                                                               \\
    \rowcolor{LightGray}                                                    & HadCM3
                                                                            & J05, G16              \\
    \rowcolor{LightGray}\multirow{-2}{*}{\shortstack{HadCM3 \(\rightarrow\) HadGEM1
    \(\rightarrow\)                                                                                             \\
    HadGEM2 \(\rightarrow\) HadGEM3 \(\rightarrow\) UM-UKCA}}               & UM-UKCA              &
    M20                                                                                                   \\
    \multirow{2}{*}{ECHAM5 \(\rightarrow\) ECHAM6 \(\rightarrow\) MPI-ESM}  & ECHAM5               &
    N15                                                                                                   \\
                                                                            & MPI-ESM              & T10  \\
  \end{tabular}
\end{table*}

%%%%%%%%%%%%%%%%%%%%%%%%%%%%%%%%%%%%%%%%%%%%%%%
% Optional Glossary, Notation or Acronym section goes here:
%
% Glossary is only allowed in Reviews of Geophysics
%  \begin{glossary}
%  \term{Term}
%   Term Definition here
%  \term{Term}
%   Term Definition here
%  \term{Term}
%   Term Definition here
%  \end{glossary}

%%%%%%%%%%%%%%%%%%%%%%%%%%%%%%%%%%%%%%%%%%%%%%%
% Acronyms
%% NOTE that acronyms in the final published version will be spelled out when used in figure captions.
\begin{acronyms}
% \newabbreviation{c2wmp}{C2W\(-\)}{CESM2(WACCM6) intermediate strength}
% \newabbreviation{c2wm}{C2W\(\downarrow\)}{CESM2(WACCM6) small strength}
% \newabbreviation{c2wsn}{C2WN\(\uparrow\)}{CESM2(WACCM6) large strength, high northern latitude}
% \newabbreviation{c2ws}{C2W\(\uparrow\)}{CESM2(WACCM6) large strength}
% \newabbreviation{c2w}{C2W}{CESM2(WACCM6) tropical}
% \newabbreviation{g16}{G16}{\citet{gregory2016}}
% \newabbreviation{j05}{J05}{\citet{jones2005}}
% \newabbreviation{m20}{M20}{\citet{marshall2020dataset}}
% \newabbreviation{n15}{N15}{\citet{niemeier2015}}
% \newabbreviation{ob16}{OB16}{\citet{ottobliesner2016}}
% \newabbreviation{t10}{T10}{\citet{timmreck2010}}
\acro{AGCM}
Atmosphere General Circulation Model
\acro{AODVISstdn}
``stratospheric aerosol optical depth 550 nm day night''
\acro{AOD}
stratospheric aerosol optical depth
\acro{AOGCM}
Atmosphere-Ocean General Circulation Model
\acro{CAM5}
Community Atmosphere Model Version 5
\acro{CAM6}
Community Atmosphere Model Version 6
\acro{CESM1}
Community Earth System Model Version 1
\acro{CESM2}
Community Earth System Model Version 2
\acro{CESM}
Community Earth System Model
\acro{CICE5}
CICE Version 5.1.2
\acro{CIME}
Common Infrastructure for Modelling the Earth
\acro{CISM2}
Community Ice Sheet Model Version 2.1
\acro{CLM5}
Community Land Model Version 5
\acro{ECS}
equilibrium climate sensitivity
\acro{ERF}
effective radiative forcing
\acro{ESM}
Earth System Model
\acro{FLNT}
``net longwave flux at the top of the model''
\acro{FPP}
Filtered Poisson Process
\acro{FSNT}
``net solar flux at the top of the model''
\acro{fSST1850}
fixed sea-surface temperature
\acro{IRF}
instantaneous radiative forcing
\acro{LWCF}
``long wave cloud forcing''
\acro{LW}
long wave
\acro{MAM3}
three mode version of the Modal Aerosol Module
\acro{MAM}
Modal Aerosol Module
\acro{MARBL}
MARine Biogeochemistry Library
\acro{MA}
middle atmosphere
\acro{MOSART}
MOdel for Scale Adaptive River Transport
\acro{POP2}
Parallel Ocean Program Version 2
\acro{QBO}
quasi-biennial oscillation
\acro{RF}
effective radiative forcing
\acro{SWCF}
``short wave cloud forcing''
\acro{SW}
short wave
\acro{TCRP}
transient climate response parameter
\acro{TOA}
top-of-the-atmosphere
\acro{TREFHT}
``reference height temperature''
\acro{WACCM6}
Whole Atmosphere Community Climate Model Version 6
\acro{WW3}
Wave Watch Version 3
\acro{YTT}
Young Toba Tuff
\acro{EMOS}
Ensemble model output statistics
\acro{ECMWF}
Centre for Medium-Range Weather Forecasts
\end{acronyms}

%%%%%%%%%%%%%%%%%%%%%%%%%%%%%%%%%%%%%%%%%%%%%%%
% Notation
%   \begin{notation}
%   \notation{$a+b$} Notation Definition here
%   \notation{$e=mc^2$}
%   Equation in German-born physicist Albert Einstein's theory of special
%  relativity that showed that the increased relativistic mass ($m$) of a
%  body comes from the energy of motion of the body—that is, its kinetic
%  energy ($E$)—divided by the speed of light squared ($c^2$).
%   \end{notation}

%%%%%%%%%%%%%%%%%%%%%%%%%%%%%%%%%%%%%%%%%%%%%%%
%
% DATA SECTION and ACKNOWLEDGMENTS
%
%%%%%%%%%%%%%%%%%%%%%%%%%%%%%%%%%%%%%%%%%%%%%%%

\section*{Open Research Section}

% This section MUST contain a statement that describes where the data supporting the
% conclusions can be obtained. Data cannot be listed as ''Available from authors'' or
% stored solely in supporting information. Citations to archived data should be included
% in your reference list. Wiley will publish it as a separate section on the paper’s page.
% Examples and complete information are here: https://www.agu.org/Publish with
% AGU/Publish/Author Resources/Data for Authors

Data generated directly from output fields of CESM2 are available at \emph{refer
  to Sigma2 archive}, and were generated using scripts available at
\url{https://github.com/engeir/cesm-data-aggregator}. Analysis scripts are available at
\url{https://github.com/engeir/paper1-code} and is published to
\url{https://zenodo.org/doi/10.5281/zenodo.10229427}. Source code used to generate
CESM2 input files are available at
\url{https://github.com/engeir/cesm2-volcano-setup}.

\add[editor]{Text to add}

\note[editor]{The note}

\annote[editor]{Text to annotate}{The note}

\add[editor]{Text to add}

\remove[editor]{Text to remove}

\change[editor]{Text to remove}{Text to add}

\acknowledgments

% Enter acknowledgments here. This section is to acknowledge funding, thank colleagues,
% enter any secondary affiliations, and so on.

The simulations were performed on resources provided by Sigma2 --- the National
Infrastructure for High Performance Computing and Data Storage in Norway.

%%%%%%%%%%%%%%%%%%%%%%%%%%%%%%%%%%%%%%%%%%%%%%%
% REFERENCES and BIBLIOGRAPHY
%
% \bibliography{<name of your .bib file>} don't specify the file extension
% don't specify bibliographystyle
%
%%%%%%%%%%%%%%%%%%%%%%%%%%%%%%%%%%%%%%%%%%%%%%%

\bibliography{references}

%Reference citation instructions and examples:
%
% Please use ONLY \cite and \citeA for reference citations.
% \cite for parenthetical references
% ...as shown in recent studies (Simpson et al., 2019)
% \citeA for in-text citations
% ...Simpson et al. (2019) have shown...
%
%
%...as shown by \citeA{jskilby}.
%...as shown by \citeA{lewin76}, \citeA{carson86}, \citeA{bartoldy02}, and \citeA{rinaldi03}.
%...has been shown \cite{jskilbye}.
%...has been shown \cite{lewin76,carson86,bartoldy02,rinaldi03}.
%... \cite <i.e.>[]{lewin76,carson86,bartoldy02,rinaldi03}.
%...has been shown by \cite <e.g.,>[and others]{lewin76}.
%
% apacite uses < > for prenotes and [ ] for postnotes
% DO NOT use other cite commands (e.g., \citet, \citep, \citeyear, \nocite, \citealp, etc.).
%

\end{document}

More Information and Advice:

%%%%%%%%%%%%%%%%%%%%%%%%%%%%%%%%%%%%%%%%%%%%%%%
%
%  SECTION HEADS
%
%%%%%%%%%%%%%%%%%%%%%%%%%%%%%%%%%%%%%%%%%%%%%%%

% Capitalize the first letter of each word (except for
% prepositions, conjunctions, and articles that are
% three or fewer letters).

% AGU follows standard outline style; therefore, there cannot be a section 1 without
% a section 2, or a section 2.3.1 without a section 2.3.2.
% Please make sure your section numbers are balanced.
% ---------------
% Level 1 head
%
% Use the \section{} command to identify level 1 heads;
% type the appropriate head wording between the curly
% brackets, as shown below.
%
%An example:
%\section{Level 1 Head: Introduction}
%
% ---------------
% Level 2 head
%
% Use the \subsection{} command to identify level 2 heads.
%An example:
%\subsection{Level 2 Head}
%
% ---------------
% Level 3 head
%
% Use the \subsubsection{} command to identify level 3 heads
%An example:
%\subsubsection{Level 3 Head}
%
%---------------
% Level 4 head
%
% Use the \subsubsubsection{} command to identify level 3 heads
% An example:
%\subsubsubsection{Level 4 Head} An example.
%
%%%%%%%%%%%%%%%%%%%%%%%%%%%%%%%%%%%%%%%%%%%%%%%
%
%  IN-TEXT LISTS
%
%%%%%%%%%%%%%%%%%%%%%%%%%%%%%%%%%%%%%%%%%%%%%%%
%
% Do not use bulleted lists; enumerated lists are okay.
% \begin{enumerate}
% \item
% \item
% \item
% \end{enumerate}
%
%%%%%%%%%%%%%%%%%%%%%%%%%%%%%%%%%%%%%%%%%%%%%%%
%
%  EQUATIONS
%
%%%%%%%%%%%%%%%%%%%%%%%%%%%%%%%%%%%%%%%%%%%%%%%

% Single-line equations are centered.
% Equation arrays will appear left-aligned.

Math coded inside display math mode \[ ...\]
will not be numbered, e.g.,:
\[ x^2=y^2 + z^2\]

Math coded inside \begin{equation} and \end{equation} will
be automatically numbered, e.g.,:
\begin{equation}
  x^2=y^2 + z^2
\end{equation}

% To create multiline equations, use the
% \begin{eqnarray} and \end{eqnarray} environment
% as demonstrated below.
\begin{eqnarray}
  x_{1} & = & (x - x_{0}) \cos \Theta \nonumber \\
  && + (y - y_{0}) \sin \Theta  \nonumber \\
  y_{1} & = & -(x - x_{0}) \sin \Theta \nonumber \\
  && + (y - y_{0}) \cos \Theta.
\end{eqnarray}

%If you don't want an equation number, use the star form:
%\begin{eqnarray*}...\end{eqnarray*}

% Break each line at a sign of operation
% (+, -, etc.) if possible, with the sign of operation
% on the new line.

% Indent second and subsequent lines to align with
% the first character following the equal sign on the
% first line.

% Use an \hspace{} command to insert horizontal space
% into your equation if necessary. Place an appropriate
% unit of measure between the curly braces, e.g.
% \hspace{1in}; you may have to experiment to achieve
% the correct amount of space.

%%%%%%%%%%%%%%%%%%%%%%%%%%%%%%%%%%%%%%%%%%%%%%%
%
%  EQUATION NUMBERING: COUNTER
%
%%%%%%%%%%%%%%%%%%%%%%%%%%%%%%%%%%%%%%%%%%%%%%%

% You may change equation numbering by resetting
% the equation counter or by explicitly numbering
% an equation.

% To explicitly number an equation, type \eqnum{}
% (with the desired number between the brackets)
% after the \begin{equation} or \begin{eqnarray}
% command.  The \eqnum{} command will affect only
% the equation it appears with; LaTeX will number
% any equations appearing later in the manuscript
% according to the equation counter.
%

% If you have a multiline equation that needs only
% one equation number, use a \nonumber command in
% front of the double backslashes (\\) as shown in
% the multiline equation above.

% If you are using line numbers, remember to surround
% equations with \begin{linenomath*}...\end{linenomath*}

%  To add line numbers to lines in equations:
%  \begin{linenomath*}
%  \begin{equation}
%  \end{equation}
%  \end{linenomath*}
