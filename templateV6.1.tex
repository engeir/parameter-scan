%% Version 6.1, 1 September 2021

% This tex file can be compiled with
% tectonic templateV5.tex
% https://tectonic-typesetting.github.io
%
% Or simply with the Overleaf latexmkrc configuration: (included in the repo)
% https://www.overleaf.com/learn/how-to/How_does_Overleaf_compile_my_project%3F
%
% A latexindent.yml config file is also included for easier and more consistent
% formatting.

%%%%%%%%%%%%%%%%%%%%%%%%%%%%%%%%%%%%%%%%%%%%%%%%%%%%%%%%%%%%%%%%%%%%%%
% TemplateV6.1.tex --  LaTeX-based blank template for submissions to the
% American Meteorological Society
%
%%%%%%%%%%%%%%%%%%%%%%%%%%%%%%%%%%%%%%%%%%%%%%%%%%%%%%%%%%%%%%%%%%%%%
% PREAMBLE
%%%%%%%%%%%%%%%%%%%%%%%%%%%%%%%%%%%%%%%%%%%%%%%%%%%%%%%%%%%%%%%%%%%%%

%% Start with one of the following:
% 1.5-SPACED VERSION FOR SUBMISSION TO THE AMS
\documentclass{ametsocV6.1}

% TWO-COLUMN JOURNAL PAGE LAYOUT---FOR AUTHOR USE ONLY
% \documentclass[twocol]{ametsocV6.1}

%%%%%%%%%%%%%%%%%%%%%%%%%%%%%%
% FOR PRINTING
\usepackage[a4paper]{geometry}
% MY ADDITIONS
\usepackage{colortbl}
\usepackage{multirow}
\usepackage[separate-uncertainty=true]{siunitx}
\usepackage[version=4]{mhchem}
\usepackage{glossaries}
\usepackage[automake]{glossaries-extra}
% \setacronymstyle{long-short}
\setabbreviationstyle[acronym]{long-short}
\setabbreviationstyle{short-long}
\makeglossaries{}
\newacronym{aod}{AOD}{aerosol optical depth}

% Create some custom commands
\newcommand{\iso}[1][i]{{#1}njected \ce{SO2}}
\definecolor{LightGray}{gray}{0.9}
% The content of the URL must be on its own line. The compiler works fine both ways, but
% the syntax highlighting is messed up by it.
\urldef\fssturl\url{
1850_CAM60%WCCM_CLM50%BGC-CROP_CICE%PRES_DOCN%DOM_MOSART_CISM2%NOEVOLVE_SWAV_TEST
}
%%%%%%%%%%%%%%%%%%%%%%%%%%%%%%

%%%%%%%%%%%%%%%%%%%%%%%%%%%%%%%%

%%% To be entered by author:

%% May use \\ to break lines in title:

\title{
From large to super-volcano eruptions: Ensemble simulations of climate responses to
diverse magnitudes of injected \ce{SO2}
}

%% Enter authors' names and affiliations as you see in the examples below.
%
%% Use \correspondingauthor{} and \thanks{} (\thanks command to be used for affiliations footnotes,
%% such as current affiliation, additional affiliation, deceased, co-first authors, etc.)
%% immediately following the appropriate author.
%
%% Note that the \correspondingauthor{} command is NECESSARY.
%% The \thanks{} commands are OPTIONAL.
%
%% Enter affiliations within the \affiliation{} field. Use \aff{#} to indicate the affiliation letter at both the
%% affiliation and at each author's name. Use \\ to insert line breaks to place each affiliation on its own line.

%\authors{Author One,\aff{a}\correspondingauthor{Author One, email@email.com}
%Author Two,\aff{a}
%Author Three,\aff{b}
%Author Four,\aff{a}
%Author Five\thanks{Author Five's current affiliation: NCAR, Boulder, Colorado},\aff{c}
%Author Six,\aff{c}
%Author Seven,\aff{d}
% and Author Eight\aff{a,d}
%}
%
%\affiliation{\aff{a}{First Affiliation}\\
%\aff{b}{Second Affiliation}\\
%\aff{c}{Third Affiliation}\\
%\aff{d}{Fourth Affiliation}
%}

\authors{
  Eirik Rolland Enger,\aff{a}\correspondingauthor{Eirik Rolland Enger, eirik.r.enger@uit.no}
  Rune Graversen,\aff{a}
  Audun Theodorsen,\aff{a}
  and Maria Rugenstein\aff{b}
}

\affiliation{
  \aff{a}{UiT The Arctic University of Norway, Tromsø, Norway}\\
  \aff{b}{Colorado State University, Fort Collins, Colorado}
}

%%%%%%%%%%%%%%%%%%%%%%%%%%%%%%%%%%%%%%%%%%%%%%%%%%%%%%%%%%%%%%%%%%%%%
% ABSTRACT
%
% Enter your abstract here
% Abstracts should not exceed 250 words in length!
%

\abstract{%
  Large to super-volcano sized eruptions are simulated in the \gls{cesm2} climate model
  with the \gls{waccm} atmosphere. Ensembles containing four members are simulated, and
  the \gls{aod}, \gls{rf} and temperature anomaly from the eruptions are compared.
  Simulating a Pinatubo-like eruption yields results consistent with several previous
  studies with a slope of \(\sim \SI{-20}{\watt\metre^{-2}\mathrm{AOD}^{-1}}\) for
  \gls{rf}. Larger eruptions, however, give a shallower slope indicative of a lower
  forcing efficiency. Moreover, a time-after-eruption dependence between \gls{aod} and
  \gls{rf} is found, where the \gls{rf} is relatively stronger than \gls{aod} the first
  year after the eruption (higher forcing efficiency), while later their ratio is
  roughly constant for a given eruption strength.
}

% NOTE: On efficiency: Pitari (Stratospheric ..) discuss this as global RF normalised to
% global optical thickness. Also M20.

\begin{document}

%% Necessary!
\maketitle{} \glsresetall{}

%%%%%%%%%%%%%%%%%%%%%%%%%%%%%%%%%%%%%%%%%%%%%%%%%%%%%%%%%%%%%%%%%%%%%
% SIGNIFICANCE STATEMENT/CAPSULE SUMMARY
%%%%%%%%%%%%%%%%%%%%%%%%%%%%%%%%%%%%%%%%%%%%%%%%%%%%%%%%%%%%%%%%%%%%%
%
% If you are including an optional significance statement for a journal article or a required capsule summary for BAMS
% (see www.ametsoc.org/ams/index.cfm/publications/authors/journal-and-bams-authors/formatting-and-manuscript-components for details),
% please apply the necessary command as shown below:
%
% Significance Statement (all journals except BAMS)
%
%\statement
%	 Enter significance statement here, no more than 120 words. See \url{www.ametsoc.org/index.cfm/ams/publications/author-information/significance-statements/} for details.
%
%% Capsule (BAMS only)
%%
%\capsule
%       Enter BAMS capsule here, no more than 30 words. See \url{www.ametsoc.org/index.cfm/ams/publications/author-information/formatting-and-manuscript-components/#capsule} for details.
%
%% * * If using twocol mode, you will need to use the commands "twocolsig" and "twocolcapsule" in place of "sig" and "capsule"
%%      to ensure that the text box correctly spans across both columns.
%

%%%%%%%%%%%%%%%%%%%%%%%%%%%%%%%%%%%%%%%%%%%%%%%%%%%%%%%%%%%%%%%%%%%%%
% MAIN BODY OF PAPER
%%%%%%%%%%%%%%%%%%%%%%%%%%%%%%%%%%%%%%%%%%%%%%%%%%%%%%%%%%%%%%%%%%%%%
%

%% In all cases, if there is only one entry of this type within
%% the higher level heading, use the star form:
%%
% \section{Section title}
% \subsection*{subsection}
% text...
% \section{Section title}

%vs

% \section{Section title}
% \subsection{subsection one}
% text...
% \subsection{subsection two}
% \section{Section title}

%%%
% \section{First primary heading}

% \subsection{First secondary heading}

% \subsubsection{First tertiary heading}

% \paragraph{First quaternary heading}

%%%%%%%%%%%%%%%%%%%%%%%%%%%%%%%%%%%%%%%%%%%%%%%%%%%%%%%%%%%%%%%%%%%%%
% TABLES---INSERT NEAR IN-TEXT DISCUSSION
%%%%%%%%%%%%%%%%%%%%%%%%%%%%%%%%%%%%%%%%%%%%%%%%%%%%%%%%%%%%%%%%%%%%%
%%  Enter tables near where they are discussed within the document.
%%  Please place tables before/after paragraphs, not within a paragraph.
%%
%
%\begin{table}[t]
%\caption{This is a sample table caption and table layout.  Enter as many tables as
%  necessary at the end of your manuscript. Table from Lorenz (1963).}\label{t1}
%\begin{center}
%\begin{tabular}{ccccrrcrc}
%\hline\hline
%$N$ & $X$ & $Y$ & $Z$\\
%\hline
% 0000 & 0000 & 0010 & 0000 \\
% 0005 & 0004 & 0012 & 0000 \\
% 0010 & 0009 & 0020 & 0000 \\
% 0015 & 0016 & 0036 & 0002 \\
% 0020 & 0030 & 0066 & 0007 \\
% 0025 & 0054 & 0115 & 0024 \\
%\hline
%\end{tabular}
%\end{center}
%\end{table}

%%%%%%%%%%%%%%%%%%%%%%%%%%%%%%%%%%%%%%%%%%%%%%%%%%%%%%%%%%%%%%%%%%%%%
% FIGURES---INSERT NEAR IN-TEXT DISCUSSION
%%%%%%%%%%%%%%%%%%%%%%%%%%%%%%%%%%%%%%%%%%%%%%%%%%%%%%%%%%%%%%%%%%%%%
%%  Enter figures near where they are discussed within the document.
%%  Please place figures before/after paragraphs, not within a paragraph.
% %
%
%\begin{figure}[t]
%  \noindent\includegraphics[width=19pc,angle=0]{figure01.pdf}\\
%  \caption{Enter the caption for your figure here.  Repeat as
%  necessary for each of your figures. Figure from \protect\cite{Knutti2008}.}\label{f1}
%\end{figure}

% NOTE: what to include in the paper, key questions.
% The paper should provide insight about what might happen if a large volcano erupted
% (order of magnitude or more than Mt.\ Pinatubo). How does the atmosphere react, for
% example in the aerosol dynamics? (QBO, SO2/AOD/RF relationship.) It should also be
% about how volcanic simulations compare in magnitude and if there is time for more
% simulations, how model complexity (dynamic ocean against slab ocean) affect things.
% - How far does the linear relation between AOD and RF go? What phases does the
%   aerosols go though? (Perhaps the most promising avenue.)
% - How much does it matter how high in the atmosphere the initial SO2 is injected?
%   (Already is some literature on this, suggesting it is not much. Also some on
%   latitude dependence, which has a bigger influence.)
% - How does the climate response change based on the state of the climate: what if we
%   run a CO2 doubling or quadrupling simulation until close to equilibrium, and let the
%   volcanoes erupt then? (Lack the doubling scenario, and setting it up has resulted in
%   strange output that must be resolved. Could take a while.)

\section{Introduction}

% NOTE: Suggested layout for the introduction
% - The objectives of the work.
% - The justification for these objectives: Why is the work important?
% - Background: Who else has done what? How? What have we done previously?
% - Guidance to the reader: What should the reader watch for in the paper? What are the
%   interesting high points? What strategy did we use?
% - Summary/conclusion: What should the reader expect as conclusion? In advanced
%   versions of the outline, you should also include all the sections that will go in
%   the Experimental section (at the level of paragraph subheadings) and indicate what
%   information will go.

% Toohey et al 2011 have a nice end of introduction.

\Gls{rf} and \gls{aod} are crucial metrics representing the energy imbalance at
\gls{toa} and the stratospheric opacity due to aerosol scattering. They are extensively
utilised to quantify the impact of major volcanic eruptions. The assumption of a linear
dependency of \gls{rf} on \gls{aod} is commonly adopted \citep{myhre2013,andersson2015},
and applying such a linear relationship have yielded reasonably accurate estimates in
climate model simulations of volcanic eruptions
\citep{mills2017,hansen2005,gregory2016,marshall2020,pitari2016b}. Despite this, there
is a wide spread in the estimated aerosol forcing efficiencies (\gls{rf} normalised by
\gls{aod}) among the studies, spanning approximately \(\sim
\SI{15}{\watt\metre^{-2}\ce{AOD}^{-1}}\) \citep{pitari2016b} to \(\sim
\SI{25}{\watt\metre^{-2}\ce{AOD}^{-1}}\) \citep{myhre2013}, predominantly based on
\gls{aod} values up to at most \(\sim 0.7\).

Although \ce{H2O}, \ce{N2} and \ce{CO2} are the most abundant gases emitted by volcanoes
\citep{robock2000}, sulphur species such as \ce{SO2} hold greater influence due to the
comparatively high concentrations of the former gases in the atmosphere. The
transformation of \ce{SO2} molecules through reactions with \ce{OH} and \ce{H2O} leads
to the formation of sulphate acid (\ce{H2SO4}) \citep{robock2000}, which scatter
sunlight, elevating planetary albedo and consequently reducing the \gls{rf}. As the
conversion from \ce{SO2} to \ce{H2SO4} occurs over weeks \citep{robock2000}, the peak
\gls{rf} from the eruption experiences a slight delay from the eruption's peak \ce{SO2}
injection. The duration of time the \ce{H2SO4} aerosols stay in the stratosphere depends
on various factors, including latitude \citep{marshall2019, toohey2019}, volcanic plume
height \citep{marshall2019}, aerosol size \citep{marshall2019}, the quasi-biennial
oscillation phase \citep{pitari2016b} and the season of the year (determining to which
hemisphere aerosols are transported) \citep{toohey2011,toohey2019}. In the case of
tropical eruptions, aerosols are typically transported poleward in the stratosphere and
back to mid-latitude troposphere within one to two years \citep{robock2000}. Upon
descending below the tropopause, these aerosols are readily removed by wet deposition
\citep{liu2012}.

Before the current era of significant anthropogenic climate forcing, volcanic eruptions
primarily dictated Earth's climate variability during the Holocene period
\citep{sigl2022}. Despite this substantial impact, few climate model experiments have
included volcanic forcing when simulating climate evolution during the Holocene
\citep{sigl2022}, implying an exaggerated positive forcing
\citep{gregory2016,solomon2011}. This absence of persistent cooling is one of several
factors that has been suggested to contribute to the common disparity between simulated
and observed global warming \citep{andersson2015}. Despite extensive attention on
understanding how volcanic eruptions influence climate, questions regarding aerosol
particle processes --- such as their growth and creation rates when \ce{OH} is scarce
--- remain unanswered
\citep[e.g.][]{robock2000,zanchettin2019,marshall2020,marshall2022}. These processes
impact aerosol scattering efficiency and potentially the \gls{rf} to \gls{aod}
relationship. \citet{marshall2020} observed higher efficiencies in years \(2\) and \(3\)
post-eruption compared to year 1, attributing this to spatial concentration in the
initial year and subsequent dispersion. This spatial alteration increases the albedo per
global mean \gls{aod}, consequently causing a stronger \gls{rf} to \gls{aod} ratio
\citep{marshall2020}.

Previous studies on both Mt.\ Pinatubo \citep{mills2017,hansen2005} and volcanoes within
the instrumental era \citep{gregory2016} have been used to estimate the relationship
between the \gls{rf} energy imbalance and change in \gls{aod} caused by volcanic
eruptions. While \citet{myhre2013} employ an \gls{rf} formula scaling \gls{aod} by
\(\SI{-25}{\watt\metre^{-2}\mathrm{AOD}^{-1}}\), recent literature reports estimates
down to \(\SI{-19.0(5)}{\watt\metre^{-2}\mathrm{AOD}^{-1}}\) \citep{gregory2016} and
\(\SI{-18.3(10)}{\watt\metre^{-2}\mathrm{AOD}^{-1}}\) \citep{mills2017}. Synthetic
volcano simulations in \citet{marshall2020} yield a scaling factor of
\(\SI{-20.5(2)}{\watt\metre^{-2}\mathrm{AOD}^{-1}}\) across 82 simulations featuring
varying injection heights and latitudes of volcanic emissions, with \iso{} ranging from
\(10\) to \(\SI{100}{\tera\gram(\ce{SO2})}\).

A similar simulation setup, albeit with notable differences, was conducted by
\citet{niemeier2015}, involving \(14\) levels of injected sulphur spanning between
\(\SI{1}{\tera\gram(\ce{S})\mathrm{yr}^{-1}}\)
(\(\SI{2}{\tera\gram(\ce{SO2})\mathrm{yr}^{-1}}\)) and
\(\SI{100}{\tera\gram(\ce{S})\mathrm{yr}^{-1}}\)
(\(\SI{200}{\tera\gram(\ce{SO2})\mathrm{yr}^{-1}}\)). These geoengineering simulations
maintained continous sulphur injections, running until a steady sulphur level was
achieved. Results indicated an inverse exponential relationship between \gls{rf} and
\iso{} rate, converging to \(\SI{-65}{\watt\metre^{-2}}\)
(eq.~\ref{eq:niemeier_exponential}). Notably, even the \(100\times\) Mt.\ Pinatubo
super-volcano simulation by \citet{jones2005}, who obtained a peak \gls{rf} of
\(\SI{-60}{\watt\metre^{-2}}\), is below the suggested limit of
\(\SI{-65}{\watt\metre^{-2}}\). Moreover, \citet{timmreck2010} find a peak \gls{rf}
anomaly of \(\SI{-18}{\watt\metre^{-2}}\) from a \(\SI{1700}{\tera\gram(\ce{SO2})}\)
eruption simulation, which corresponds well with the function estimated by
\citet{niemeier2015} at the given \ce{SO2} level.

One avenue that has garnered considerable attention is comparing the magnitude of
volcanic or volcano-like forcings to increased \ce{CO2} levels. Several studies explore
the connection between volcanic forcing and the climate sensitivity to a doubling of
\ce{CO2}
\citep{boer2007,marvel2016,merlis2014,ollila2016,richardson2019,salvi2022,wigley2005}.
This comparison aims to mitigate the large uncertainty in estimates of the sensitivity
of the real climate system. Inferring climate sensitivity from volcanic events has been
attempted as a means of constraining the sensitivity \citep{boer2007}, assuming that
volcanic and \ce{CO2} forcings produce similar feedbacks \citep{pauling2023}. Earlier
studies suggest the potential for constraining \gls{ecs} using volcanoes
\citep{bender2010}, provided that \gls{ecs} is constrained by \gls{erf} rather than
\gls{irf}, as \gls{erf} accounts for rapid atmospheric adjustments in contrast to
\gls{irf} \citep{richardson2019}. However, other studies refute this by indicating
different sensitivities between volcanic forcing and \ce{CO2} doubling
\citep{douglass2006} or asserting that constraining the \gls{ecs} by \gls{erf} lacks
accuracy due to the precision of climate simulations \citep{boer2007,salvi2022}.
Although \gls{erf} offers a more suitable indicator of forcing than \gls{irf}
\citep{marvel2016,richardson2019}, more recent studies conclude that \gls{ecs} cannot be
constrained from volcanic events \citep{pauling2023}.

Several studies have demonstrated a linear relationship of approximately
\(-\SI{20}{\watt\metre^{-2}\mathrm{AOD}^{-1}}\) between \gls{rf} and \gls{aod}, although
substantial variability exists in the slope among studies
\citep{mills2017,hansen2005,gregory2016,marshall2020,pitari2016b}. Moreover, a
time-after-eruption dependence on the \gls{rf} to \gls{aod} ratio is found in
\citet{marshall2020}, whereas \citet{niemeier2015} revealed a non-linear relationship
between \gls{rf} and \iso{}. Thus, a consensus on the relationship between \iso{},
\gls{aod} and \gls{rf} has yet to be established.

To address these issues we conducted ensemble simulations of volcanic eruptions in the
\gls{cesm2}, spanning \iso{} levels of three orders of magnitude:
\(\SI{26}{\tera\gram(\ce{SO2})}\), \(\SI{400}{\tera\gram(\ce{SO2})}\) and
\(\SI{1629}{\tera\gram(\ce{SO2})}\). Details regarding the experimental setup are
provided in section~\ref{sec:method}. Our findings reveal clear non-linear \gls{rf} to
\gls{aod} dependencies for large to super-volcano sized eruptions. Additionally, we
observe a time-dependent variation in the \gls{rf} to \gls{aod} ratio, detailed in
section~\ref{sec:results} and discussed in section~\ref{sec:discussion}. Furthermore,
our data, along with insights from previous studies, suggest that the \gls{rf} to \iso{}
dependency identified by \citet{niemeier2015} acts as a lower boundary. Our conclusions
are presented in section~\ref{sec:conclusions}.

\section{Method}\label{sec:method}

\subsection{Model}

We utilised the \gls{cesm2} \citep{danabasoglu2020} in conjunction with the \gls{waccm}
\citep{gettleman2019} and the fully dynamical ocean component \gls{pop}
\citep{smith2010, danabasoglu2020}. The atmosphere model was run at nominal
\(\SI{2}{\degree}\) resolution with \(70\) vertical levels in the \gls{ma}
configuration.

The \gls{waccm} version employed in the \gls{ma} configuration uses the \gls{mam3}
\citep{gettleman2019}, a simplified and computationally efficient default setting within
the \gls{cam5} \citep{liu2016}, as described in \citet{liu2012}. The \gls{mam3} was
developed from MAM7, consisting of the seven modes Aitken, accumulation, primary carbon,
fine dust, fine sea salt, coarse dust and coarse sea salt. Instantaneous internal mixing
of primary carbonaceous aerosols with secondary aerosols and instantaneous ageing of
primary carbonaceous particles is assumed by emitting primary carbon in the accumulation
mode \citep{liu2016}. Fine sea salt is assimilated into the accumulation mode, and as
dust absorbs water efficiently it is expected to be removed by wet deposition similarly
to sea salt, allowing fine dust to be merged into the accumulation mode as well.
Likewise, coarse dust is merged with coarse sea salt into a coarse mode, and as both
absorb water efficiently, the coarse mode will quickly retain its background state below
the tropopause \citep{liu2012}. Consequently, \gls{mam3} features the three modes
Aitken, accumulation and coarse \citep{liu2016}.

\subsection{Simulations}

Appendix A provides a detailed description of the simulation setup and utilised output
variables. Table~\ref{tab:simulation-overview} summarizes the simulations, encompassing
three \ce{SO2} injection magnitudes across four seasons: February 15th, May 15th, August
15th, and November 15th. The magnitudes vary over three orders of magnitude:
\(\SI{26}{\tera\gram(\ce{SO2})}\), \(\SI{400}{\tera\gram(\ce{SO2})}\), and
\(\SI{1629}{\tera\gram(\ce{SO2})}\).

The smallest eruption case, \gls{c2wm}, aligns with the magnitude of events like Mt.\
Pinatubo \citep[\(\sim10\)--\(\SI{20}{\tera\gram(\ce{SO2})}\);][]{timmreck2018} and Mt.\
Tambora \citep[\(\sim\SI{56.2}{\tera\gram(\ce{SO2})}\);][]{zanchettin2016}. The
intermediate case, \gls{c2wmp}, resembles the 1257 Samalas eruption
\citep[\(\sim{118.8}\)--\(\SI{173.1}{\tera\gram(\ce{SO2})}\);][]{toohey2017,ottobliesner2016},
while the largest eruption case, \gls{c2ws}, is similar to the \gls{ytt} eruption
\citep[\(100\)--\(\SI{10000}{\tera\gram(\ce{SO2})}\);][]{jones2005}. All eruptions were
situated at the equator (\(\SI{0}{\degree N}\), \(\SI{1}{\degree E}\)) with \ce{SO2}
injected between \(\SI{18}{\kilo\meter}\) and \(\SI{20}{\kilo\meter}\) altitude.
Collectively, the three eruption cases \gls{c2wm}, \gls{c2wmp} and \gls{c2ws} are
referred to as \gls{c2w}. Two additional high-latitude eruptions, labelled \gls{c2wsn},
of the same \iso{} magnitude as \gls{c2ws} were simulated at \(\SI{56}{\degree N}\),
\(\SI{287.7}{\degree E}\) with a six-month separation (February 15th and August 15th).

Employing eruptions in the large to super-volcano size enhances the signal-to-noise
ratio without necessitating an extensive and computationally expensive ensemble.
However, the forcing values obtained from the climate model simulations may not be
realistic, or more specifically, be an inappropriate representation of smaller eruptions
typical of the instrumental era \citep{gregory2016}.
% TODO: Mention this in relation to the different ratios for eruption size? In relation
% to J05 and my calculations of climate resistance.

\begin{table*}
  \centering

  \caption{Simulations done with the \gls{cesm2}. \gls{c2wsn} and \gls{c2ws} are the same
    in eruption magnitude, but while \gls{c2ws} is located at the equator, \gls{c2wsn} is
    located at high latitude. \gls{c2wmp} and \gls{c2wm} are located at the equator, but
    with different magnitudes to \gls{c2ws}}\label{tab:simulation-overview}%
  \begin{center}
    \begin{tabular}[c]{cccc}
      Name           & \(\si{\tera\gram(\ce{SO2})}\)         & Lat, lon, alt [\si{\degree\mathrm{N}}, \si{\degree\mathrm{E}}, \si{\kilo\metre}] &
      Eruption months                                                                                                                             \\
      \gls{c2wsn}    & \(1629\)                              &
      \(56\), \(287.7\),
      \(18\)--\(20\) & Feb,\hphantom{May,}Aug\hphantom{,Nov}                                                                                      \\
      \gls{c2ws}     & \(1629\)                              &
      \(\hphantom{1}0\), \(\hphantom{28}1\hphantom{.7}\), \(18\)--\(20\)
                     & Feb,May,Aug,Nov                                                                                                            \\
      \gls{c2wmp}    & \(\hphantom{1}400\)                   &
      \(\hphantom{1}0\),
      \(\hphantom{28}1\hphantom{.7}\),
      \(18\)--\(20\) & Feb,May,Aug,Nov                                                                                                            \\
      \gls{c2wm}     & \(\hphantom{14}26\)                   &
      \(\hphantom{1}0\),
      \(\hphantom{28}1\hphantom{.7}\), \(18\)--\(20\)
                     & Feb,May,Aug,Nov                                                                                                            \\
    \end{tabular}
  \end{center}
\end{table*}

\section{Results}\label{sec:results}

% NOTE: the results should be laid out in a logical way, with the most
% interesting/important stuff first, then tangents that dig deeper at specific things
% later.
% 1. RF to AOD time-after-eruption dependence should be top priority (8 figs atm.)
% 2. Then probably temperature scaling since we discuss the shape of both AOD and RF
%    time series before that (MOTIVATION: can we expect a specific temperature time
%    series shape based on the shape of either of or both of the RF and AOD time
%    series?)
% 3. If there is something interesting to say about the rest of the figures (all the
%    comparing of parameters), then this should come here.

\subsection{Analysis of the time series}

Fig.~\ref{fig:compare-waveform-temp} illustrates time series of global mean \gls{aod},
\gls{rf}, and surface air temperature. The black lines represent the medians across the
four-member ensembles, while shading indicates the 5th to 95th percentiles. Three
distinct forcing magnitudes (\gls{c2wm}, \gls{c2wmp}, and \gls{c2ws}), outlined in
table~\ref{tab:simulation-overview}, have been used. The time series in
fig.~\ref{fig:compare-waveform-temp} are normalised by setting the peak value to unity,
defined based on the peak of a fit utilising a Savitzky-Golay filter of 3rd order and a
one-year window length.

A notable feature across all three subfigures of fig.~\ref{fig:compare-waveform-temp} is
the earlier peak occurrence of the \gls{c2wm} case compared to the larger eruption
cases. Cases \gls{c2wmp} and \gls{c2ws} peak at similar times, but the \gls{c2wmp} case
exhibit a faster rise and slower decay in the \gls{aod} time series
(fig.~\ref{fig:compare-waveform-temp}a). Generally, the \gls{aod} time series from
stronger eruption cases seem to display a sharper peak, or similarly a slower rise and
faster decay. The rise across the eruption cases in
fig.~\ref{fig:compare-waveform-temp}b is similar, but where the \gls{c2wm} case reach
the peak just prior to the other two cases. During the decay phase, all cases appear to
decay at a similar rate, maintaining the offset caused by the earlier peak arrival of
\gls{c2wm}. Cases \gls{c2wmp} and \gls{c2ws} exhibit indistinguishable \gls{rf} time
series. Similarly, when observing the temperature evolution, cases \gls{c2wmp} and
\gls{c2ws} are indistinguishable. The \gls{c2wm} case spends less time around the peak
but more importantly decays much faster compared to the \gls{c2wmp} and \gls{c2ws}
cases.

Across all cases in each of the parameters shown in
fig.~\ref{fig:compare-waveform-temp}, no non-linear effects appear to have an impact,
even in the \gls{c2ws} super-volcano case. Comparing the two strongest eruption cases
(\gls{c2wmp} and \gls{c2ws}), where noise play a much smaller role, the \gls{rf} and
temperature time series are indistinguishable from each other. Therefore, similar
dynamics are expected to be at play in all cases.

\begin{figure}
  \centering
  \includegraphics{figures/figure1.png}

  \caption{\gls{aod} (a), \gls{rf} (b) and temperature response (c) time series to the
    three tropical volcanic eruption cases, \gls{c2wm}, \gls{c2wmp} and \gls{c2ws}. The time
    series have been normalised to have peak values at unity, where \(C\) is the
    normalisation constant. Black lines indicate the median across the four-member
    ensembles, while shading marks the 5th and 95th
    percentiles.}\label{fig:compare-waveform-temp}%
\end{figure}

Upon asking whether the shape of the temperature time series can be inferred from the
shape of either of the forcing time series (\gls{aod} or \gls{rf}), we find that the
shapes of the \gls{rf} time series are consistent over different eruption strengths
(fig.~\ref{fig:compare-waveform-temp}b), suggesting a strong dependence of temperature
on \gls{rf}. The same can largely be said about the \gls{aod} time series, but they show
a slight change in shape from smaller to larger eruptions
(fig.~\ref{fig:compare-waveform-temp}a). Specifically, the \gls{aod} time series from
smaller eruptions display a fast rise and a flat peak before decaying back to their
equilibrium state. From the larger eruptions, we find a slower rise time but a sharper
peak, resulting in a decay to equilibrium happening at a similar time after the eruption
and at a similar rate.

\subsection{\gls{rf} dependency on \gls{aod}}

We next focus on how the \gls{aod} and \gls{rf} time series develop relative to each
other. Similar comparisons were conducted in \citet[][their Fig.\ 4]{gregory2016} and
\citet[][their Fig.\ 1]{marshall2020}, with \gls{rf} plotted against \gls{aod}.
Fig.~\ref{fig:aod_vs_toa_ses_avg} displays annual mean values from the four simulation
cases in table~\ref{tab:simulation-overview}; the small eruption case (\gls{c2wm}) as
blue downward pointing triangles, the intermediate eruption case (\gls{c2wmp}) as orange
thick diamonds, the large tropical eruption case (\gls{c2ws}) as green upward pointing
triangles, and the large northern hemisphere eruption case (\gls{c2wsn}) as brown upward
pointing three-branched twigs. Also shown are the data from \citet[][Fig.\ 4, black
  crosses from HadCM3 sstPiHistVol]{gregory2016} as grey crosses labelled \gls{g16}
(described in Appendix B, section~\ref{ap:g16}). Additionally, the estimated peak values
from the Mt.\ Pinatubo and Mt.\ Tambora eruptions are plotted as a purple star and a
yellow plus, while the peak from the \citet{jones2005} simulation is shown as a pink
square labelled \gls{j05}. Finally, red circles represent the peak values obtained from
the \gls{c2w} tropical eruption cases. The gradient lines are the same as shown by
\citet{gregory2016}. The full data range is shown in fig.~\ref{fig:aod_vs_toa_ses_avg}a
while fig.~\ref{fig:aod_vs_toa_ses_avg}b highlights a narrow range, focusing on the
\gls{c2wm} case.

The annual mean data from the Pinatubo-like \gls{c2wm} case in
fig.~\ref{fig:aod_vs_toa_ses_avg}b have \gls{rf} values as a function of \gls{aod} that
follow almost the same constant gradient as the \gls{g16} data. However, in
fig.~\ref{fig:aod_vs_toa_ses_avg}a we observe that the stronger eruptions lead to
dissimilar responses in \gls{aod} and \gls{rf}, where the slope of the \gls{c2wmp} case
seems to follow close to a \(-10\) gradient and the \gls{c2ws} case is closer to a
\(-5\) gradient. The peak values (red circles) suggest a non-linear functional shape
dependence, while within each eruption strength (same colour) the annual mean values
fall relatively close to a straight line.

\begin{figure}
  \centering
  \includegraphics{figures/figure2.png}

  \caption{\gls{rf} as a function of \gls{aod}, yearly means. Data from the four
    simulations listed in table~\ref{tab:simulation-overview} (\gls{c2wm}, \gls{c2wmp},
    \gls{c2ws} and \gls{c2wsn}) are shown along with the data from the HadCM3 sstPiHistVol
    simulation by \citet{gregory2016} (grey crosses). Also shown are the estimated peak
    values of the Mt.\ Pinatubo (purple star) and Mt.\ Tambora (yellow plus) eruptions. In
    (a) the simulated super-volcano of \citet{jones2005} (pink square) is shown as well as
    the peak values from the simulations \gls{c2wm}, \gls{c2wmp} and \gls{c2ws} as red
    circles. All peak values (as opposed to annual means) have an asterisk (\(\ast{}\)) in
    their label. The grey gradient lines are the same regression fits as in \citet[][Fig.\
      4]{gregory2016}, where the solid line is the fit to the Gregory et al.\ data (grey
    crosses). (b): Zooming in on the smallest values.}\label{fig:aod_vs_toa_ses_avg}%
\end{figure}

To observe how the responses in \gls{aod} and \gls{rf} develop relative to each other
over time, we plot in fig.~\ref{fig:aod_vs_toa_avg_loop_ratios} seasonal means of the
\gls{rf} to \gls{aod} ratio, where the start of the time series is taken as that of the
eruption day. The plot shows all the eruption cases given in
table~\ref{tab:simulation-overview}, as well as the tropical eruptions from the
\citet{marshall2020dataset} dataset, labelled \gls{m20} and described in Appendix B,
section~\ref{ap:m20}. In fig.~\ref{fig:aod_vs_toa_avg_loop_ratios}a, straight lines are
linear regression fits to the seasonal means across all four ensembles, summarised in
table~\ref{tab:slope-gradients}. Shaded regions are the standard deviation around the
seasonal means. A similar shading is plotted in
fig.~\ref{fig:aod_vs_toa_avg_loop_ratios}b, but where the regression fits have been
omitted to avoid further cluttering the plot. The years where the signal-to-noise ratio
is the lowest are years \(1\) and \(2\), as well as year \(0\) (the noise is mostly due
to the \gls{rf} time series, shown in fig.~\ref{fig:compare-waveform-temp}b). For this
reason, the ratio of \gls{rf} to \gls{aod} is calculated for the second season of the
first year until the end of the third year.

Although the ratio changes between the eruption magnitudes, we find that the gradient at
which the ratio is changing is similar across large eruption magnitudes, as seen in
table~\ref{tab:slope-gradients}. A slope of around \(4\)--\(5\) during the first period
is a good fit for both the \gls{c2wmp} case and the \gls{c2ws} case (row
``\ref{fig:aod_vs_toa_avg_loop_ratios}a''), and even though the spread in the \gls{c2wm}
case is large in the \(y\)-direction, they tend to follow a similarly inclined, albeit
steeper, slope. Normalising the time series before computing the ratio, shown in
fig.~\ref{fig:aod_vs_toa_avg_loop_ratios}b, yields a similarly inclined slope across all
forcing magnitude ensembles except the \gls{c2wsn} case at high latitude. In the 2nd
period, from the second season of the second year, the ratios stay close to constant for
the reminder of the decaying phase of the time series. Again, this is necessarily the
case also in the normalised version in fig.~\ref{fig:aod_vs_toa_avg_loop_ratios}b, but
here the effect of a smaller signal-to-noise ratio into the third year becomes more
apparent.

\begin{table}
  \centering

  \caption{Gradient and standard deviation for the regression lines to the data found in
    fig.~\ref{fig:aod_vs_toa_avg_loop_ratios}. The regression fit in the top half of the
    table indicate the regression fits to fig.~\ref{fig:aod_vs_toa_avg_loop_ratios}a, while
    the bottom half is the regression fits to fig.~\ref{fig:aod_vs_toa_avg_loop_ratios}b.
    The columns ``1st period'' and ``2nd period'' refer to the period from year \(0\) to
    year \(1\)) and the period from year \(1\) to year \(3\).}\label{tab:slope-gradients}%
  \begin{tabular}{cccc}
    Figure                                                  & Simulation  & 1st period      & 2nd period       \\
    \rowcolor{LightGray}                                    & \gls{c2wsn} & \(0.34\pm1.20\) & \(0.68\pm1.34\)  \\
    \rowcolor{LightGray}                                    & \gls{c2ws}  & \(4.00\pm0.52\) & \(-3.00\pm0.55\) \\
    \rowcolor{LightGray}                                    & \gls{c2wmp} & \(4.84\pm0.83\) & \(-2.76\pm0.46\) \\
    \rowcolor{LightGray}                                    & \gls{c2wm}  & \(9.22\pm2.67\) & \(-0.55\pm2.51\) \\
    \rowcolor{LightGray}                                    & \gls{m20}   & \(6.34\pm1.77\) & \(-0.36\pm1.33\) \\
    \multirow{5}{*}{\ref{fig:aod_vs_toa_avg_loop_ratios}b}  & \gls{c2wsn} & \(0.06\pm0.21\) & \(0.12\pm0.24\)  \\
    \multirow{-9}{*}{\ref{fig:aod_vs_toa_avg_loop_ratios}a} & \gls{c2ws}  & \(0.78\pm0.10\) & \(-0.59\pm0.11\) \\
                                                            & \gls{c2wmp} & \(0.48\pm0.08\) & \(-0.28\pm0.05\) \\
                                                            & \gls{c2wm}  & \(0.43\pm0.13\) & \(-0.03\pm0.12\) \\
                                                            & \gls{m20}   & \(0.33\pm0.07\) & \(-0.02\pm0.08\) \\
  \end{tabular}
\end{table}

\citet[][their Fig.\ 1c,d]{marshall2020} present results that demonstrate a
time-dependent relationship in the conversion between \gls{aod} and \gls{rf}. Contrary
to the results from the \gls{cesm2} simulations, \gls{rf} displays larger magnitudes
compared to \gls{aod} at later stages in the eruption evolution (higher efficiency), not
smaller. This phenomenon is explained by \citet{marshall2020} as the aerosols initially
being spatially confined to the hemisphere where the eruption occurred. Subsequently,
during the second and third years, they spread globally, resulting in a higher global
mean albedo per \gls{aod} and consequently stronger \gls{rf} per \gls{aod} ratio with
time. When a similar analysis to that in fig.~\ref{fig:aod_vs_toa_avg_loop_ratios}a is
applied to the \gls{m20} data (utilised by \citet{marshall2020}), considering eruptions
between \(-10\) and \(\SI{10}{\degree\mathrm{N}}\), the ratio increase (weaker aerosol
forcing efficiency) similarly to the \gls{c2wmp} case rather than decrease (stronger
aerosol forcing efficiency). The slopes observed are approximately
\(\sim\SI{6.34}{\watt\metre^{-2}\mathrm{AOD}^{-1}}\) and
\(\sim\SI{-0.36}{\watt\metre^{-2}\mathrm{AOD}^{-1}}\), akin to the \gls{c2wm} case.
Additionally, we note that the slope obtained from our high-latitude case exhibits a
significantly weaker slope than the tropical cases, aligning with the \gls{m20} data.
Considering the \gls{m20} data and the results depicted in
fig.~\ref{fig:aod_vs_toa_avg_loop_ratios}, it appears that various and competing effects
determine the values of \gls{aod} and \gls{rf}, but with latitude playing a pivotal
role.

The change in ratio, where the weakest ratio (less negative) corresponds to larger
eruptions and the strongest ratio (more negative) corresponds to smaller eruptions,
aligns with the shift observed in peak values as illustrated in
fig.~\ref{fig:aod_vs_toa_ses_avg}a. Here, the red circles indicate that the peak values
saturate in \gls{rf}. The peak magnitudes of \gls{rf} increase at a slower rate compared
to the \gls{aod} peak magnitudes, which rise almost linearly with \iso{} (also visible
in fig.~\ref{fig:parameter_scan}a). This trend might be due to larger aerosols having
time to develop as the amount of \iso{} increases \citep{niemeier2015,marshall2019}.
Consequently, this leads to smaller eruptions' forcing being more efficient than that of
large eruptions, as smaller aerosols scatter radiation more efficiently, resulting in a
stronger (more negative) \gls{rf} to \gls{aod} ratio.

\begin{figure}
  \centering
  \includegraphics{figures/figure3.png}

  \caption{(a): Ratio of \gls{rf} to \gls{aod}, with time-after-eruption on the horizontal
    axis. Slopes are linear regression fits and are described in
    table~\ref{tab:slope-gradients}, while shaded regions are the standard deviation across
    the ensembles for each season. (b): Same as in (a), but where the underlying \gls{aod}
    and \gls{rf} time series have been scaled to have peak values at unity. Shown are data
    from table~\ref{tab:simulation-overview} along with tropical eruptions from
    \gls{m20}.}\label{fig:aod_vs_toa_avg_loop_ratios}%
\end{figure}

\subsection{Parameter scan}

In fig.~\ref{fig:parameter_scan}, we compare all relevant parameters against each other.
The primary input parameter in the \gls{cesm2} is \iso{}. For our tropical cases
(\gls{c2w}), we observe an almost linear relationship between \gls{aod} peak values and
\iso{}. The latitude also play a role for the extent of the \gls{aod} perturbation,
evident from the \gls{c2wsn} data point. This weak yet significant latitude dependence
aligns with findings by \citet{marshall2019}, indicating that \(\SI{72}{\percent}\) of
the \gls{aod} variance can be attributed to \iso{}, while latitude accounts for only
\(\SI{16}{\percent}\) of the variance. Peak values from their data (82 simulations)
plotted as red thin diamonds displays a similar pattern, with \gls{aod} exhibiting close
to linear dependence on \iso{}, but where latitude introduce a spread in \gls{aod}. Peak
values from Mt.\ Pinatubo (P) and Mt.\ Tambora (T) are shown for reference, along with
peak values from \gls{j05} and \gls{t10}.

The almost linear relationship between \gls{aod} and \iso{} for the \gls{c2w} data
suggest a comparable trend for \gls{rf} versus \iso{} as seen in \gls{rf} versus
\gls{aod}. In fig.~\ref{fig:parameter_scan}b, \gls{rf} plotted against \iso{} (with the
absolute value of \gls{rf} on the \(y\)-axis) indicates a substantial damping effect on
\gls{rf} as \iso{} increases for the \gls{c2w} data, aligning with expectations, with
the same behaviour seen in the \gls{ob16} data. The analysis details of the \gls{ob16}
data can be found in Appendix B, section~\ref{ap:ob16}. Despite the model complexity
difference, \citet{ottobliesner2016}'s simulations using \gls{cesm1} with a low-top
atmosphere (\gls{cam5}) produce \glspl{rf} comparable to our findings.

\begin{figure*}
  \centering
  \includegraphics{figures/figure4.png}

  \caption{(a) \gls{aod} (b) \gls{rf} and (c) temperature as a function of \iso{}\@. (d)
    \gls{rf} and (e) temperature as a function of \gls{aod}. (f) Temperature as a function
    of \gls{rf}. Blue diamonds labelled \gls{c2w} represent tropical cases (\gls{c2wm},
    \gls{c2wmp}, \gls{c2ws}), the brown three-branched twig signifies the \gls{c2wsn} case,
    and green downward triangles denote \gls{ob16} data from \citet{ottobliesner2016}. The
    red thin diamonds labelled \gls{m20} display the \citet{marshall2020dataset} data.
    Purple star and yellow plus indicate Mt.\ Pinatubo and Mt.\ Tambora estimates based on
    observations. The pink square labelled \gls{j05} refers to the one-hundred times Mt.\
    Pinatubo super-volcano from \citet{jones2005}, and the pink disk labelled \gls{t10}
    represents the \gls{ytt} super-volcano from \citet{timmreck2010}. The pink dashed line
    labelled \gls{n15} is from \citet{niemeier2015}, indicating the function in
    eq.~\ref{eq:niemeier_exponential}.}\label{fig:parameter_scan}%
\end{figure*}

% INFO: the conversion between S and SO2 is confirmed by Niemeier and Timmreck (2015)'s
% reference to the Bekki et al. (1996) paper. Bekki uses 6000 Mt SO2, Niemeier uses 3000
% Tg(S).
\citet{niemeier2015} conducted simulations of continuous sulphur injections up to
\(\SI{200}{\tera\gram(\ce{SO2})\mathrm{yr}^{-1}}\) in the ECHAM5's middle atmosphere
version \citep{giorgetta2006} with aerosol microphysics from HAM \citep{stier2005}. They
observed an \gls{rf} dependence on injection rate \(x\) following an inverse exponential
converging to \(\SI{-65}{\watt\meter^{-2}}\), depicted in fig.~\ref{fig:parameter_scan}b
as the stippled pink line and given as

\begin{equation}
  \Delta
  R_{\mathrm{TOA}} =
  -\SI{65}{\watt\metre^{-2}}
  \mathrm{e}^{-{\left(\frac{\SI{2246}{\tera\gram(S)yr^{-1}}}{x}\right)}^{0.23}}.
  \label{eq:niemeier_exponential}
\end{equation}

Both our simulations and the \gls{ob16} data exhibit a notably faster increase than the
exponential function. Interestingly, the simulations by \citet{timmreck2010} closely
correspond to the function presented in eq.~\ref{eq:niemeier_exponential}. Starting from
an initial input of \(\SI{850}{\tera\gram(\ce{S})}\) (equivalent to
\(\SI{1700}{\tera\gram(\ce{SO2})}\) and representing the \gls{ytt} eruption), their
estimated \gls{aod} led to a peak \gls{rf} of \(\SI{-18}{\watt\metre^{-2}}\) (pink
filled circle in fig.~\ref{fig:parameter_scan}b, labelled \gls{t10}), from a simulation
utilising the MPI-ESM and driven by \gls{aod} data from the HAM aerosol model. Thus, the
alignment likely stem from using the same aerosol microphysical model in both
\citet{timmreck2010} and \citet{niemeier2015}, alongside highly similar climate models
in the MPI-ESM and the ECHAM5 \citep{kuma2023}. The climate model family relations are
further examined in Appendix C. Notably, the peak values from the \gls{m20} dataset
align neatly within the upper boundary from the \gls{c2w} and \gls{ob16} data, and the
lower limit defined by eq.~\ref{eq:niemeier_exponential}. Eruptions closer to the
equator within the \gls{m20} dataset correspond to data points near the upper limit,
while eruptions at higher latitudes yield weaker peak \gls{rf} values closer to the
lower boundary. Crucially, none of the eruption simulations violated the suggested upper
threshold of \(\SI{-65}{\watt\metre^{-2}}\) as defined in
eq.~\ref{eq:niemeier_exponential}.

Figure~\ref{fig:parameter_scan}c illustrates the response of temperature against \iso{}.
Similar to fig.~\ref{fig:parameter_scan}b, the \gls{c2w} data deviates, indicating a
relatively weaker temperature response with increased \iso{} levels. Notably, the
\gls{ob16} dataset takes a different trajectory compared to the \gls{c2w} data,
showcasing less extreme temperature fluctuations as \iso{} rises. In contrast to our
findings, \gls{t10} demonstrates a considerably weaker temperature perturbation, noting
a maximum temperature anomaly of only \(\SI{-3.5}{\kelvin}\) for their
\(\SI{1700}{\tera\gram(\ce{SO2})}\) eruption, while \gls{j05} records a substantially
greater maximum temperature anomaly of \(\SI{-10.7}{\kelvin}\) compared to our \gls{c2w}
simulations.

Moving to fig.~\ref{fig:parameter_scan}d, we revisit the relationship between \gls{rf}
and \gls{aod}, focusing on peak values rather than annual and seasonal averages. As
previously discussed, the \gls{rf} to \gls{aod} ratio displays weaker gradients than
previous studies \citep{jones2005, marshall2020, timmreck2010}, with \gls{c2w} peak
values not conforming to a linear trend. This comparison suggests potential significant
dependencies on the model and its input parameters, such as latitude, but most notably
to an inherent non-linear \gls{rf} dependence on \gls{aod}. Both the \gls{g16} and
\gls{j05} data stem from the same climate model, and similarly to what we find from the
\gls{c2w} data, we find the ratio to be much stronger for small eruptions in the
industrial era (\gls{g16}) compared to the super-volcano eruption (\gls{j05}).

For fig.~\ref{fig:parameter_scan}e, the \gls{c2w} data is anticipated to resemble the
patterns observed in fig.~\ref{fig:parameter_scan}c due to the nearly linear association
identified between \gls{aod} and \iso{} in fig.~\ref{fig:parameter_scan}a. This is
indeed the case, and in addition both the \gls{c2wsn} and the \gls{j05} cases align
well, with the \gls{t10} case following a similar trend. The results from the \gls{m20}
data show temperature anomalies of smaller extent, similar to what was found in
fig.~\ref{fig:parameter_scan}c. However, the \gls{m20} data were conducted with
prescribed sea surface temperatures \citep{marshall2020}, preventing the temperature
from being fully perturbed.

Finally, in fig.~\ref{fig:parameter_scan}f, we compare the temperature and \gls{rf}
responses. Both the \gls{c2w} and \gls{ob16} datasets showcase a near-linear
relationship between temperature and \gls{rf}. The \gls{c2w} data indicates a steeper
gradient, potentially implying stronger temperature perturbations compared to
\gls{ob16}. However, there are potential biases in the values from the analysis of the
\gls{ob16} data, as outlined in Appendix B, section~\ref{ap:ob16}. This, along with
considerable noise, make the results from the analysis of the \gls{ob16} data less
reliable. As in fig.~\ref{fig:parameter_scan}e, the \gls{c2wsn} case along with the
\gls{j05} and \gls{t10} cases closely follow the trend set out by the \gls{c2w} cases.

\section{Discussion}\label{sec:discussion}

% NOTE: Suggested layout for the
% Discussion:
% - Explain the results and emphasize significant findings clearly
% - Discuss the impact and importance of results compared with recent relevant research
% Conclusion
% - The justification for these objectives: Why is the work important?
% - Summarize the key points made in the other sections
% - Conclude overall discussion of article
% - Link this section to the introduction

\subsection{Linearity between \gls{aod} and \gls{rf}}

Figures~\ref{fig:aod_vs_toa_ses_avg},~\ref{fig:aod_vs_toa_avg_loop_ratios} and
\ref{fig:parameter_scan}d collectively demonstrate that as the \gls{aod} surpasses
approximately \(1.0\), the scaling of approximately
\(\SI{-20}{\watt\metre^{-2}\mathrm{AOD}^{-1}}\) no longer corresponds to proportional
values of \gls{rf}. For the intermediate volcanic eruption (\gls{c2wmp}) at
\(\SI{400}{\tera\gram(\ce{SO2})}\), a mostly linear relationship between \gls{rf} and
\gls{aod} emerge, displaying a slope of roughly
\(\SI{-10}{\watt\metre^{-2}\mathrm{AOD}^{-1}}\). As for the largest eruption, an
\gls{rf} scaling of approximately \(\sim \SI{-5}{\watt\metre^{-2}\mathrm{AOD}^{-1}}\)
yields reasonable values. The almost linear relationship between \iso{} and \gls{aod} in
fig.~\ref{fig:parameter_scan}a indicates that larger eruptions injecting more \ce{SO2}
create larger aerosols, leading to less effective radiation scattering, thereby
influencing the \gls{rf} but not the \gls{aod} \citep{english2013, timmreck2010,
  timmreck2018}.

\citet{timmreck2010} highlights that for sufficiently large eruptions like Mt.\ Pinatubo
and \gls{ytt}, \ce{OH} radicals are not abundant enough, which limit \ce{SO2} oxidation.
The \gls{aod} peak in the \gls{ytt} simulation of \citet{timmreck2010} occur six months
after Mt.\ Pinatubo's peak. This observation aligns with our findings, as illustrated in
fig.~\ref{fig:compare-waveform-temp}a, where smaller eruptions show an earlier \gls{aod}
peak. While \citet{timmreck2010} reports a peak \gls{rf} anomaly occurring \(7\)--\(8\)
months post-eruption, \citet{jones2005} suggests a peak anomaly one year post-eruption.
The \gls{rf} peak preceding the \gls{aod} peak, approximately \(6\)--\(8\) months
post-eruption in \gls{cesm2} (see fig.~\ref{fig:compare-waveform-temp}b), aligns well
with what was found by \citet{timmreck2010} for the \gls{ytt}. Hence, both the \gls{aod}
and \gls{rf} time series appears influenced by \iso{} magnitude and the \ce{OH}
abundance, affecting the peak timing as well as the magnitude.

Although the \gls{j05} data is comparable to the \gls{c2ws} case concerning \gls{aod}
and \gls{rf} peak values, the temperature response reported by \citet{jones2005} appears
much stronger than what our strongest eruption produce. As \citet{jones2005} multiplies
the \gls{aod} time series from Mt.\ Pinatubo by one hundred to represent the \gls{aod}
time series of a super-volcano, this could potentially deviate significantly from the
real \gls{aod} time series of the super-volcano, both in shape and magnitude.
Subsequently, this may cause a substantially different temperature perturbation.
\citet{timmreck2010} obtained their \gls{aod} estimate from an initial injection of
\ce{SO2}, which resulted in a delayed peak but also a much smaller peak value compared
to the \gls{j05} data. Their maximum temperature perturbation is much smaller than the
\gls{j05} temperature perturbation, largely due to the big difference in \gls{aod}
magnitude. But as the \gls{j05} temperature was greater than the temperature
perturbation from the \gls{c2ws} case as well, it is expected that the shape of the
\gls{aod} time series is also significant in determining the strength of the aerosol
forcing and corresponding temperature perturbation.

The biggest spread in the data is found when converting from \iso{} to any of the three
output parameters when comparing across models. Conversion from \iso{} to \gls{aod} is
consistent within similar models, even when comparing simulations of volcanic eruptions
\citep{timmreck2010} and continuous injection of \ce{SO2} \citep{niemeier2015}, but has
a wide spread at large values of \iso{} across model families
(figs.~\ref{fig:parameter_scan}a,b,c). Comparatively, the \gls{rf}
(fig.~\ref{fig:parameter_scan}d) and temperature (fig.~\ref{fig:parameter_scan}e) as a
function of \gls{aod} demonstrate a smaller spread across models, and consequently, the
spread for temperature as a function of \gls{rf} (fig.~\ref{fig:parameter_scan}f) is
also small. Previous studies assumed a roughly linear relationship between \gls{rf} and
\gls{aod}, particularly for lower values of \gls{aod} and \gls{rf}, where the estimated
slope was notably steeper at around \(\SI{-20}{\watt\metre^{-2}\mathrm{AOD}^{-1}}\) for
\(\mathrm{AOD}<1\) compared to the approximately
\(\SI{-5}{\watt\metre^{-2}\mathrm{AOD}^{-1}}\) observed here at \(\mathrm{AOD}\gg1\).
Hence, a linear relationship appears to be an accurate estimate of \gls{rf} dependence
on \gls{aod} for eruptions similar to or smaller than Mt.\ Pinatubo. However, for larger
eruptions, factors like \ce{OH} scarcity and aerosol growth, influencing reflectance and
their gravitational pull, substantially impact both \gls{aod} and \gls{rf} evolution.

From the \gls{c2w} cases, a post-eruption time dependence on the \gls{rf} to \gls{aod}
ratio emerges. \citet{marshall2020} discusses a similar aspect, noting that while the
aerosol forcing efficiency strengthens from year 1 to year 2 when examining \gls{aod}
and \gls{rf}, here, the efficiency weakens over time, as depicted in
fig.~\ref{fig:aod_vs_toa_avg_loop_ratios}. This is by \citet{marshall2020} attributed to
the time taken for aerosol dispersion, affecting global albedo and consequently
\gls{rf}, whereas \gls{aod} is less affected by aerosol dispersion. Focusing solely on
tropical eruptions in \gls{m20} (between \(-10\) and \(\SI{10}{\degree\mathrm{N}}\)),
the \gls{rf} to \gls{aod} ratio closely resembles the findings from the \gls{c2wm} case.
Thus, while \iso{} is crucial for estimating the time-average of the \gls{rf} to
\gls{aod} ratio, latitude and specifically aerosol dispersion seem more influential in
determining the post-eruption evolution of the ratio. Given that the \gls{c2wsn} case
lack a significant increase in ratio akin to the tropical eruption cases (\gls{c2w}),
the substantial difference in eruption latitude appears to be a likely cause.

\citet{marshall2019, marshall2020, marshall2021} utilise a code with seven log-normal
modes to simulate aerosol mass and number concentrations, along with an atmosphere-only
configuration of the UM-UKCA with prescribed sea surface temperatures and sea ice extent
\citep{marshall2019}. This approach contrasts with \gls{cesm2}, which operates as an
\gls{esm} but with a simpler aerosol chemistry model in the \gls{mam3}. The family of
models the \gls{m20} data is from is different from the \gls{c2w} data, and also
different from the \gls{t10} and \gls{n15} data, as described in Appendix C. Based on
fig.~\ref{fig:parameter_scan}, the model family seems pivotal in determining the
estimated \gls{aod} and \gls{rf} magnitudes from \iso{}, whereas the various models
generally demonstrate more consistency in representing \gls{rf} from \gls{aod}. Given
that \gls{m20} employs a model from a distinct family compared to both \gls{ob16} and
\gls{c2w} cases, and \gls{t10} and \gls{n15}, while covering the parameter space between
the two families, it would be intriguing to include higher \iso{} values in the
\gls{m20} data to explore whether it remains bounded below (by \gls{t10} and \gls{n15})
and above (by \gls{ob16} and \gls{c2w}). This also prompts questions about whether
\iso{} saturation at a specific level, as illustrated in \citet{niemeier2015}, yields a
lower bound on the corresponding \gls{rf}, and similar to what a high-latitude eruption
might produce. Alternatively, differences in model aerosol chemistry might account for
the wide range in \gls{rf} as a function of \iso{}.

In summary, smaller eruptions and estimates from them produce a relatively substantial
\gls{rf} to \gls{aod} ratio (\(\sim \SI{-20}{\watt\metre^{-2}\mathrm{AOD}^{-1}}\)),
whereas larger eruptions result in estimates with smaller magnitudes (\(\sim
\SI{-10}{\watt\metre^{-2}\mathrm{AOD}^{-1}}\) to \(\sim
\SI{-5}{\watt\metre^{-2}\mathrm{AOD}^{-1}}\), as depicted in
fig.~\ref{fig:aod_vs_toa_avg_loop_ratios}). \citet{niemeier2017} indicates a decrease in
aerosol forcing efficiency as the injection rate increases, connected to how larger
volcanic eruptions lead to larger aerosol particles that scatter sunlight less
efficiently, thereby decreasing the forcing efficiency per \iso{} \citep{english2013,
  timmreck2018}.

\subsection{Climate sensitivity estimate}

As previously mentioned, the outcomes from \gls{j05} closely resemble our \gls{c2ws}
scenario concerning both \gls{aod} and \gls{rf} values, yet significantly differ in
temperature. To delve deeper into this discrepancy, we aim to conduct a comparison
between their reported climate feedback parameter \(\alpha \) (where \(s=1/\alpha \) is
the climate sensitivity parameter) with our estimations of climate resistance, denoted
as \(\rho \), and the \gls{tcrp} \(1/\rho\) (where \(\mathrm{TCS}=F_{2\times}\times
\mathrm{TCRP}\) is the transient climate sensitivity). Although volcanic eruption
forcing typically endures for about a year, a duration too brief for the timescales at
which \(F=\rho T\) remains valid \citep{gregory2016}, a workaround involves using a
time-integral form introduced by \citet{merlis2014}:

\begin{equation}
  \int_0^{\tau}F \mathrm{d}t=\rho\int_{0}^{\tau}T \mathrm{d}t
\end{equation}
\begin{equation}
  \rho=\frac{\int_0^{\tau}F \mathrm{d}t}{\int_{0}^{\tau}T \mathrm{d}t}.
  \label{eq:climate-resistance}
\end{equation}

If the upper bound of the integral, \(\tau \), is sufficiently large so that the upper
ocean heat capacity is the same at \(t=0\) and \(t=\tau \), this aligns with \(F=\rho
T\) \citep{gregory2016} (\citet{merlis2014} utilised \(\tau =\SI{15}{\mathrm{y}}\)).
Additionally, it's worth noting that the climate resistance and the climate feedback
parameter are associated with the ocean heat uptake efficiency (\(\kappa \)) through
\(\rho =\alpha +\kappa \).

The climate feedback parameter, as estimated by \citet{jones2005}, stands at
approximately \(\alpha \simeq \SI{4}{\watt\metre^{-2}\kelvin^{-1}}\), exceeding twice
the value obtained by \citet{gregory2016} in their simulations using Mt.\ Pinatubo
within the HadCM3 climate model.

We determine the climate resistance using the integral-form computation outlined in
eq.~\ref{eq:climate-resistance} and adopting \(\tau =\SI{8}{\mathrm{yr}}\), coinciding
with the duration of our simulations. The estimated climate resistance \(\rho \) from
the three tropical simulation cases (with four in each ensemble) yields
\(\SI{3(2)}{\watt\metre^{-2}\kelvin^{-1}}\), and \gls{tcrp} (\(1/\rho\)) values of
\(\SI{0.4(1)}{\kelvin\watt^{-1}\metre^{2}}\), as demonstrated in table~\ref{tab:trcp}.
One outlier was found in the dataset from the \gls{c2wm} case, and omitting this data
point leaves estimates of \(\SI{2.4(2)}{\watt\metre^{-2}\kelvin^{-1}}\) and
\(\SI{0.42(4)}{\kelvin\watt^{-1}\metre^{2}}\).

The climate resistance parameter \(\rho\) differs from the climate feedback parameter
estimated by \citet{jones2005} (\(\alpha\)). However, considering both \(\alpha\) and
\(\kappa\) are positive, and our calculated values of \(\rho(=\alpha+\kappa)\) are
smaller than the \citet{jones2005} estimate of \(\alpha \simeq
\SI{4}{\watt\metre^{-2}\kelvin^{-1}}\), we deduce that the climate feedback parameter
linked to the simulations conducted here must be notably lower than what
\citet{jones2005} found. It is worth noting that the temperature time series are not
fully equilibrated after eight years, but as integrating over a longer time would only
reduce the estimated values of \(\rho\), it is expected that this contribution is
unimportant when comparing our \(\rho\) to the \(\alpha\) used in \gls{j05}.

Since the temperature perturbation obtained by \gls{j05} was higher than any achieved
here, it indicates that the forcing used by \gls{j05} must be stronger.
\citet{gregory2016} mentioned that the peak forcing value was about \(20\) times smaller
than expected, but from the results shown here, this reduced aerosol forcing efficiency
is expected and akin to the \gls{rf} dependency on \gls{aod} found here. Inspecting
fig.~\ref{fig:parameter_scan} we find that the aerosol forcing efficiency is somewhat
smaller in \gls{j05}. We therefore expect the primary contributor to the overall
increased forcing strength to stem from the shape of the forcing time series utilised,
not the magnitude.

\begin{table}
  \centering

  \caption{Estimated climate resistance and \gls{tcrp} by use of the method outlined by
    \citet{merlis2014}. Estimates are based on ensembles with four members, and where \(\tau
    =\SI{8}{\mathrm{yr}}\) in eq.~\ref{eq:climate-resistance}. One data point within the
    \gls{c2wm} case had an outlier, and the same estimate but with the outliter removed is
    indicated as ``w/o outliter'' (without outlier).}\label{tab:trcp}%
  \begin{tabular}{ccc}
    Simulation type          & \(\rho [\si{\watt\metre^{-2}\kelvin^{-1}}]\) & \(1/\rho\)        \\
    \gls{c2ws}               & \(\num{2.2(1)}\)                             & \(\num{0.46(3)}\) \\
    \gls{c2wmp}              & \(\num{2.5(1)}\)                             & \(\num{0.41(2)}\) \\
    \gls{c2wm}               & \(\num{5(4)}\)                               & \(\num{0.3(1)}\)  \\
    \gls{c2wm} (w/o outlier) & \(\num{2.6(2)}\)                             & \(\num{0.39(3)}\) \\
    Total                    & \(\num{3(2)}\)                               & \(\num{0.4(1)}\)  \\
    Total (w/o outlier)      & \(\num{2.4(2)}\)                             & \(\num{0.42(4)}\) \\
  \end{tabular}
\end{table}

% C2W^:         2.2+-0.1        0.46+-0.03
% C2W-:         2.5+-0.1        0.41+-0.02
% C2W_:         5+-4            0.3+-0.1
% C2W_ (1:):    2.6+-0.2        0.39+-0.03
% Total:        3+-2            0.4+-0.1
% Total (1:):   2.4+-0.2        0.42+-0.04

\section{Summary and conclusions}\label{sec:conclusions}

In this paper we considered three large to super-volcano sized eruptions and compared
them to previously reported results. We find that the peak arrival in the \gls{aod} time
series is later post-eruption for larger volcanoes than smaller, and also that larger
volcanoes produce a sharper peak in the \gls{aod} time series. The \gls{rf} time series
are similar across all volcano sizes, and while the smallest volcano experience a faster
temperature decay, the two larger volcanoes produce time series indistinguishable in
shape for both \gls{rf} and temperature. Thus, a simple scaling of the \gls{aod} time
series from a smaller volcano is insufficient in representing a larger volcanic
eruption.

We investigate the \gls{rf} as a function of \gls{aod}, and find that an \gls{rf}
dependence of \(\sim\SI{-20}{\watt\metre^{-2}\mathrm{AOD}^{-1}}\) is consistent with our
results for volcanoes of similar size in terms of \iso{} as Mt.\ Pinatubo. Larger
volcanoes with one to two orders of magnitude more \iso{} is found to produce a much
more shallow \gls{rf} gradient closer to \(\sim
\SI{-5}{\watt\metre^{-2}\mathrm{AOD}^{-1}}\). A more shallow gradient for larger
volcanoes is also consistent with data from previous studies of super-volcanoes.

The time-after-eruption dependence of the ratio between \gls{rf} and \gls{aod} is found
to weaken with time resulting in a reduced aerosol forcing efficiency. The effect is
found across all volcano sizes, but only the tropical cases show a clear trend. The
high-latitude case experience an almost constant efficiency with time. A similar
analysis has been carried out before by \citet{marshall2020}, who found that the
efficiency increase with time when all eruptions were considered. However, we find that
when only their tropical eruptions are considered, a reduced efficiency is found as well
as a similar ratio to what our volcanoes of similar size produced. Thus, the results
align well when comparing tropical eruptions, and it is evident that latitude is
significant in deciding the aerosol forcing efficiency generally, and as a function of
time-after-eruption specifically.

There is a large spread in the conversion between \iso{} and \gls{aod} and \gls{rf}
across models and in particular model families. Improving the consistency in how
\ce{SO2} and \ce{H2SO4} is treated in models would be an important step in improving the
accuracy of volcanic eruptions influence on climate. Simulations of larger volcanic
eruptions with \iso{} of at least \(200\)--\(\SI{400}{\tera\gram(\mathrm{SO2})}\) would
provide useful data for a more precise determination of the functional shape of the
\gls{rf} to \gls{aod} ratio. Allowing for different latitudes, similar to the \gls{m20}
dataset, would also provide fruitful data to study more closely whether the functional
given in eq.~\ref{eq:niemeier_exponential} is indeed a lower bound for \gls{rf} as a
function of \iso{}.

\clearpage
%%%%%%%%%%%%%%%%%%%%%%%%%%%%%%%%%%%%%%%%%%%%%%%%%%%%%%%%%%%%%%%%%%%%%
% ACKNOWLEDGMENTS
%%%%%%%%%%%%%%%%%%%%%%%%%%%%%%%%%%%%%%%%%%%%%%%%%%%%%%%%%%%%%%%%%%%%%
\acknowledgments{}
%  Keep acknowledgments (note correct spelling: no ``e'' between the ``g'' and
% ``m'') as brief as possible. In general, acknowledge only direct help in
%  writing or research. Financial support (e.g., grant numbers) for the work done,
%  for an author, or for the laboratory where the work was performed must be
%  acknowledged here rather than as footnotes to the title or to an author's name.
%  Contribution numbers (if the work has been published by the author's institution
%  or organization) should be placed in the acknowledgments rather than as
%  footnotes to the title or to an author's name.

% https://www.sigma2.no/acknowledgements
The simulations were performed on resources provided by Sigma2 --- the National
Infrastructure for High Performance Computing and Data Storage in Norway.

%%%%%%%%%%%%%%%%%%%%%%%%%%%%%%%%%%%%%%%%%%%%%%%%%%%%%%%%%%%%%%%%%%%%%
% DATA AVAILABILITY STATEMENT
%%%%%%%%%%%%%%%%%%%%%%%%%%%%%%%%%%%%%%%%%%%%%%%%%%%%%%%%%%%%%%%%%%%%%
%
%
\datastatement{}
%  The data availability statement is where authors should describe how the data underlying
%  the findings within the article can be accessed and reused. Authors should attempt to
%  provide unrestricted access to all data and materials underlying reported findings.
%  If data access is restricted, authors must mention this in the statement. See
%  {http://www.ametsoc.org/PubsDataPolicy} for more info.

% https://documentation.sigma2.no/nird_archive/user-guide.html
Data generated directly from output fields of \gls{cesm2} are available at \emph{refer
  to Sigma2 archive}, and were generated using scripts available at
\url{https://github.com/engeir/cesm-data-aggregator}. Analysis scripts are available at
\url{https://github.com/engeir/paper1-code} and is published to
\url{https://zenodo.org/doi/10.5281/zenodo.10229427}. Source code used to generate
\gls{cesm2} input files are available at
\url{https://github.com/engeir/cesm2-volcano-setup}.

%%%%%%%%%%%%%%%%%%%%%%%%%%%%%%%%%%%%%%%%%%%%%%%%%%%%%%%%%%%%%%%%%%%%%
% APPENDIXES
% https://www.ametsoc.org/index.cfm/ams/publications/author-information/latex-author-info/documentation-for-ams-latex-template1/
%%%%%%%%%%%%%%%%%%%%%%%%%%%%%%%%%%%%%%%%%%%%%%%%%%%%%%%%%%%%%%%%%%%%%
%
%% If only one appendix, use

%\appendix

%% If more than one appendix, use \appendix[<letter>], e.g.,

%\appendix[A]

%% Appendix title is necessary! For appendix title:

%\appendixtitle{Title of Appendix}

%%% Appendix section numbering (note, skip \section and begin with \subsection)
%
% \subsection{First primary heading}

% \subsubsection{First secondary heading}

% \paragraph{First tertiary heading}

\appendix

\appendix[A]

\appendixtitle{Simulation set up and output}

Input files used in the simulations were created using a modified version of the file
\url{http://svn.code.sf.net/p/codescripts/code/trunk/ncl/emission/createVolcEruptV3.ncl},
via a Python project developed on GitHub at
\url{https://github.com/engeir/volcano-cooking}. The project is also available from the
Python package manager PyPI\@. The program creates volcanoes with a given \ce{SO2}
amount that is injected over six
hours\footnote{\url{http://svn.code.sf.net/p/codescripts/code/trunk/ncl/emission/createVolcEruptV3.ncl}}
at a given latitude, longitude and altitude. All volcanic \ce{SO2} files are created by
setting the eruption details in a~.json file that is read to the
\texttt{volcano-cooking} CLI at a fixed version, making for a reproducible experiment
setup.

We are using the \texttt{BWma1850} component
setup\footnote{\url{https://docs.cesm.ucar.edu/models/cesm2/config/2.1.0/compsets.html}}
to run the \gls{cesm2}, and an accompanying \gls{fsst} simulation to obtain estimates of
the \gls{rf}. The \gls{fsst} simulation used is not from a standardised component setup
as of \gls{cesm2} (v2.1.3), but is instead specified in full.\footnote{\fssturl} They
differ in \texttt{CICE -> CICE\%PRES}, which is prescribed sea-ice,
\texttt{POP2\%ECO\%DEP -> DOCN\%DOM} which is from a dynamical ocean to a prescribed
data ocean and the wave component \texttt{WW3 -> SWAV} which is now a stub wave
component instead of the full \gls{ww3}.

\gls{rf} is calculated as the combined (\gls{sw} and \gls{lw}) all-sky \gls{toa} energy
imbalance, where the \gls{cesm2} provide the output variables \gls{fsnt} and \gls{flnt}.
Thus, \(\mathrm{RF_*}= \mathrm{FSNT} - \mathrm{FLNT}\), and taking the difference
between volcanic simulations and a control simulation gives the final estimate of
\gls{rf} (\(\mathrm{RF}=\mathrm{RF_{VOLC}}-\mathrm{RF_{CONTROL}}\))
\citep{marshall2020}. The output parameters that go into this estimate are from he
\gls{fsst} simulation, hence this outline specifically describe how to calculate
\gls{erf} as opposed to \gls{irf}, which instead is the difference between the \gls{erf}
and the sum of all rapid atmospheric adjustments \citep{marshall2020,smith2018}. The
\gls{aod} is obtained from the output variable \gls{aodm}, while global temperature is
saved by \gls{cesm2} to the variable \gls{trefht}. These four output variables are all
that are used throughout this paper. The important input data used in the model
simulations are \iso{} in units of teragrams (\(\si{\tera\gram(\ce{SO2})}\)), used to
simulate volcanic eruptions.

\appendix[B]

\appendixtitle{External data}

\subsection{Otto-Bliesner data analysis}\label{ap:ob16}

Data from \citet{ottobliesner2016} are the original input data of \iso{} as used in
their model simulations, where corresponding \gls{rf} and temperature data are found as
the value of the time series at the time of an eruption (according to the \iso{} time
series). Therefore, \gls{rf} and temperature values may be somewhat smaller in
figs.~\ref{fig:parameter_scan}b,c,f than their true value. Specifically, an ensemble of
5 is used for both \gls{rf} and temperature, and a mean from the 5 is used as the de
facto \gls{rf} and temperature time series. A control simulation of a single time series
is used to remove seasonal dependence from the temperature, where the control simulation
is averaged into a climatology mean. Further, a drift in the temperature is removed by
subtracting a linear regression fit. \gls{rf} have seasonality removed in the Fourier
domain. The forcing (\ce{SO2}) can be downloaded with direct link
\url{https://svn-ccsm-inputdata.cgd.ucar.edu/trunk/inputdata/atm/cam/volc/IVI2LoadingLatHeight501-2000_L18_c20100518.nc},
or found at \url{https://www.cesm.ucar.edu/working-groups/paleo/simulations/ccsm4-lm}
and \url{https://svn-ccsm-inputdata.cgd.ucar.edu/trunk/inputdata/atm/cam/volc/}.

The time of an eruption is decided based on a best attempt at aligning the \ce{SO2} time
series with both the \gls{rf} time series and the temperature time series individually.
The \gls{rf} and temperature values used in the \gls{ob16} data points are the exact
values at day 340 (\gls{rf}) and 400 (temperature) after the eruption according to the
\iso{} time series. Thus, missing the peak means the estimates will be biased towards
lower values, while other eruptions occurring close in time will contribute a bias to
higher values.

\subsection{Marshall data analysis}\label{ap:m20}

Data used to generate the \gls{m20} data points was from \citet{marshall2020dataset},
available at \url{https://doi.org/10.5285/232164e8b1444978a41f2acf8bbbfe91}. As each
file contain a single eruption, peak values of \gls{aod}, \gls{rf} and temperature was
found by applying a Savitzky-Golay filter of third order and one year window length, and
choosing the maximum value.

\subsection{Gregory data analysis}\label{ap:g16}

Data used to generate the \gls{g16} data points was kindly provided by Jonathan Gregory
(personal communication). The full 160-year long time series were further analysed by
computing annual means.

\appendix[C]

\appendixtitle{Model families}

The model utilised here was the \gls{cesm2} which is an ancestor of \gls{cesm1} utilised
by \gls{ob16}. They belong to a different model family than both the HadCM3 (\gls{j05}
and \gls{g16}) and the UM-UKCA (\gls{m20}), which is an extended version of HadGEM3
\citep{dhomse2014}, and an ancestor of HadCM3. A third model family is represented
through ECHAM5 (\gls{n15}) and MPI-ESM (\gls{t10}), where the latter is related to the
former via the ECHAM6. A summary of the model code genealogy is detailed in
table~\ref{tab:model-family}, based on the model code genealogy map created by
\citet{kuma2023}.

\begin{table*}
  \centering
  \caption{Overview of various model codes grouped into families according to the model
    code genealogy map by \citet{kuma2023}, with each table entry also indicating the
    specific model code used in the referenced papers of this
    study.}\label{tab:model-family}

  \begin{tabular}{ccc}
    Family relation                                                         & Model name           & Data point \\
    \multirow{2}{*}{CESM1 \(\rightarrow\) CESM1-CAM5 \(\rightarrow\) CESM2} & CESM1                & \gls{ob16} \\
                                                                            & CESM2
                                                                            & \emph{This
    contribution}                                                                                               \\
    \rowcolor{LightGray}                                                    & HadCM3
                                                                            & \gls{j05}, \gls{g16}              \\
    \rowcolor{LightGray}\multirow{-2}{*}{\shortstack{HadCM3 \(\rightarrow\) HadGEM1
    \(\rightarrow\)                                                                                             \\
    HadGEM2 \(\rightarrow\) HadGEM3 \(\rightarrow\) UM-UKCA}}               & UM-UKCA              &
    \gls{m20}                                                                                                   \\
    \multirow{2}{*}{ECHAM5 \(\rightarrow\) ECHAM6 \(\rightarrow\) MPI-ESM}  & ECHAM5               &
    \gls{n15}                                                                                                   \\
                                                                            & MPI-ESM              & \gls{t10}  \\
  \end{tabular}
\end{table*}

%%%%%%%%%%%%%%%%%%%%%%%%%%%%%%%%%%%%%%%%%%%%%%%%%%%%%%%%%%%%%%%%%%%%%
% REFERENCES
%%%%%%%%%%%%%%%%%%%%%%%%%%%%%%%%%%%%%%%%%%%%%%%%%%%%%%%%%%%%%%%%%%%%%
% Make your BibTeX bibliography by using these commands:
% \bibliographystyle{ametsocV6}
% \bibliography{references}

\bibliographystyle{ametsocV6} \bibliography{references} \clearpage
\printglossary[type=\acronymtype,title=List of Acronyms]

\end{document}
