%% Version 6.1, 1 September 2021
%
%%%%%%%%%%%%%%%%%%%%%%%%%%%%%%%%%%%%%%%%%%%%%%%%%%%%%%%%%%%%%%%%%%%%%%
% TemplateV6.1.tex --  LaTeX-based blank template for submissions to the 
% American Meteorological Society
%
%%%%%%%%%%%%%%%%%%%%%%%%%%%%%%%%%%%%%%%%%%%%%%%%%%%%%%%%%%%%%%%%%%%%%
% PREAMBLE
%%%%%%%%%%%%%%%%%%%%%%%%%%%%%%%%%%%%%%%%%%%%%%%%%%%%%%%%%%%%%%%%%%%%%


%% Start with one of the following:
% 1.5-SPACED VERSION FOR SUBMISSION TO THE AMS
% \documentclass{ametsocV6.1}


% TWO-COLUMN JOURNAL PAGE LAYOUT---FOR AUTHOR USE ONLY
\documentclass[twocol]{ametsocV6.1}

%%%%%%%%%%%%%%%%%%%%%%%%%%%%%%
% MY ADDITIONS
\usepackage{multirow}
\usepackage[separate-uncertainty=true]{siunitx}
\usepackage[version=4]{mhchem}
\usepackage[acronym]{glossaries}
\setacronymstyle{long-short}
\makeglossaries{}
\newacronym{aod}{AOD}{aerosol optical depth}

% Create some custom commands
\newcommand{\iso}[1][i]{{#1}njected \ce{SO2}}
% The content of the URL must be on its own line. The compiler works fine both ways, but
% the syntax highlighting is messed up by it.
\urldef\fssturl\url{
1850_CAM60%WCCM_CLM50%BGC-CROP_CICE%PRES_DOCN%DOM_MOSART_CISM2%NOEVOLVE_SWAV_TEST
}
%%%%%%%%%%%%%%%%%%%%%%%%%%%%%%

%%%%%%%%%%%%%%%%%%%%%%%%%%%%%%%%

%%% To be entered by author:

%% May use \\ to break lines in title:

\title{
  Parameter Scan: Volcanic influence on climate across multiple magnitudes of injected
  \ce{SO2}
}

%% Enter authors' names and affiliations as you see in the examples below.
%
%% Use \correspondingauthor{} and \thanks{} (\thanks command to be used for affiliations footnotes, 
%% such as current affiliation, additional affiliation, deceased, co-first authors, etc.)
%% immediately following the appropriate author.
%
%% Note that the \correspondingauthor{} command is NECESSARY.
%% The \thanks{} commands are OPTIONAL.
%
%% Enter affiliations within the \affiliation{} field. Use \aff{#} to indicate the affiliation letter at both the
%% affiliation and at each author's name. Use \\ to insert line breaks to place each affiliation on its own line.


%\authors{Author One,\aff{a}\correspondingauthor{Author One, email@email.com} 
%Author Two,\aff{a} 
%Author Three,\aff{b} 
%Author Four,\aff{a} 
%Author Five\thanks{Author Five's current affiliation: NCAR, Boulder, Colorado},\aff{c} 
%Author Six,\aff{c} 
%Author Seven,\aff{d}
% and Author Eight\aff{a,d}
%}
%
%\affiliation{\aff{a}{First Affiliation}\\
%\aff{b}{Second Affiliation}\\
%\aff{c}{Third Affiliation}\\
%\aff{d}{Fourth Affiliation}
%}


\authors{
  Eirik Rolland Enger,\aff{a}\correspondingauthor{Eirik Rolland Enger, eirik.r.enger@uit.no}
  Rune Graversen,\aff{a}
  Martin Rypdal,\aff{a}
  Audun Theodorsen,\aff{a}
  and Maria Rugenstein\aff{b}
}

\affiliation{
  \aff{a}{UiT The Arctic University of Norway, Tromsø, Norway}\\
  \aff{b}{Colorado State University, Fort Collins, Colorado}
}

%%%%%%%%%%%%%%%%%%%%%%%%%%%%%%%%%%%%%%%%%%%%%%%%%%%%%%%%%%%%%%%%%%%%%
% ABSTRACT
%
% Enter your abstract here
% Abstracts should not exceed 250 words in length!
%

\abstract{%
  Large to super-volcano sized eruptions are simulated in the \glsentrylong{cesm2}
  (\glsentryshort{cesm2}) climate model with the \glsentrylong{waccm}
  (\glsentryshort{waccm}) atmosphere. Ensembles containing four members are simulated
  and the \glsentrylong{aod} (\glsentryshort{aod}) and \glsentrylong{rf}
  (\glsentryshort{rf}) from the eruptions are compared to determine how far the linear
  relationship between the two parameters hold. Simulating a Pinatubo-like eruption
  yields results consistent with several previous studies showing a slope of \(\sim
  \SI{-20}{\watt\metre^{-2}}\) per unit \glsentryshort{aod} for \glsentryshort{rf}.
  Larger eruptions, however, show a more shallow slope indicative of a lower cooling
  efficiency. Moreover, a time-after-eruption dependence between \glsentryshort{aod} and
  \glsentryshort{rf} is found, where the \glsentryshort{rf} is relatively stronger than
  \glsentryshort{aod} the first year after the eruption (higher cooling efficiency),
  while later their ratio is roughly constant for a given eruption strength.
}


\begin{document}

%% Necessary!
\maketitle

%%%%%%%%%%%%%%%%%%%%%%%%%%%%%%%%%%%%%%%%%%%%%%%%%%%%%%%%%%%%%%%%%%%%%
% SIGNIFICANCE STATEMENT/CAPSULE SUMMARY
%%%%%%%%%%%%%%%%%%%%%%%%%%%%%%%%%%%%%%%%%%%%%%%%%%%%%%%%%%%%%%%%%%%%%
%
% If you are including an optional significance statement for a journal article or a required capsule summary for BAMS 
% (see www.ametsoc.org/ams/index.cfm/publications/authors/journal-and-bams-authors/formatting-and-manuscript-components for details), 
% please apply the necessary command as shown below:
%
% Significance Statement (all journals except BAMS)
%
%\statement
%	 Enter significance statement here, no more than 120 words. See \url{www.ametsoc.org/index.cfm/ams/publications/author-information/significance-statements/} for details.
%
%% Capsule (BAMS only)
%%
%\capsule
%       Enter BAMS capsule here, no more than 30 words. See \url{www.ametsoc.org/index.cfm/ams/publications/author-information/formatting-and-manuscript-components/#capsule} for details.
%
%% * * If using twocol mode, you will need to use the commands "twocolsig" and "twocolcapsule" in place of "sig" and "capsule"
%%      to ensure that the text box correctly spans across both columns.
%

%%%%%%%%%%%%%%%%%%%%%%%%%%%%%%%%%%%%%%%%%%%%%%%%%%%%%%%%%%%%%%%%%%%%%
% MAIN BODY OF PAPER
%%%%%%%%%%%%%%%%%%%%%%%%%%%%%%%%%%%%%%%%%%%%%%%%%%%%%%%%%%%%%%%%%%%%%
%

%% In all cases, if there is only one entry of this type within
%% the higher level heading, use the star form: 
%%
% \section{Section title}
% \subsection*{subsection}
% text...
% \section{Section title}

%vs

% \section{Section title}
% \subsection{subsection one}
% text...
% \subsection{subsection two}
% \section{Section title}

%%%
% \section{First primary heading}

% \subsection{First secondary heading}

% \subsubsection{First tertiary heading}

% \paragraph{First quaternary heading}

%%%%%%%%%%%%%%%%%%%%%%%%%%%%%%%%%%%%%%%%%%%%%%%%%%%%%%%%%%%%%%%%%%%%%
% TABLES---INSERT NEAR IN-TEXT DISCUSSION
%%%%%%%%%%%%%%%%%%%%%%%%%%%%%%%%%%%%%%%%%%%%%%%%%%%%%%%%%%%%%%%%%%%%%
%%  Enter tables near where they are discussed within the document. 
%%  Please place tables before/after paragraphs, not within a paragraph.
%%
%
%\begin{table}[t]
%\caption{This is a sample table caption and table layout.  Enter as many tables as
%  necessary at the end of your manuscript. Table from Lorenz (1963).}\label{t1}
%\begin{center}
%\begin{tabular}{ccccrrcrc}
%\hline\hline
%$N$ & $X$ & $Y$ & $Z$\\
%\hline
% 0000 & 0000 & 0010 & 0000 \\
% 0005 & 0004 & 0012 & 0000 \\
% 0010 & 0009 & 0020 & 0000 \\
% 0015 & 0016 & 0036 & 0002 \\
% 0020 & 0030 & 0066 & 0007 \\
% 0025 & 0054 & 0115 & 0024 \\
%\hline
%\end{tabular}
%\end{center}
%\end{table}

%%%%%%%%%%%%%%%%%%%%%%%%%%%%%%%%%%%%%%%%%%%%%%%%%%%%%%%%%%%%%%%%%%%%%
% FIGURES---INSERT NEAR IN-TEXT DISCUSSION
%%%%%%%%%%%%%%%%%%%%%%%%%%%%%%%%%%%%%%%%%%%%%%%%%%%%%%%%%%%%%%%%%%%%%
%%  Enter figures near where they are discussed within the document.
%%  Please place figures before/after paragraphs, not within a paragraph.
% %
%
%\begin{figure}[t]
%  \noindent\includegraphics[width=19pc,angle=0]{figure01.pdf}\\
%  \caption{Enter the caption for your figure here.  Repeat as
%  necessary for each of your figures. Figure from \protect\cite{Knutti2008}.}\label{f1}
%\end{figure}

% NOTE: what to include in the paper, key questions.
% The paper should provide insight about what might happen if a large volcano erupted
% (order of magnitude or more than Mt.\ Pinatubo). How does the atmosphere react, for
% example in the aerosol dynamics? (QBO, SO2/AOD/RF relationship.) It should also be
% about how volcanic simulations compare in magnitude and if there is time for more
% simulations, how model complexity (dynamic ocean against slab ocean) affect things.
% - How far does the linear relation between AOD and RF go? What phases does the
%   aerosols go though? (Perhaps the most promising avenue.)
% - How much does it matter how high in the atmosphere the initial SO2 is injected?
%   (Already is some literature on this, suggesting it is not much. Also some on
%   latitude dependence, which has a bigger influence.)
% - How does the climate response change based on the state of the climate: what if we
%   run a CO2 doubling or quadrupling simulation until close to equilibrium, and let the
%   volcanoes erupt then? (Lack the doubling scenario, and setting it up has resulted in
%   strange output that must be resolved. Could take a while.)

\section{Introduction}

% NOTE: Suggested layout for the introduction
% - The objectives of the work.
% - The justification for these objectives: Why is the work important?
% - Background: Who else has done what? How? What have we done previously?
% - Guidance to the reader: What should the reader watch for in the paper? What are the
%   interesting high points? What strategy did we use?
% - Summary/conclusion: What should the reader expect as conclusion? In advanced
%   versions of the outline, you should also include all the sections that will go in
%   the Experimental section (at the level of paragraph subheadings) and indicate what
%   information will go.

% Toohey et al 2011 have a nice end of introduction.

% \paragraph*{Historical background: Volcanoes in climate science}

During much of the past few thousand years, the Holocene, the natural climate
variability on Earth has been mostly forced by volcanic eruptions \citep{sigl2022}.
Despite the strong climate variability impact of volcanic eruptions, in particular up to
the current anthropogenic forcing of the climate, few climate model experiments have
included volcanic forcing when simulating climate evolution during the Holocene
\citep{sigl2022}, implying an exaggerated positive forcing \citep{gregory2016}. Thus,
even though the understanding of how volcanic eruptions force the climate has been given
much attention, questions for example related to aerosol particle processes such as how
they grow, which impacts their scattering efficiency and possibly the \gls{aod} to
\gls{rf} relationship is still unanswered
\citep[e.g.][]{robock2000,zanchettin2019,marshall2020,marshall2022}.

% NOTE: Section on similarities between CO2 and volcanic forcing. (feedback params
% references). Richardson is doing something that looks very similar to Gunther 2022.
% Mention that as well? (Similarity bw 0.5CO2 and volcanoes.)

% \paragraph*{Historical background: Can volcanoes aid \ce{CO2} in constraining \gls{ecs}?}

% NOTE: take-away should be how RF is estimated and that it is sometimes defined as ERF
% and IRF.

One avenue that has been given much attention is how forcing from volcanoes or
volcano-like forcings compares to the forcing that is due to increased \ce{CO2} levels.
\citet{boer2007,marvel2016,merlis2014,ollila2016,richardson2019,salvi2022,wigley2005}
all investigate the link, or if any exists, between volcanic forcing and the climate
sensitivity to a doubling of \ce{CO2}. This comparison is motivated by the large
uncertainty in estimates of the sensitivity of the real climate system, and inferring
climate sensitivity from for example volcanoes have been used as an attempt to constrain
the sensitivity \citep{boer2007}. Such a constraint rely on the assumption that volcanic
forcing and \ce{CO2} forcing produce correlated feedbacks \citep{pauling2023}. Earlier
studies suggest that it might be possible to constrain \gls{ecs} with volcanoes
\citep{bender2010}, as long as \gls{ecs} is constrained by \gls{erf} and not \gls{irf},
which differ in that \gls{erf} account for rapid atmospheric adjustments while \gls{irf}
do not \citep{richardson2019}. Other studies suggest either that constraining \gls{ecs}
by \gls{erf} is not possible and that the sensitivity to volcanic forcing and \ce{CO2}
doubling are different \citep{douglass2006}, or that constraining the \gls{ecs} by
\gls{erf} will be inaccurate since the precision at which climate simulations are run is
not high enough \citep{boer2007,salvi2022}. Even though \gls{erf} is a more suited
indicator of the temperature response to a forcing than \gls{irf}
\citep{marvel2016,richardson2019}, more recent studies conclude that \gls{ecs} cannot be
constrained from volcanoes \citep{pauling2023}.

% \paragraph*{Theoretical background: Aerosol evolution and atm.\ dynamics}

\ce{H2O}, \ce{N2} and \ce{CO2} are the most abundant gases emitted by volcanoes
\citep{robock2000}, but still, sulphur species like \ce{SO2} are the most influential
due to the already relatively high concentrations of the former gases in the atmosphere.
From \ce{SO2} \ce{H2SO4} is formed after reactions with \ce{OH} and \ce{H2O}
\citep{robock2000}. Sulphate acid, \ce{H2SO4}, produces the largest effect by
backscattering sunlight and thus increasing the planetary albedo and reducing the
\gls{rf} from the Sun, and since its formation from \ce{SO2} is happening over weeks
\citep{robock2000}, the peak \gls{rf} from the eruption has a slight delay from the
onset of the eruption. The duration of time that the \ce{H2SO4} aerosols stay in the
atmosphere depends on several factors, including the phase of the \gls{qbo}
\citep{pitari2016b}, the season of the year (determining to which hemisphere aerosols
are transported) \citep{toohey2011,toohey2019}, latitude \citep{marshall2019,
  toohey2019}, volcanic plume height \citep{marshall2019} and the aerosol size
\citep{marshall2019}. In the case of tropical eruptions, aerosols are typically
transported poleward in the stratosphere and back to mid-latitude troposphere within one
to two years \citep{robock2000}. As the aerosols move below the tropopause they are
removed by wet deposition before retaining their background state \citep{liu2012}.

% \paragraph*{Similar contributions: Comparing \gls{rf} and \gls{aod}}

Previous studies on both Mt.\ Pinatubo alone \citep{mills2017,hansen2005} and volcanoes
in the instrumental era \citep{gregory2016} have been used to estimate the relationship
between the \gls{rf} energy imbalance caused by volcanic eruptions and \gls{aod}.
\citet{myhre2013} use a formula for \gls{rf} that scales \gls{aod} by
\(\SI{-25}{\watt\metre^{-2}\mathrm{AOD}^{-1}}\) while estimates down to
\(\SI{-19.0(5)}{\watt\metre^{-2}\mathrm{AOD}^{-1}}\) \citep{gregory2016} and
\(\SI{-18.3(10)}{\watt\metre^{-2}\mathrm{AOD}^{-1}}\) \citep{mills2017} have been
reported more recently. Simulations done with synthetic volcanoes have also been used,
and in \citet{marshall2020} they obtain a scaling factor of
\(\SI{-20.5(2)}{\watt\metre^{-2}\mathrm{AOD}^{-1}}\) from \(82\) simulations of volcanic
eruptions with varying injection height and latitude, ranging from \(10\) to
\(\SI{100}{\tera\gram}\) of \iso{}.

A similar simulation setup, albeit with notable differences, was done by
\citet{niemeier2015}, where \(14\) different levels of injected sulphur ranging between
\(\SI{1}{\tera\gram(\ce{S})\mathrm{yr}^{-1}}\)
(\(\SI{2}{\tera\gram(\ce{SO2})\mathrm{yr}^{-1}}\)) and
\(\SI{100}{\tera\gram(\ce{S})\mathrm{yr}^{-1}}\)
(\(\SI{200}{\tera\gram(\ce{SO2})\mathrm{yr}^{-1}}\)) was simulated. These were
geoengineering simulations where sulphur would be continually injected, and the
simulations run until the sulphur level was in a steady state. From this, they found
that the \gls{rf} to injected \ce{SO2} rate ratio would follow an exponential and
converge to a value of \(\SI{-65}{\watt\metre^{-2}}\). While the results from the
super-volcano simulation by \citet{jones2005} indicate a weaker gradient of \(\sim
\SI{-4}{\watt\metre^{-2}\mathrm{AOD}^{-1}}\) than the
\(-18.3\)--\(\SI{-25}{\watt\metre^{-2}\mathrm{AOD}^{-1}}\) found in more recent
literature, the peak \gls{rf} obtained in the super-volcano simulation was
\(\SI{-60}{\watt\metre^{-2}}\), suggesting the \citet{niemeier2015} results might be
transferable to a volcanic eruption simulation scenario. Moreover, \citet{timmreck2010}
find a peak \gls{rf} anomaly of \(\SI{-18}{\watt\metre^{-2}}\) from
\(\SI{1700}{\tera\gram}\) of \iso{}, which corresponds well with the function estimated
by \citet{niemeier2015}.

% \paragraph*{Similar contributions: Time-after-eruption dependence}

The aerosol evolution and lifetime in the stratosphere influence the \gls{aod} and the
\gls{rf}, but not always in the same way. \citet{marshall2020} present results showing
higher efficiencies in years 2 and 3 after an eruption, compared to year 1, suggesting
that this is due to the aerosols being concentrated more spatially in the first year and
more spread out in later years. This has the effect of increasing the albedo per global
mean \gls{aod} which in turn increases the \gls{rf} to \gls{aod} ratio
\citep{marshall2020}.
% TODO: Also, \gls{aod} to \gls{rf} depend on insulation, cloudiness and surface albedo
% among others, which vary across eruptions \citep{marshall2021,andersson2015}.
% Andersson note that including the lowermost stratosphere is in some latitudes part in
% increaseing the AOD by 45%, which means to include AOD estimates between the
% tropopause and 380K potential temperature level.
% gettleman2019 refer to AOD at being above the tropopause, so my initial guess is the
% CESM2 do consider the full vertical profile that Andersson calls for.

% \paragraph*{This paper's contributions}

This work aims to evaluate the relationship between \gls{aod} and \gls{rf} for volcanic
eruptions of super-volcano size. This is considering the linear relationship often found
between \gls{aod} and \gls{rf} of approximately
\(\SI{20}{\watt\metre^{-2}\mathrm{AOD}^{-1}}\) \citep{gregory2016, marshall2020,
  mills2017, myhre2013} along with studies indicating a non-linear relationship
\citep{niemeier2015}, as well as a time-after-eruption dependence \citep{marshall2020}.

To address these issues we simulate ensembles of volcanic eruptions in the
\glsentrylong{cesm2}, with \iso{} spanning three orders of magnitude of
\(\SI{26}{\tera\gram(\ce{SO2})}\), \(\SI{400}{\tera\gram(\ce{SO2})}\) and
\(\SI{1629}{\tera\gram(\ce{SO2})}\), with further details on the experiment setup in
\ref{sec:method}. We find that for large to super-volcano sized volcanic eruptions there
is both a clear non-linear \gls{rf} to \gls{aod} dependence, but also a
time-after-eruption dependence on both \gls{aod} and \gls{rf}, which is presented in
\ref{sec:results} and discussed in \ref{sec:discussion}. Finally, in
\ref{sec:conclusions}, conclusions are made.

\section{Method}\label{sec:method}

\subsection{Model}

\begin{table*}
  \centering

  \caption{\glsentrylong{cesm2} model components}\label{tab:cesm-components}%
  \begin{center}
    \begin{tabular}[c]{ll}
      \multicolumn{1}{c}{\textbf{Component name}} & 
      \multicolumn{1}{c}{\textbf{Reference}}                                              \\
      \glsentrylong{cesm2}                        & \citet{danabasoglu2020}               \\
      \glsentrylong{waccm}                        & \citet{gettleman2019}                 \\
      \glsentrylong{pop}                          & \citet{smith2010, danabasoglu2020}    \\
      \glsentrylong{mosart}                       & \citet{li2013, danabasoglu2020}       \\
      \glsentrylong{clm}                          & \citet{lawrence2019, danabasoglu2020} \\
      \glsentrylong{ww3}                          & \citet{danabasoglu2020}               \\
      \glsentrylong{cice}                         & \citet{danabasoglu2020}               \\
      \glsentrylong{cism}                         & \citet{danabasoglu2020}               \\
      \glsentrylong{cime}                         & \citet{danabasoglu2020}\\
    \end{tabular}
  \end{center}
\end{table*}

We apply the \gls{cesm2} \citep{danabasoglu2020}, along with the \gls{waccm}
\citep{gettleman2019} and the fully dynamical ocean component \gls{pop}
\citep{smith2010, danabasoglu2020}. A complete list of model components is found in
\ref{tab:cesm-components}. The atmosphere model was run at nominal \(\SI{2}{\degree}\)
resolution, with \(70\) vertical levels, in the \gls{ma} configuration.

The \gls{ma} version of \gls{waccm} uses the \gls{mam3} \citep{gettleman2019}. This is
implemented as a simplified and computationally efficient default setting in the
\gls{cam5} \citep{liu2016}, and is described in \citet{liu2012}. The \gls{mam3} was
developed from MAM7 (seven modes) by merging the primary carbon mode with the
accumulation mode, and assuming instantaneous internal mixing of primary carbonaceous
aerosols with secondary aerosols \citep{liu2016}. Specifically, the three modes are
Aitken, accumulation and coarse (MAM7 includes Aitken, accumulation, primary carbon,
fine dust and fine sea salt, coarse dust and coarse sea salt modes) \citep{liu2016}.
Instantaneous ageing of primary carbonaceous particles is assumed by emitting them in
the accumulation mode. The \gls{mam3} include \(15\) transported aerosol tracers
\citep{liu2016}.

Dust absorbs water efficiently and is thus expected to be removed by wet deposition
similarly to sea salt, which makes the course mode (of which sea salt and soil dust are
both part of) quickly retain its background state below the tropopause \citep{liu2012}.
Likewise, fine dust and fine sea salt are both merged into the accumulation mode.

\subsection{Simulation set up}

Simulations were created using a modified version of the file
\url{http://svn.code.sf.net/p/codescripts/code/trunk/ncl/emission/createVolcEruptV3.ncl},
via a Python project developed on GitHub at
\url{https://github.com/engeir/volcano-cooking}. The project is also available from the
Python package manager PyPI\@. The program creates volcanoes with a given \ce{SO2}
amount that is injected over six
hours\footnote{\url{http://svn.code.sf.net/p/codescripts/code/trunk/ncl/emission/createVolcEruptV3.ncl}}
at a given altitude, latitude and longitude. All volcanic \ce{SO2} files are created by
setting the eruption details in a~.json file that is read to the
\texttt{volcano-cooking} CLI at a fixed version, making for a reproducible experiment
setup.

We are using the \texttt{BWma1850} component
setup\footnote{\url{https://docs.cesm.ucar.edu/models/cesm2/config/2.1.0/compsets.html}}
to run the \gls{cesm2}, and an accompanying \gls{fsst}
% WARN: how should the fSST compset be referred to
simulation to obtain estimates of the \gls{rf}. The \gls{fsst} simulation used is not
from a standardised component setup as of \gls{cesm2} (v2.1.3), but is instead specified
in full.\footnote{\fssturl} They differ in \texttt{CICE -> CICE\%PRES}, which is
prescribed sea-ice, \texttt{POP2\%ECO\%DEP -> DOCN\%DOM} which is from a dynamical ocean
to a prescribed data ocean and the wave component \texttt{WW3 -> SWAV} which is now a
stub wave component instead of the full \gls{ww3}.

\subsection{Output variables}

\gls{rf} is calculated as the combined (\gls{sw} and \gls{lw}) all-sky \gls{toa} energy
imbalance, where the \gls{cesm2} provide the output variables \gls{fsnt} and \gls{flnt}.
Thus, \(\mathrm{RF_*}= \mathrm{FSNT} - \mathrm{FLNT}\), and taking the difference
between volcanic simulations and a control simulation gives the final estimate of
\gls{rf} (\(\mathrm{RF}=\mathrm{RF_{VOLC}}-\mathrm{RF_{CONTROL}}\))
\citep{marshall2020}. This outline specifically describe how to calculate \gls{erf} as
opposed to \gls{irf}, which instead is the difference between the \gls{erf} and the sum
of all rapid atmospheric adjustments \citep{marshall2020,smith2018}. The \gls{aod} is
obtained from the output variable \gls{aodm}, while global temperature is saved by
\gls{cesm2} to the variable \gls{trefht}. These four output variables are all that are
used throughout this paper. The important input data used in the model simulations are
\iso{} in units of teragrams, used to simulate volcanic eruptions.

\subsection{Simulations}

The simulations are summarized in \ref{tab:simulation-overview} and cover three \ce{SO2}
injection magnitudes and four seasons; 15 February, 15 May, 15 August and 15 November.
The magnitudes vary across three orders of magnitude: \(\SI{26}{\tera\gram}\),
\(\SI{400}{\tera\gram}\) and \(\SI{1629}{\tera\gram}\). The smallest eruption case
(\gls{c2wm}) is of the same order of magnitude as Mt.\ Pinatubo
\citep[\(\sim10\)--\(\SI{20}{\tera\gram}\);~e.g.][]{timmreck2018} and Mt.\ Tambora
\citep[\(\sim\SI{56.2}{\tera\gram}\);~e.g.][]{zanchettin2016}, the intermediate case
(\gls{c2wmp}) is similar to the 1257 Samalas eruption
\citep[\(\sim{118.8}\)--\(\SI{173.1}{\tera\gram}\);~e.g.][]{toohey2017,ottobliesner2016},
and the largest case (\gls{c2ws}) is similar to the \gls{ytt} eruption
\citep[\(100\)--\(\SI{10000}{\tera\gram}\);~e.g.][]{jones2005}. They are all located at
the equator, at \(\SI{0}{\degree N}\), \(\SI{1}{\degree E}\) and \ce{SO2} is injected
between \(\SI{18}{\kilo\meter}\) and \(\SI{20}{\kilo\meter}\). An ensemble of two
additional eruptions at high latitude were also simulated, separated by six months (15
February and 15 August) (\gls{c2wsn}). They are of the same magnitude in \iso{} as
\gls{c2ws}, but located at \(\SI{56}{\degree \mathrm{N}}\). An advantage with having
eruptions this big, in the large to super-volcano size, is improvement of the
signal-to-noise ratio without having to run a large and computationally expensive
ensemble. However, this still leave the question of whether this will give realistic
values for the forcing open \citep{gregory2016}.

\begin{table}
  \centering

  \caption{Simulations done with the \gls{cesm2}. \glsentrylong{c2wsn} and
    \glsentrylong{c2ws} are the same in eruption magnitude, but while \gls{c2ws} is located
    at the equator, \gls{c2wsn} is located at high latitude. \glsentrylong{c2wmp} and
    \glsentrylong{c2wm} are located at the equator, but with different magnitudes to
    \gls{c2ws}}\label{tab:simulation-overview}%
  \begin{center}
    \begin{tabular}[c]{ccc}
      \textbf{Name} & \textbf{\ce{SO2}}         & \textbf{Lon, lat, alt}                  \\
      \gls{c2wsn}   & \(\SI{1629}{\tera\gram}\) & 
      \SI{1}{\degree\mathrm{E}}, \SI{56}{\degree\mathrm{N}}, \(18\)--\SI{20}{\kilo\metre} \\
      \gls{c2ws}    & \(\SI{1629}{\tera\gram}\) & 
      \SI{1}{\degree\mathrm{E}}, \SI{0}{\degree\mathrm{N}}, \(18\)--\SI{20}{\kilo\metre}  \\
      \gls{c2wmp}   & \(\SI{400}{\tera\gram}\)  & 
      \SI{1}{\degree\mathrm{E}}, \SI{0}{\degree\mathrm{N}}, \(18\)--\SI{20}{\kilo\metre}  \\
      \gls{c2wm}    & \(\SI{26}{\tera\gram}\)   & 
      \SI{1}{\degree\mathrm{E}}, \SI{0}{\degree\mathrm{N}}, \(18\)--\SI{20}{\kilo\metre}  \\
    \end{tabular}
  \end{center}
\end{table}

\section{Results}\label{sec:results}

% NOTE: the results should be laid out in a logical way, with the most
% interesting/important stuff first, then tangents that dig deeper at specific things
% later.
% 1. RF to AOD time-after-eruption dependence should be top priority (8 figs atm.)
% 2. Then probably temperature scaling since we discuss the shape of both AOD and RF
%    time series before that (MOTIVATION: can we expect a specific temperature time
%    series shape based on the shape of either of or both of the RF and AOD time
%    series?)
% 3. If there is something interesting to say about the rest of the figures (all the
%    comparing of parameters), then this should come here.

\subsection{Part 1}

We start by looking at the time series of the global mean surface air temperature.
Medians with the 5th to 95th percentiles are shown in \ref{fig:compare-waveform}, where
the black lines are the medians across the four member ensembles, while shading indicate
the percentiles. Three different forcing magnitudes have been used, as outlined in
\ref{tab:simulation-overview}. The figures below show two alternative ways of
normalizing the temperature response to these volcanic forcing events. First, the three
temperature responses are normalised by setting the peak value equal to unity, shown in
\ref{fig:compare-waveform}a. The second method of normalizing, shown in
\ref{fig:compare-waveform}b, is to divide by the integral of the time series,
such that the scaled time series integrate to unity.

When normalizing by setting their amplitude to unity, the initial rise across all three
time series is comparable. The difference among the three is how quickly the temperature
reverts back to equilibrium in the small eruption case (solid black and blue shading in
\ref{fig:compare-waveform}a). However, when normalizing by enforcing all time series
to integrate to the same value, the tails of the temperature time series by construction
become similar across the three eruption magnitudes. From both of these two
normalizations, the amount of time spent at the peak temperature appears different. The
temperature from the smaller eruption simulation starts to revert sooner after it
reaches the peak temperature. The shape of the initial rise and the tail is more similar
across the three, while the peak is sharper in the small eruption case compared to the
two larger.

Despite some differences, the shape of the temperature time series is similar across
forcing magnitudes. No non-linear effects appear to have a strong impact even in the
super-volcano case. Between the two strongest eruption cases where noise play a much
smaller role, shown in short stippled and long stippled lines with orange and green
shading, the temperature time series are indistinguishable from one another. As such, it
is expected that similar dynamics are at play in all cases.

% NOTE: Any similar results that is consistent with a faster temperature recovery for
% weaker volcanic forcing? Not, sure, but one curiosity is that in the AOD signals, the
% smaller eruption spent more time at the peak, contrary to what seems to be the case
% for the temperature. The RF time series have similar shape for all forcing magnitudes.
% See figs. 9 (aod_arrays_normalised) and 10 (toa_arrays_normalised).

\begin{figure}
  \centering
  \includegraphics[width=0.95\linewidth]{figures/compare-waveform.png}

  \caption{Temperature response to the three tropical volcanic eruptions cases,
    \gls{c2wm}, \gls{c2wmp} and \gls{c2ws}. The temperature time series have been normalised
    to have (a) peak values at unity and
    (b) so that they integrate to one, where \(C\) is the
    normalisation constant. Black lines indicate the median across the four member
    ensembles, while shading mark the 5th and 95th percentiles.}\label{fig:compare-waveform}%
\end{figure}

We next look at whether the shape of the temperature time series can be inferred from
the shape of either of the forcing time series or not (\gls{aod} or \gls{rf}). When
plotting the \gls{rf} time series, we find that their shapes are consistent over
different eruption strengths (\ref{fig:arrays_normalised}b), and as such, that the
temperature seem to have a strong dependence on \gls{rf}. The same can largely be said
about the \gls{aod} as well, but the \gls{aod} time series do show a slight change in
shape from smaller to larger eruptions (\ref{fig:arrays_normalised}a). Specifically,
the \gls{aod} time series from the smaller eruptions have a fast rise and a flat peak
before it decays back to its equilibrium state, while from the larger eruptions we find
a slower rise time, but a sharper peak, making the decay to equilibrium happening at a
similar time after the eruption and at a similar rate.
% TODO: Why? Marshall et al. (2019) and references therein may be the best source for a
% suggested explanation.

\begin{figure}
  \centering
  \includegraphics[width=0.95\linewidth]{figures/arrays_combined_normalized.png}

  \caption{The \gls{aod} time series (a) and \gls{rf} time
    series (b) from the simulations summarised in
    \ref{tab:simulation-overview}, scaled to have peak values at unity
  }\label{fig:arrays_normalised}
\end{figure}

\subsection{Part 2}

We next focus our attention towards how the \gls{aod} and \gls{rf} time series develop
relative to each other, which are both parameters providing insight about how volcanic
eruptions impact climate. A similar comparison was carried out in \citet[][their Fig.\
  4]{gregory2016} and \citet[][their Fig.\ 1]{marshall2020}, with \gls{rf} to \gls{aod}
plots. \ref{fig:aod_vs_toa_ses_avg} show annual mean values from the four simulation
cases in \ref{tab:simulation-overview}; the small eruption case (\gls{c2wm}) as blue
downward pointing triangles, the intermediate eruption case (\gls{c2wmp}) as orange
thick diamonds, the large tropical eruption case (\gls{c2ws}) as green upward pointing
triangles, the large northern hemisphere eruption case (\gls{c2wsn}) as brown upward
pointing three-branched twigs, and the data from \citet[][Fig.\ 4, black crosses from
  HadCM3 sstPiHistVol]{gregory2016} as grey crosses labelled G16. Additionally, the
estimated peak values from the Mt.\ Pinatubo eruption and Mt.\ Tambora eruption are
plotted as a purple star and a yellow plus, while the peak from the \gls{p100} is shown
as a pink square. Finally, red circles represent the peak values obtained from the three
tropical eruptions (\gls{c2wm}, \gls{c2wmp} and \gls{c2ws}). The gradient lines are the
same as shown by \citet{gregory2016}. The full data range is shown in
\ref{fig:aod_vs_toa_ses_avg}a while \ref{fig:aod_vs_toa_ses_avg}b show a narrow range
highlighting the \gls{c2wm} case.

The annual mean data from the Pinatubo-like \gls{c2wm} case shown in
\ref{fig:aod_vs_toa_ses_avg}b as blue downward facing triangles, has \gls{rf} values as a
function of \gls{aod} that follow almost the same constant gradient as the
\citet{gregory2016} data. However, in \ref{fig:aod_vs_toa_ses_avg}a we find that the
stronger eruptions lead to dissimilar responses in \gls{aod} and \gls{rf}, where the
slope of the \gls{c2wmp} case seem to follow close to a \(-10\) gradient and the
\gls{c2ws} case is closer to a \(-4\) gradient. The peak values (red circles) suggest a
non-linear functional shape dependence, while within each eruption strength the annual
mean values fall relatively close to a straight line.

\begin{figure}
  \centering
  \includegraphics[width=0.95\linewidth]{figures/aod_vs_toa_avg.png}

  \caption{\gls{rf} as a function of \gls{aod}, yearly means. Data from the four
    simulations listed in \ref{tab:simulation-overview} (\gls{c2wm}, \gls{c2wmp}, \gls{c2ws}
    and \gls{c2wsn}) are shown along with the data from the HadCM3 sstPiHistVol simulation
    by \citet{gregory2016} (grey crosses). Also shown are the estimated peak values of the
    Mt.\ Pinatubo (purple star) and Mt.\ Tambora (yellow plus) eruptions. In
    (a) the simulated super-volcano of \citet{jones2005} (pink
    square) is shown as well as the peak values from the simulations \gls{c2wm}, \gls{c2wmp}
    and \gls{c2ws} as red circles. All peak values (as opposed to annual means) have an
    asterisk (\(\ast{}\)) in their label. The grey gradient lines are the same regression
    fits as in \citet[][Fig.\ 4]{gregory2016}, where the solid line is the fit to the
    Gregory et al.\ data (grey crosses). (b): Zooming in on the
    smallest values}\label{fig:aod_vs_toa_ses_avg}%
\end{figure}

\subsection{Part 3}

To get a feeling for how the responses in \gls{aod} and \gls{rf} develop relative to
each other over time, as briefly mentioned in relation to \ref{fig:arrays_normalised},
we plot in \ref{fig:aod_vs_toa_avg_loop_ratios} seasonal means of the \gls{rf} to
\gls{aod} ratio, where the start of the time series is taken as that of the eruption
day. \ref{fig:aod_vs_toa_ses_avg} show the
\gls{rf} to \gls{aod} ratios from \ref{fig:aod_vs_toa_ses_avg}, but
where the data is averaged to seasonal means rather than yearly means. In
\ref{fig:aod_vs_toa_avg_loop_ratios}a, straight lines are linear regression fits to the
seasonal means across all four ensembles, summarised in \ref{tab:slope-gradients}.
Shaded regions are the standard deviation around the seasonal means. A similar shading
is plotted in \ref{fig:aod_vs_toa_avg_loop_ratios}b, but where the regression fits
have been omitted to avoid further cluttering the plot. The years where the
signal-to-noise ratio is the lowest are years \(1\) and \(2\), as well as year \(0\)
(the noise is mostly due to the \gls{rf} time series, shown in
\ref{fig:arrays_normalised}b). Hence, the ratio of \gls{rf} to \gls{aod} is
calculated for the second season of the first year until the end of the third year.

Even though the ratio changes between the eruption magnitudes, we find that the gradient
at which the ratio is changing is similar across large eruption magnitudes. A slope of
approximately \(5.8\) during the first period is a good fit for both the \gls{c2wmp}
case (orange thick diamonds) and the \gls{c2ws} case (green upward triangles)
(\ref{tab:slope-gradients}, ``1st period'' and ``\ref{fig:aod_vs_toa_avg_loop_ratios}a''
columns), and even though the spread in the \gls{c2wm} case (blue downward triangles) is
large in the \(y\)-direction, they tend to follow a similarly inclined, albeit steeper,
slope. Normalizing the time series before computing the ratio, shown in
\ref{fig:aod_vs_toa_avg_loop_ratios}b yield a similarly inclined slope across all
forcing magnitude ensembles. In the 2nd period, from the second season of the second
year, the ratios stay close to constant for the reminder of the decaying phase of the
time series. Again this become more clear in the normalised version in
\ref{fig:aod_vs_toa_avg_loop_ratios}b, where the ratios are similar between the
ensembles.

\begin{table}
  \centering

  \caption{Gradient and standard deviation for the regression lines to the data found in
    \ref{fig:aod_vs_toa_avg_loop_ratios}. The regression
    fit in the top row of all individual simulations indicate the regression fit to the
    first period (from \(0\) years to \(1\) years), while the bottom is the regression fit
    to the second period (from \(1\) years to \(3\) years)}\label{tab:slope-gradients}%
  \begin{tabular}{cccc}
    Simulation                   & Period & \ref{fig:aod_vs_toa_avg_loop_ratios}a & 
    \ref{fig:aod_vs_toa_avg_loop_ratios}b                                                     \\
    \multirow{2}{*}{\gls{c2wsn}} & 1st    & \(3.04\pm1.28\)                     & \(0.67\pm0.28\)  \\
                                 & 2nd    & \(0.65\pm0.44\)                     & \(0.14\pm0.10\)  \\
    \multirow{2}{*}{\gls{c2ws}}  & 1st    & \(5.78\pm0.59\)                     & \(1.13\pm0.11\)  \\
                                 & 2nd    & \(-0.83\pm0.36\)                    & \(-0.16\pm0.07\) \\
    \multirow{2}{*}{\gls{c2wmp}} & 1st    & \(5.84\pm0.93\)                     & \(0.55\pm0.09\)  \\
                                 & 2nd    & \(-0.72\pm0.43\)                    & \(-0.07\pm0.04\) \\
    \multirow{2}{*}{\gls{c2wm}}  & 1st    & \(10.23\pm3.90\)                    & \(0.43\pm0.16\)  \\
                                 & 2nd    & \(1.03\pm2.03\)                     & \(0.04\pm0.08\)  \\
  \end{tabular}
\end{table}

\citet[][their Fig.\ 1c,d]{marshall2020} present results showing a time dependence in
the conversion between \gls{aod} and \gls{rf}, but where \gls{rf} is larger later in the
eruption evolution when compared to \gls{aod} (higher efficiency), not smaller. This is
understood by \citet{marshall2020} to happen since the aerosols are initially spatially
confined to the hemisphere where the eruption was situated, before they during the
second and third years spread globally, leading to a larger global mean albedo per
\gls{aod} and in turn larger \gls{rf} per \gls{aod}. From their results and the results
shown here in \ref{fig:aod_vs_toa_avg_loop_ratios},
there seem to be several and competing effects that decide on the values of \gls{aod}
and \gls{rf}.

The change in ratio, where the smallest is found in the larger eruptions and the largest
ratio is from the smaller eruptions, is consistent with the change in peak values seen
in \ref{fig:aod_vs_toa_ses_avg}a, where the red circles indicate that the peak values are
saturating in \gls{rf}. The \gls{rf} peak magnitudes increases more slowly than the
\gls{aod} peak magnitudes which increases close to linearly with injected \ce{SO2} (see
also \ref{fig:parameter_scan}a). This may be the result of larger aerosols having time to
develop as the amount of injected \ce{SO2} increases \citep{niemeier2015,marshall2019}.
This in turn make the forcing from smaller eruptions more efficient than from large
eruptions since larger aerosols scatter radiation less efficiently, causing a smaller
ratio between \gls{rf} and \gls{aod}.

\begin{figure}
  \centering
  \includegraphics[width=0.95\linewidth]{figures/aod_vs_toa_loop.png}

  \caption{(a): Ratio of \gls{rf} to \gls{aod}, with
    time-after-eruption on the horizontal axis. Slopes are linear regression fits and are
    described in \ref{tab:slope-gradients}, while shaded regions are the standard deviation
    across the ensembles for each season. (b): Same
    as in (a), but where the underlying \gls{aod} and
    \gls{rf} time series have been scaled to have peak values at
    unity}\label{fig:aod_vs_toa_avg_loop_ratios}%
\end{figure}

\subsection{Part 4}

In \ref{fig:parameter_scan} we compare all relevant parameters to each other. The initial input
parameter to the \gls{cesm2} is injected \ce{SO2}. Compared to \gls{aod} peak values we
get an almost linear relationship against \iso{} for our tropical \gls{c2w} cases, shown
in \ref{fig:parameter_scan}a. The latitude also plays a role for how large the \gls{aod}
perturbation becomes, as we can see from the \gls{c2wsn} data point. The weak dependence
on eruption latitude is also reported in \citet{marshall2019}, where they find
\(\SI{72}{\percent}\) of the \gls{aod} variance can be explained by \iso{}, while
latitude contributes to only \(\SI{16}{\percent}\) of the variance.

With the almost linear relation between injected \ce{SO2} and \gls{aod}, we should
expect to see a similar plot for \iso{} versus \gls{rf} as for \gls{aod} versus
\gls{rf}. \iso[I] versus \gls{rf} is shown in \ref{fig:parameter_scan}b, but where the
absolute value of the \gls{rf} is potted along the \(y\)-axis. Just as in
\ref{fig:aod_vs_toa_ses_avg}a we find the \gls{cesm2} data points to be heavily damped in
\gls{rf} as \iso{} increases. The same effect can be seen in the data from
\citet{ottobliesner2016} (labelled OB16, red points), which follow much the same path as
our \gls{c2w} data. A description of how the OB16 data was analysed can be found in
\ref{app:ob16}. The simulations used by \citet{ottobliesner2016} was using \gls{cesm1}
with a low-top atmosphere (\gls{cam5}), thus we find that despite the difference in
model complexity, similar \glspl{rf} are obtained by \citet{ottobliesner2016} as what we
find.

\begin{figure*}
  \centering
  \includegraphics[width=0.95\linewidth]{figures/parameter_scan.png}

  \caption{(a) \gls{aod} (b) \gls{rf} and (c) temperature as a function of \iso{}\@. (d)
    \gls{rf} and (e) temperature as a function of \gls{aod}. (f) Temperature as a function
    of \gls{rf}. \gls{c2w} cases are shown as blue diamonds, the \gls{c2wsn} case is the
    brown three-branched twig, OB16 are data from \citet{ottobliesner2016} marked as red
    asterisks, the purple star and yellow plus are Mt.\ Pinatubo and Mt.\ Tambora estimates,
    the pink square is the one-hundred times Mt.\ Pinatubo super-volcano from
    \citet{jones2005} and the pink disk is the \gls{ytt} super-volcano from
    \citet{timmreck2010}. A pink stippled line is in (b) used to draw the function from
    \ref{eq:niemeier_exponential}}\label{fig:parameter_scan}%
\end{figure*}

\citet{niemeier2015} did simulations of continually emitting injections of sulphur, up
to \(\SI{100}{\tera\gram \mathrm{(S)}\mathrm{yr}^{-1}}\), in the middle atmosphere
version of the GCM ECHAM5 \citep{giorgetta2006} and with aerosol microphysics from HAM
\citep{stier2005}. They found that the impact of increasing the injection rate lead to
an \gls{rf} as a function of injection rate \(x\) that had an exponential decay
converging to \(\SI{-65}{\watt\meter^{-2}}\),
\begin{equation}
  \Delta
  R_{\mathrm{TOA}} =
  -\SI{65}{\watt\metre^{-2}}
  \mathrm{e}^{-{\left(\frac{\SI{2246}{\tera\gram(S)yr^{-1}}}{x}\right)}^{0.23}}.
  \label{eq:niemeier_exponential}
\end{equation}
%
This is shown in~\ref{fig:parameter_scan} as the stippled pink line to have a much
slower rise than the data from both our simulations and the simulations by
\citet{ottobliesner2016}. However, \citet{timmreck2010} did simulations in the MPI-ESM
that was forced with \gls{aod} obtained from the HAM aerosol model, of a
\(\SI{850}{\tera\gram(\ce{S})}\) eruption representing the \gls{ytt} eruption. That is,
the same aerosol microphysical model was used in \citet{timmreck2010} and
\citet{niemeier2015}, as well as very similar \glspl{esm}; the MIP-ESM is the ancestor
of ECHAM6 which was the next version update from ECHAM5 \citep{kuma2023}. From the
initial input of \(\SI{850}{\tera\gram(\ce{S})}\) (equivalent to
\(\SI{1700}{\tera\gram(\ce{SO2})}\)), via an \gls{aod} estimate, they got a peak
\gls{rf} of only \(\SI{-18}{\watt\metre^{-2}}\) (pink filled circle in
\ref{fig:parameter_scan}b). This is in good agreement with the functional given in
\ref{eq:niemeier_exponential}.
% INFO: the conversion between S and SO2 is confirmed by Niemeier and Timmreck (2015)'s
% reference to the Bekki et al. (1996) paper. Bekki uses 6000 Mt SO2, Niemeier uses 3000
% Tg(S).

\ref{fig:parameter_scan}c show \iso{} versus temperature response. Just as in
\ref{fig:parameter_scan}b, the data from the \gls{cesm2} simulations bend off, making the
temperature response weaker with higher values of \iso. We find here that the data from
the \gls{cesm1} simulations by \citet{ottobliesner2016} follow a different path than the
\gls{c2w} data, with temperature fluctuations being less extreme as \iso{} increases.
\citet{timmreck2010} also find a much weaker temperature response than our results
suggest, with a maximum temperature perturbation of only \(\SI{-3.5}{\kelvin}\) for
their \(\SI{1700}{\tera\gram}\) \ce{SO2} eruption, while the opposite is the case for
the \citet{jones2005} simulation which resulted in a maximum temperature perturbation of
\(\SI{-10.7}{\kelvin}\), much greater than we obtain in our \gls{cesm2} simulations.

In \ref{fig:parameter_scan}d we again look at \gls{rf} as a function of \gls{aod}, but this
time focus on the peak values rather than the annual and seasonal means. As previously
discussed, the gradient is much weaker than what previous studies have found, and in the
case of our \gls{c2w} data the peak values do not follow a straight line. The results
from other studies \citep{jones2005, ottobliesner2016, timmreck2010} do however suggest
there might be large dependencies on the model that was used as well as on the input
parameters of the model, for example related to latitude. These contributions add on top
of noise, which for the relatively small ensemble size of four used here may play a
role.

\ref{fig:parameter_scan}e should in the case of the \gls{cesm2} data points be akin in shape
to the paths found in \ref{fig:parameter_scan}c due to the close to linear relationship
found between \iso{} and \gls{aod} in \ref{fig:parameter_scan}a. The relationship between
\gls{aod} and temperature do tend to follow a similar functional shape, but more data
would be needed to draw strong conclusions from this.

In \ref{fig:parameter_scan}f we compare the \gls{rf} and temperature responses. Both the
\gls{cesm2} data and the \gls{cesm1} data from \citet{ottobliesner2016} show a close to
linear relationship between temperature and \gls{rf}, but where the \gls{cesm2} data has
a steeper gradient and as such seem to lead to stronger temperature perturbations
compared to \gls{cesm1}. However, the overlap in data is not big, and the \gls{cesm2}
data are too sparse to make strong conclusions, in addition to the difference in model
complexity. It is also important to note that the temperature values used in the
\citet{ottobliesner2016} data points are the exact temperature at the time of the
eruption according to the \iso{} time series, which means the peak temperatures will be
biased low, while close eruptions will contribute a bias to higher temperatures. This,
in combination with strong noise make the data less reliable.

\section{Discussion}\label{sec:discussion}

% NOTE: Suggested layout for the
% Discussion:
% - Explain the results and emphasize significant findings clearly
% - Discuss the impact and importance of results compared with recent relevant research
% Conclusion
% - The justification for these objectives: Why is the work important?
% - Summarize the key points made in the other sections
% - Conclude overall discussion of article
% - Link this section to the introduction

\subsection{Linearity between \gls{aod} and \gls{rf}}

Fig.~\ref{fig:aod_vs_toa_ses_avg},~\ref{fig:aod_vs_toa_avg_loop_ratios},~\ref{fig:parameter_scan}d all show that
as the \gls{aod} increases past \(\sim 1.0\), the scaling of \(\sim -20\) no longer map
to corresponding values of \gls{rf}. When looking at the intermediate volcanic eruption
with \(\SI{400}{\tera\gram}\) \iso{} there seem to be a mostly linear relationship
between the \gls{aod} and \gls{rf} values, with a slope of approximately \(-10\). As for
the largest eruption, scaling \gls{aod} by \(\sim-5\) yields reasonable \gls{rf} values.
From \ref{fig:parameter_scan}a we have an almost linear relation between \iso{} and
\gls{aod}, and previous results indicate that larger eruptions where more \ce{SO2} is
injected creates larger aerosols which are less effective at scattering radiation,
making the cooling less efficient \citep{english2013,timmreck2010,timmreck2018}.
\citet{timmreck2010} also find that for very large eruptions, they look at Pinatubo and
\gls{ytt}, \ce{OH} radicals are not abundant enough which limit the \ce{SO2} oxidation.
(Originally in \citep{bekki1995}.) The peak in \gls{aod} levels for the \gls{ytt}
simulation appear six months after the peak of Pinatubo in \citet{timmreck2010}, and at
a much smaller value when compared to a \(100\times\) scaling, see their Fig.\ 1c.
\citet{timmreck2010} also obtain a peak \gls{rf} anomaly of
\(\SI{-18}{\watt\metre^{-2}}\) (\(7\)-\(8\) months after) compared to \citet{jones2005}
of \(\SI{-60}{\watt\metre^{-2}}\) (one year after). \gls{rf} peak occurring before the
\gls{aod} peak is the same as what has been found here for \gls{cesm2}, approximately
\(6\)--\(8\) months after the eruption \ref{fig:arrays_normalised}.

While the \gls{p100} data have been comparable to the large \gls{cesm2} data point in
both \gls{aod} and \gls{rf}, the temperature response reported by \citet{jones2005} is
much stronger than what our strongest eruption produced. Since \citet{jones2005} used
the \gls{aod} forcing from Mt.\ Pinatubo multiplied by one hundred, the shape of the
time series could be notably different to what a super-eruption would create, even
though the peak is representable, as we have seen in \ref{fig:arrays_normalised}a
where the \gls{aod} shape differ as the eruption strength is increased. This might be
enough to cause a much stronger temperature perturbation. We also note that
\citet{timmreck2010} used a more realistic \gls{aod} forcing and obtained cooling that
was much smaller than \citet{jones2005}.

A bigger spread is nevertheless found when converting from \iso{} to any of the three
output parameters when comparing across models. Conversion from \iso{} to \gls{aod} is
consistent within similar models, even when comparing simulations of volcanic eruptions
\citep{timmreck2010} and continuous injection of \ce{SO2} \citep{niemeier2015}, but has
a wide spread at large values of \iso{} across models
(\ref{fig:parameter_scan}a,b,c). Compared to \gls{aod} as a
function of \iso{}, both \gls{rf} (\ref{fig:parameter_scan}d) and temperature
(\ref{fig:parameter_scan}e) as a function of \gls{aod} result in a small spread in the data
across models, and necessarily the spread will also be small for temperature as a
function of \gls{rf} (\ref{fig:parameter_scan}f). A roughly linear relationship between
\gls{aod} and \gls{rf} have been assumed is several previous studies, but (1) for
smaller values of \gls{aod} and \gls{rf}, and (2) where the estimated slope is
significantly steeper; around \(\SI{-20}{\watt\metre^{-2}\mathrm{AOD}^{-1}}\) at
\(\mathrm{AOD}<1\) rather than \(\SI{-5}{\watt\metre^{-2}\mathrm{AOD}^{-1}}\) found here
at \(\mathrm{AOD}>1\). We therefore expect a significantly shallower slope to be a more
precise estimate of the ratio between \gls{rf} and \gls{aod} for eruptions one magnitude
or more greater than Mt.\ Pinatubo.

From the \gls{c2wm} case and the \gls{c2wmp} case, an apparent roughly linear
relationship between \gls{aod} and \gls{rf} emerge in \ref{fig:aod_vs_toa_ses_avg}.
There is, however, some spread in the data in particular of the \gls{c2ws} case.
Importantly, there appear to be a time-after-eruption dependency on the \gls{rf} to
\gls{aod} ratio. \citet{marshall2020} discuss a similar feature they find when looking
at \gls{aod} and \gls{rf}, but there, the efficiency increases from year 1 to year 2,
while here efficiency decreases with time, as shown in
\ref{fig:aod_vs_toa_avg_loop_ratios}. \citet{marshall2020} explain that this is due to
the time it takes for the aerosols to spread which affects the global albedo and in turn
the \gls{rf}, with the \gls{aod} being less affected by the aerosols spreading.
\citet{marshall2019, marshall2020, marshall2021} used a code with seven log-normal modes
to simulate aerosol mass and number concentrations, but their model is run in an
atmosphere-only configuration with prescribed sea surface temperatures and sea ice
extent \citep{marshall2019}, as opposed to the \gls{cesm2} which is run as an \gls{esm}.
The configuration used by \citet{marshall2019} is from the model UM-UKCA, which is an
extended version of the HadGEM3 \citep{dhomse2014}, which in turn is from a different
family of models than both \gls{cesm} and ECHAM5/MIP-ESM (used by
\citet{timmreck2010,niemeier2015}), but an ancestor of HadCM3 (used by
\citet{gregory2016}) \citep{kuma2023}.

Smaller eruptions and estimates from them give an \gls{rf} to \gls{aod} ratio that is
quite large (\(\sim \SI{-20}{\watt\metre^{-2}\mathrm{AOD}^{-1}}\)), while larger
eruptions (or yearly \iso{} in the atmosphere as used in \citet{niemeier2015}) result in
estimates that are smaller in magnitude (\(\sim
\SI{-10}{\watt\metre^{-2}\mathrm{AOD}^{-1}}\) to \(\sim
\SI{-5}{\watt\metre^{-2}\mathrm{AOD}^{-1}}\), as shown in
\ref{fig:aod_vs_toa_avg_loop_ratios}). \citet{niemeier2017} show that the efficiency
decrease as the injection rate increase, which is related to
% the \gls{qbo}, and
that larger volcanic eruptions results in larger aerosol particle sizes which in turn
result in a decreasing cooling efficiency per \iso{} \citep{english2013, timmreck2018}.
% \emph{Can this be used to reason that the efficiency is initially high, then decreasing?
%   Perhaps several and competing effects (albedo, \gls{qbo}, particle size, growth \&
%   sedimentation phases)?}

\subsection{Climate sensitivity estimate}

% NOTE: climate feedback/response parameter

The climate feedback parameter is estimated by \citet{jones2005} to be \(\alpha \simeq
\SI{4}{\watt\metre^{-2}\kelvin^{-1}}\)\footnote{Still not sure how they do this, you
  need constant forcing, no?}, which was more than twice (and half the sensitivity, since
\(\alpha =1/s\) where \(s\) is the climate sensitivity parameter) of what
\citet{gregory2016} obtained in their simulations using Pinatubo in the HadCM3 climate
model. (They used a step volcanic forcing simulation to obtain their estimate).

Instead of estimating \(\alpha \), we focus our attention to the climate resistance
\(\rho \), and the \gls{tcrp} \(1/\rho\) (where \(\mathrm{TCS}=F_{2\times}\times
\mathrm{TCRP}\) is the transient climate sensitivity). Even though the forcing from
volcanic eruptions last only for about a year, which is too short for the timescales at
which \(F=\rho T\) is valid \citep{gregory2016}, we may get around this by using a
time-integral form introduced by \citet{merlis2014}
\begin{equation}
  \int_0^{\tau}F \mathrm{d}t=\rho\int_{0}^{\tau}T \mathrm{d}t
\end{equation}
\begin{equation}
  \rho=\frac{\int_0^{\tau}F \mathrm{d}t}{\int_{0}^{\tau}T \mathrm{d}t}.
  \label{eq:climate-resistance}
\end{equation}
%
If the upper bound of the integral, \(\tau \), is big enough such that the upper ocean
heat capacity is the same at \(t=0\) and \(t=\tau \), then this agrees with \(F=\rho T\)
\citep{gregory2016} (\citet{merlis2014} used \(\tau =\SI{15}{\mathrm{y}}\)). We also
note that the climate resistance and the climate feedback parameter are related to the
ocean heat uptake efficiency \(\kappa \) through \(\rho =\alpha +\kappa \).

We estimate the climate resistance following the integral-form computation laid out in
\ref{eq:climate-resistance} and using \(\tau =\SI{8}{\mathrm{yr}}\), as this is as long
as our simulations run. The temperature has not quite reached back to equilibrium at
this time, as seen in \ref{fig:compare-waveform}. The three tropical simulation cases
(four in each ensemble) give estimated climate resistance \(\rho \) of \(\num{3.3(9)}\),
\(\num{3.12(9)}\) and \(\num{2.91(8)}\), and \gls{tcrp} (\(1/\rho\)) of
\(\num{0.32(7)}\), \(\num{0.321(9)}\) and \(\num{0.34(1)}\), shown in \ref{tab:trcp}.
The climate resistance parameter \(\rho\) is not the same as the climate feedback
parameter \citet{jones2005} estimated (\(\alpha\)), but as both \(\alpha \) and \(\kappa
\) are positive, and our estimates of \(\rho (=\alpha +\kappa) \) are all smaller than
the \citet{jones2005} estimate of \(\alpha \simeq
\SI{4}{\watt\metre^{-2}\kelvin^{-1}}\), we conclude that the climate feedback parameter
related to the simulations performed here must be significantly smaller than what
\citet{jones2005} got.

\begin{table}
  \centering

  \caption{Estimated climate resistance and \gls{tcrp} by use of the method outlined by
    \citet{merlis2014}. Estimates are based on ensembles with four members, and where \(\tau
    =\SI{9}{\mathrm{y}}\) in \ref{eq:climate-resistance}}\label{tab:trcp}%
  \begin{tabular}{ccc}
    Simulation type & \(\rho [\si{\watt\metre^{-2}\kelvin^{-1}}]\) & \(1/\rho\)         \\
    \gls{c2ws}      & \(\num{2.91(8)}\)                            & \(\num{0.34(1)}\)  \\
    \gls{c2wmp}     & \(\num{3.12(9)}\)                            & \(\num{0.321(9)}\) \\
    \gls{c2wm}      & \(\num{3.3(9)}\)                             & \(\num{0.32(7)}\)  \\
  \end{tabular}
\end{table}

\section{Conclusions}\label{sec:conclusions}

In this paper we considered three large to super-volcano sized eruptions. We investigate
the \gls{rf} as a function of \gls{aod} and look at their ratio, and find that the
\gls{rf} dependence of \(\sim\SI{-20}{\watt\metre^{-2}\mathrm{AOD}^{-1}}\) is consistent
with our results for eruptions of similar size in terms of \iso{} as Mt.\ Pinatubo.
Larger eruptions with one to two orders of magnitude more \iso{} is found to produce a
much more shallow gradient closer to \(\sim
\SI{-5}{\watt\metre^{-2}\mathrm{AOD}^{-1}}\). A more shallow gradient for larger
eruptions is also consistent with data from previous studies of super-volcanoes.

We find that there is generally hard to find a consistent conversion between \iso{} and
\gls{aod} that translates well between models, while from our simulations in the
\gls{cesm2} there is close to a linear relationship between \iso{} and \gls{aod}, with
both having a exponential decay across large injection magnitudes.

The time-after-eruption dependence of the ratio between \gls{rf} and \gls{aod} have been
reported before, but where the efficiency increased with time \citep{marshall2020}, that
is, \gls{rf} became relatively larger when compared with \gls{aod}. Our simulations
rather show a decrease in the aerosols cooling efficiency, where the \gls{aod} was
relatively larger than \gls{rf} later in the eruption phase.

% TODO: would smaller alpha give smaller or larger temp in Jones et al.?
Climate resistance estimates. \emph{Maybe the difference in climate resistance we get
  compared to what \citet{jones2005} used can be a pointer on that the small \gls{rf} to
  \gls{aod} ratio they got was too small even for such a large eruption.}

An interesting aspect concerning the similar decaying phase seen in both the \gls{aod}
time series and the \gls{rf} time series is the influence of these decaying phases on
the temperature time series, perhaps inducing a trend. That is, do the temperature time
series have different shape in the rising phase, but similar shape in the decaying
phase? Allowing the simulations to run for at least twenty years to enable the tail to
be well resolved is therefore another avenue that could be explored further.

\clearpage
%%%%%%%%%%%%%%%%%%%%%%%%%%%%%%%%%%%%%%%%%%%%%%%%%%%%%%%%%%%%%%%%%%%%%
% ACKNOWLEDGMENTS
%%%%%%%%%%%%%%%%%%%%%%%%%%%%%%%%%%%%%%%%%%%%%%%%%%%%%%%%%%%%%%%%%%%%%
\acknowledgments
%  Keep acknowledgments (note correct spelling: no ``e'' between the ``g'' and
% ``m'') as brief as possible. In general, acknowledge only direct help in
%  writing or research. Financial support (e.g., grant numbers) for the work done, 
%  for an author, or for the laboratory where the work was performed must be 
%  acknowledged here rather than as footnotes to the title or to an author's name.
%  Contribution numbers (if the work has been published by the author's institution 
%  or organization) should be placed in the acknowledgments rather than as 
%  footnotes to the title or to an author's name.

%%%%%%%%%%%%%%%%%%%%%%%%%%%%%%%%%%%%%%%%%%%%%%%%%%%%%%%%%%%%%%%%%%%%%
% DATA AVAILABILITY STATEMENT
%%%%%%%%%%%%%%%%%%%%%%%%%%%%%%%%%%%%%%%%%%%%%%%%%%%%%%%%%%%%%%%%%%%%%
% 
%
\datastatement
%  The data availability statement is where authors should describe how the data underlying 
%  the findings within the article can be accessed and reused. Authors should attempt to 
%  provide unrestricted access to all data and materials underlying reported findings. 
%  If data access is restricted, authors must mention this in the statement. See
%  {http://www.ametsoc.org/PubsDataPolicy} for more info.

%%%%%%%%%%%%%%%%%%%%%%%%%%%%%%%%%%%%%%%%%%%%%%%%%%%%%%%%%%%%%%%%%%%%%
% APPENDIXES
%%%%%%%%%%%%%%%%%%%%%%%%%%%%%%%%%%%%%%%%%%%%%%%%%%%%%%%%%%%%%%%%%%%%%
%
%% If only one appendix, use

%\appendix

%% If more than one appendix, use \appendix[<letter>], e.g.,

%\appendix[A] 

%% Appendix title is necessary! For appendix title:

%\appendixtitle{Title of Appendix}

%%% Appendix section numbering (note, skip \section and begin with \subsection)
%
% \subsection{First primary heading}

% \subsubsection{First secondary heading}

% \paragraph{First tertiary heading}

\appendix

\appendixtitle{Otto-Bliesner data analysis}\label{app:ob16}

Data from \citet{ottobliesner2016} are the original input data of \iso{} as used in
their model simulations, where corresponding \gls{rf} and temperature data are found as
the value of the time series at the time of an eruption (according to the \iso{} time
series). Therefore, \gls{rf} and temperature values may be somewhat smaller in
\ref{fig:parameter_scan}b,c,f than their true value.
Specifically, an ensemble of 5 is used for both \gls{rf} and temperature, and a mean
from the 5 is used as the de facto \gls{rf} and temperature time series. A control
simulation of a single time series is used to remove seasonal dependence from the
temperature, where the control simulation is averaged into a climatology mean. Further,
a drift in the temperature is removed by subtracting a linear regression fit. \gls{rf}
have seasonality removed in the Fourier domain. The forcing (\ce{SO2}) can be downloaded
with direct link
\url{https://svn-ccsm-inputdata.cgd.ucar.edu/trunk/inputdata/atm/cam/volc/IVI2LoadingLatHeight501-2000_L18_c20100518.nc},
or found at \url{https://www.cesm.ucar.edu/working-groups/paleo/simulations/ccsm4-lm}
and \url{https://svn-ccsm-inputdata.cgd.ucar.edu/trunk/inputdata/atm/cam/volc/}

%%%%%%%%%%%%%%%%%%%%%%%%%%%%%%%%%%%%%%%%%%%%%%%%%%%%%%%%%%%%%%%%%%%%%
% REFERENCES
%%%%%%%%%%%%%%%%%%%%%%%%%%%%%%%%%%%%%%%%%%%%%%%%%%%%%%%%%%%%%%%%%%%%%
% Make your BibTeX bibliography by using these commands:
% \bibliographystyle{ametsocV6}
% \bibliography{references}

\bibliographystyle{ametsocV6} \bibliography{references} \clearpage
\printglossary[type=\acronymtype,title=List of Acronyms]

\end{document}
